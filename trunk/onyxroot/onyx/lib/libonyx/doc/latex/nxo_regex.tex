%-*-mode:latex-*-
%%%%%%%%%%%%%%%%%%%%%%%%%%%%%%%%%%%%%%%%%%%%%%%%%%%%%%%%%%%%%%%%%%%%%%%%%%%%%
%
% <Copyright = jasone>
% <License>
%
%%%%%%%%%%%%%%%%%%%%%%%%%%%%%%%%%%%%%%%%%%%%%%%%%%%%%%%%%%%%%%%%%%%%%%%%%%%%%
%
% Version: Onyx <Version = onyx>
%
% nxo_regex portion of Onyx Manual.
%              
%%%%%%%%%%%%%%%%%%%%%%%%%%%%%%%%%%%%%%%%%%%%%%%%%%%%%%%%%%%%%%%%%%%%%%%%%%%%%

\subsection{nxo\_regex}
\label{nxo_regex}
\index{nxo_regex@\classname{nxo\_regex}{}}

The \classname{nxo\_regex} class is a subclass of the \classname{nxo} class.

\subsubsection{API}
\begin{capi}
\label{nxo_regex_new}
\index{nxo_regex_new@\cfunc{nxo\_regex\_new}{}}
\citem{\cfunc[cw\_nxn\_t]{nxo\_regex\_new}{cw\_nxo\_t *a\_nxo, cw\_nx\_t *a\_nx,
const cw\_uint8\_t *a\_pattern, cw\_uint32\_t a\_len, cw\_bool\_t
a\_insensitive, cw\_bool\_t a\_multiline, cw\_bool\_t a\_singleline,
cw\_uint32\_t a\_limit}}
	\begin{capilist}
	\item[Input(s): ]
		\begin{description}\item[]
		\item[a\_nxo: ]
			Pointer to a regex \classname{nxo}.
		\item[a\_nx: ]
			Pointer to an \classname{nx}.
		\item[a\_pattern: ]
			Pointer to a string that specifies a regular expression.
		\item[a\_len: ]
			Length of \cvar{a\_str}.
		\item[a\_insensitive: ]
			Match with case insensitivity if TRUE.
		\item[a\_multiline: ]
			Treat input as a multi-line string if TRUE.
		\item[a\_singleline: ]
			Treat input as a single line, so that the dot
			metacharacter matches any character, including a
			newline.
		\item[a\_limit: ]
			Maximum number of matches to allow (0 means unlimited).
		\end{description}
	\item[Output(s): ]
		\begin{description}\item[]
		\item[retval: ]
			\begin{description}\item[]
			\item[NXN\_ZERO: ] Success.
			\item[NXN\_regexerror: ] Regular expression error.
			\end{description}
		\end{description}
	\item[Exception(s): ]
		\begin{description}\item[]
		\item[\htmlref{CW\_ONYXX\_OOM}{CW_ONYXX_OOM}.]
		\end{description}
	\item[Description: ]
		Constructor.
	\end{capilist}
\label{nxo_regex_match}
\index{nxo_regex_match@\cfunc{nxo\_regex\_match}{}}
\citem{\cfunc[void]{nxo\_regex\_match}{cw\_nxo\_t *a\_nxo, cw\_nxo\_t 
*a\_thread, cw\_nxo\_t *a\_input, cw\_nxo\_t *r\_matches}}
	\begin{capilist}
	\item[Input(s): ]
		\begin{description}\item[]
		\item[a\_nxo: ]
			Pointer to a regex \classname{nxo}.
		\item[a\_thread: ]
			Pointer to a thread \classname{nxo}.
		\item[a\_input: ]
			Pointer to a string \classname{nxo}.
		\item[r\_matches: ]
			Pointer to an \classname{nxo} to dup the resulting
			matches array to.
		\end{description}
	\item[Output(s): ]
		\begin{description}\item[]
		\item[r\_matches: ]
			Pointer to an array \classname{nxo} that contains an
			element for each match in \cvar{a\_nxo}.
		\end{description}
	\item[Exception(s): ]
		\begin{description}\item[]
		\item[\htmlref{CW\_ONYXX\_OOM}{CW_ONYXX_OOM}.]
		\end{description}
	\item[Description: ]
		Find substrings in \cvar{a\_input} that match the regex pointed
		to by \cvar{a\_nxo} and create an array of the matching
		substrings.
	\end{capilist}
\label{nxo_regex_new}
\index{nxo_regex_nonew_match@\cfunc{nxo\_regex\_nonew\_match}{}}
\citem{\cfunc[cw\_nxn\_t]{nxo\_regex\_nonew\_match}{cw\_nxo\_t *a\_thread, const
cw\_uint8\_t *a\_pattern, cw\_uint32\_t a\_len, cw\_bool\_t a\_insensitive,
cw\_bool\_t a\_multiline, cw\_bool\_t a\_singleline, cw\_uint32\_t a\_limit,
cw\_nxo\_t *a\_input, cw\_nxo\_t *r\_matches}}
	\begin{capilist}
	\item[Input(s): ]
		\begin{description}\item[]
		\item[a\_thread: ]
			Pointer to a thread \classname{nxo}.
		\item[a\_pattern: ]
			Pointer to a string that specifies a regular expression.
		\item[a\_len: ]
			Length of \cvar{a\_str}.
		\item[a\_insensitive: ]
			Match with case insensitivity if TRUE.
		\item[a\_multiline: ]
			Treat input as a multi-line string if TRUE.
		\item[a\_singleline: ]
			Treat input as a single line, so that the dot
			metacharacter matches any character, including a
			newline.
		\item[a\_limit: ]
			Maximum number of matches to allow (0 means unlimited).
		\item[a\_input: ]
			Pointer to a string \classname{nxo}.
		\item[r\_matches: ]
			Pointer to an \classname{nxo} to dup the resulting
			matches array to.
		\end{description}
	\item[Output(s): ]
		\begin{description}\item[]
		\item[retval: ]
			\begin{description}\item[]
			\item[NXN\_ZERO: ] Success.
			\item[NXN\_regexerror: ] Regular expression error.
			\end{description}
		\item[r\_matches: ]
			Pointer to an array \classname{nxo} that contains an
			element for each match in \cvar{a\_nxo}.
		\end{description}
	\item[Exception(s): ]
		\begin{description}\item[]
		\item[\htmlref{CW\_ONYXX\_OOM}{CW_ONYXX_OOM}.]
		\end{description}
	\item[Description: ]
		Find substrings in \cvar{a\_input} that match the pattern and
		create an array of the matching substrings.  This function
		combines \cfunc{nxo\_regex\_new}{} and
		\cfunc{nxo\_regex\_match}{} in such a way that no Onyx object is
		created, thus providing a more efficient way of doing a one-off
		match.
	\end{capilist}
\end{capi}

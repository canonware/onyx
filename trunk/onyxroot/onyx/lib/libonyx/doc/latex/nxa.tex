%-*-mode:latex-*-
%%%%%%%%%%%%%%%%%%%%%%%%%%%%%%%%%%%%%%%%%%%%%%%%%%%%%%%%%%%%%%%%%%%%%%%%%%%%%
%
% <Copyright = jasone>
% <License>
%
%%%%%%%%%%%%%%%%%%%%%%%%%%%%%%%%%%%%%%%%%%%%%%%%%%%%%%%%%%%%%%%%%%%%%%%%%%%%%
%
% Version: <Version>
%
% nxa portion of Canonware Software Manual.
%              
%%%%%%%%%%%%%%%%%%%%%%%%%%%%%%%%%%%%%%%%%%%%%%%%%%%%%%%%%%%%%%%%%%%%%%%%%%%%%

\subsection{nxa}
\label{nxa}
\index{\classname{nxa}{}}

The \classname{nxa} class implements garbage collection.  The garbage collector
runs in a separate thread and is controlled via an asynchronous message queue.

\subsubsection{API}
\begin{capi}
\label{nxa_new}
\index{\cfunc{nxa\_new}{}}
\citem{\cfunc[void]{nxa\_new}{cw\_nxa\_t *a\_nxa, cw\_nx\_t *a\_nx}}
	\begin{capilist}
	\item[Input(s): ]
		\begin{description}\item[]
		\item[a\_nxa: ]
			Pointer to a \classname{nxa}.
		\item[a\_nx: ]
			Pointer to a \classname{nx}.
		\end{description}
	\item[Output(s): ] None.
	\item[Exception(s): ]
		\begin{description}\item[]
		\item[\htmlref{\_CW\_STASHX\_OOM}{_CW_STASHX_OOM}.]
		\end{description}
	\item[Description: ]
		Constructor.
	\end{capilist}
\label{nxa_delete}
\index{\cfunc{nxa\_delete}{}}
\citem{\cfunc[void]{nxa\_delete}{cw\_nxa\_t *a\_nxa}}
	\begin{capilist}
	\item[Input(s): ]
		\begin{description}\item[]
		\item[a\_nxa: ]
			Pointer to a \classname{nxa}.
		\end{description}
	\item[Output(s): ] None.
	\item[Exception(s): ] None.
	\item[Description: ]
		Destructor.
	\end{capilist}
\label{nxa_collect}
\index{\cfunc{nxa\_collect}{}}
\citem{\cfunc[void]{nxa\_collect}{cw\_nxa\_t *a\_nxa}}
	\begin{capilist}
	\item[Input(s): ]
		\begin{description}\item[]
		\item[a\_nxa: ]
			Pointer to a \classname{nxa}.
		\end{description}
	\item[Output(s): ] None.
	\item[Exception(s): ]
		\begin{description}\item[]
		\item[\htmlref{\_CW\_STASHX\_OOM}{_CW_STASHX_OOM}.]
		\end{description}
	\item[Description: ]
		Send a message to the garbage collector thread to do a
		collection.
	\end{capilist}
\label{nxa_dump}
\index{\cfunc{nxa\_dump}{}}
\citem{\cfunc[void]{nxa\_dump}{cw\_nxa\_t *a\_nxa, cw\_nxo\_t *a\_thread}}
	\begin{capilist}
	\item[Input(s): ]
		\begin{description}\item[]
		\item[a\_nxa: ]
			Pointer to a \classname{nxa}.
		\item[a\_thread: ]
			Pointer to a thread \classname{nxo}.
		\end{description}
	\item[Output(s): ]
		Output printed to \onyxop{stdout}{}.
	\item[Exception(s): ]
		\begin{description}\item[]
		\item[\htmlref{\_CW\_STASHX\_OOM}{_CW_STASHX_OOM}.]
		\end{description}
	\item[Description: ]
		Print the internal state of gcdict to \onyxop{stdout}{}.
	\end{capilist}
\label{nxa_active_get}
\index{\cfunc{nxa\_active\_get}{}}
\citem{\cfunc[cw\_bool\_t]{nxa\_active\_get}{cw\_nxa\_t *a\_nxa}}
	\begin{capilist}
	\item[Input(s): ]
		\begin{description}\item[]
		\item[a\_nxa: ]
			Pointer to a \classname{nxa}.
		\end{description}
	\item[Output(s): ]
		\begin{description}\item[]
		\item[retval: ]
			\begin{description}\item[]
			\item[FALSE: ]
				Garbage collector deactivated.
			\item[TRUE: ]
				Garbage collector active.
			\end{description}
		\end{description}
	\item[Exception(s): ] None.
	\item[Description: ]
		Return whether the garbage collector is active (runnable).
	\end{capilist}
\label{nxa_active_set}
\index{\cfunc{nxa\_active\_set}{}}
\citem{\cfunc[void]{nxa\_active\_set}{cw\_nxa\_t *a\_nxa, cw\_bool\_t
a\_active}}
	\begin{capilist}
	\item[Input(s): ]
		\begin{description}\item[]
		\item[a\_nxa: ]
			Pointer to a \classname{nxa}.
		\item[a\_active: ]
			\begin{description}\item[]
			\item[FALSE: ]
				Deactivate garbage collector.
			\item[TRUE: ]
				Activate garbage collector.
			\end{description}
		\end{description}
	\item[Output(s): ] None.
	\item[Exception(s): ]
		\begin{description}\item[]
		\item[\htmlref{\_CW\_STASHX\_OOM}{_CW_STASHX_OOM}.]
		\end{description}
	\item[Description: ]
		Send a message to the garbage collector to activate or
		deactivate.  The asynchronous nature of the message means that
		it is possible for the garbage collector to run after this
		function returns, even if a deactivation message has been sent.
	\end{capilist}
\label{nxa_period_get}
\index{\cfunc{nxa\_period\_get}{}}
\citem{\cfunc[cw\_nxoi\_t]{nxa\_period\_get}{cw\_nxa\_t *a\_nxa}}
	\begin{capilist}
	\item[Input(s): ]
		\begin{description}\item[]
		\item[a\_nxa: ]
			Pointer to a \classname{nxa}.
		\end{description}
	\item[Output(s): ]
		\begin{description}\item[]
		\item[retval: ]
			Current inactivity period in seconds that the garbage
			collector waits before doing a collection.
		\end{description}
	\item[Exception(s): ] None.
	\item[Description: ]
		Return the current inactivity period in seconds that the garbage
		collector waits before doing a collection.
	\end{capilist}
\label{nxa_period_set}
\index{\cfunc{nxa\_period\_set}{}}
\citem{\cfunc[void]{nxa\_period\_set}{cw\_nxa\_t *a\_nxa, cw\_nxoi\_t
a\_period}}
	\begin{capilist}
	\item[Input(s): ]
		\begin{description}\item[]
		\item[a\_nxa: ]
			Pointer to a \classname{nxa}.
		\item[a\_period: ]
			Inactivity period in seconds that the garbage collector
			should wait before doing a collection.  If 0, the
			garbage collector will never run due to inactivity.
		\end{description}
	\item[Output(s): ] None.
	\item[Exception(s): ]
		\begin{description}\item[]
		\item[\htmlref{\_CW\_STASHX\_OOM}{_CW_STASHX_OOM}.]
		\end{description}
	\item[Description: ]
		Set the inactivity period in seconds that the garbage collector
		should wait before doing a collection.
	\end{capilist}
\label{nxa_threshold_get}
\index{\cfunc{nxa\_threshold\_get}{}}
\citem{\cfunc[cw\_nxoi\_t]{nxa\_threshold\_get}{cw\_nxa\_t *a\_nxa}}
	\begin{capilist}
	\item[Input(s): ]
		\begin{description}\item[]
		\item[a\_nxa: ]
			Pointer to a \classname{nxa}.
		\end{description}
	\item[Output(s): ]
		\begin{description}\item[]
		\item[retval: ]
			Number of objects allocated since the last garbage
			collection that will trigger the garbage collector to
			run.
		\end{description}
	\item[Exception(s): ] None.
	\item[Description: ]
		Return the number of objects allocated since the last garbage
		collection that will trigger the garbage collector to run.
	\end{capilist}
\label{nxa_threshold_set}
\index{\cfunc{nxa\_threshold\_set}{}}
\citem{\cfunc[void]{nxa\_threshold\_set}{cw\_nxa\_t *a\_nxa, cw\_nxoi\_t
a\_threshold}}
	\begin{capilist}
	\item[Input(s): ]
		\begin{description}\item[]
		\item[a\_nxa: ]
			Pointer to a \classname{nxa}.
		\item[a\_threshold: ]
			The number of objects allocated since the last garbage
			collection that will trigger the garbage collector to
			run.
		\end{description}
	\item[Output(s): ] None.
	\item[Exception(s): ]
		\begin{description}\item[]
		\item[\htmlref{\_CW\_STASHX\_OOM}{_CW_STASHX_OOM}.]
		\end{description}
	\item[Description: ]
		Set the number of objects allocated since the last garbage
		collection that will trigger the garbage collector to run.
	\end{capilist}
\label{nxa_collections_get}
\index{\cfunc{nxa\_collections\_get}{}}
\citem{\cfunc[cw\_nxoi\_t]{nxa\_collections\_get}{cw\_nxa\_t *a\_nxa}}
	\begin{capilist}
	\item[Input(s): ]
		\begin{description}\item[]
		\item[a\_nxa: ]
			Pointer to a \classname{nxa}.
		\end{description}
	\item[Output(s): ]
		\begin{description}\item[]
		\item[retval: ]
			The number of times the garbage collector has run.
		\end{description}
	\item[Exception(s): ] None.
	\item[Description: ]
		Return the number of times the garbage collector has run.
	\end{capilist}
\label{nxa_new_get}
\index{\cfunc{nxa\_new\_get}{}}
\citem{\cfunc[cw\_nxoi\_t]{nxa\_new\_get}{cw\_nxa\_t *a\_nxa}}
	\begin{capilist}
	\item[Input(s): ]
		\begin{description}\item[]
		\item[a\_nxa: ]
			Pointer to a \classname{nxa}.
		\end{description}
	\item[Output(s): ]
		\begin{description}\item[]
		\item[retval: ]
			The number of objects that have been allocated since the
			last garbage collection.
		\end{description}
	\item[Exception(s): ] None.
	\item[Description: ]
		Return the numbber of objects that have been allocated since the
		last garbage collection.
	\end{capilist}
\label{nxa_current_get}
\index{\cfunc{nxa\_current\_get}{}}
\citem{\cfunc[void]{nxa\_current\_get}{cw\_nxa\_t *a\_nxa}}
	\begin{capilist}
	\item[Input(s): ]
		\begin{description}\item[]
		\item[a\_nxa: ]
			Pointer to a \classname{nxa}.
		\item[r\_count: ]
			Pointer to an integer.
		\item[r\_mark: ]
			Pointer to an integer.
		\item[r\_sweep: ]
			Pointer to an integer.
		\end{description}
	\item[Output(s): ]
		\begin{description}\item[]
		\item[*r\_count: ]
			Number of objects.
		\item[*r\_mark: ]
			Number of microseconds spent in the mark phase of
			garbage collection.
		\item[*r\_sweep: ]
			Number of microseconts spent in the sweep phase of
			garbage collection.
		\end{description}
	\item[Exception(s): ] None.
	\item[Description: ]
		Return the current garbage collector statistics.
		\cvar{*r\_count} is the current number of composite objects,
		whereas \cvar{*r\_mark} and \cvar{*r\_sweep} are times for the
		most recent garbage collection.
	\end{capilist}
\label{nxa_maximum_get}
\index{\cfunc{nxa\_maximum\_get}{}}
\citem{\cfunc[void]{nxa\_maximum\_get}{cw\_nxa\_t *a\_nxa}}
	\begin{capilist}
	\item[Input(s): ]
		\begin{description}\item[]
		\item[a\_nxa: ]
			Pointer to a \classname{nxa}.
		\item[r\_count: ]
			Pointer to an integer.
		\item[r\_mark: ]
			Pointer to an integer.
		\item[r\_sweep: ]
			Pointer to an integer.
		\end{description}
	\item[Output(s): ]
		\begin{description}\item[]
		\item[*r\_count: ]
			Number of objects.
		\item[*r\_mark: ]
			Number of microseconds spent in the mark phase of
			garbage collection.
		\item[*r\_sweep: ]
			Number of microseconts spent in the sweep phase of
			garbage collection.
		\end{description}
	\item[Exception(s): ] None.
	\item[Description: ]
		Return the maximum garbage collector statistics.
		\cvar{*r\_count} is the maximum number of composite objects that
		have ever existed. \cvar{*r\_mark} and \cvar{*r\_sweep} are
		times for the longest recent garbage collection phases.
	\end{capilist}
\label{nxa_sum_get}
\index{\cfunc{nxa\_sum\_get}{}}
\citem{\cfunc[void]{nxa\_sum\_get}{cw\_nxa\_t *a\_nxa}}
	\begin{capilist}
	\item[Input(s): ]
		\begin{description}\item[]
		\item[a\_nxa: ]
			Pointer to a \classname{nxa}.
		\item[r\_count: ]
			Pointer to an integer.
		\item[r\_mark: ]
			Pointer to an integer.
		\item[r\_sweep: ]
			Pointer to an integer.
		\end{description}
	\item[Output(s): ]
		\begin{description}\item[]
		\item[*r\_count: ]
			Number of objects.
		\item[*r\_mark: ]
			Number of microseconds spent in the mark phase of
			garbage collection.
		\item[*r\_sweep: ]
			Number of microseconts spent in the sweep phase of
			garbage collection.
		\end{description}
	\item[Exception(s): ] None.
	\item[Description: ]
		Return the sum garbage collector statistics.
		\cvar{*r\_count} is the total number of composite objects that
		have ever been created.  \cvar{*r\_mark} and \cvar{*r\_sweep}
		are the total times spent in garbage collection.
	\end{capilist}
\label{nxa_gcdict_get}
\index{\cppmacro{nxa\_gcdict\_get}{}}
\citem{\cppmacro[cw\_nxo\_t *]{nxa\_gcdict\_get}{cw\_nxa\_t *a\_nxa}}
	\begin{capilist}
	\item[Input(s): ]
		\begin{description}\item[]
		\item[a\_nxa: ]
			Pointer to a \classname{nxa}.
		\end{description}
	\item[Output(s): ]
		\begin{description}\item[]
		\item[retval: ]
			Pointer to a dict \classname{nxo}.
		\end{description}
	\item[Exception(s): ] None.
	\item[Description: ]
		Return a pointer to the dict \classname{nxo} corresponding to
		\onyxop{gcdict}{}.
	\end{capilist}
\end{capi}

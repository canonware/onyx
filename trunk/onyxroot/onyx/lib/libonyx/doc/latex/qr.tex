%-*-mode:latex-*-
%%%%%%%%%%%%%%%%%%%%%%%%%%%%%%%%%%%%%%%%%%%%%%%%%%%%%%%%%%%%%%%%%%%%%%%%%%%%%
%
% <Copyright = jasone>
% <License>
%
%%%%%%%%%%%%%%%%%%%%%%%%%%%%%%%%%%%%%%%%%%%%%%%%%%%%%%%%%%%%%%%%%%%%%%%%%%%%%
%
% Version: Onyx <Version = onyx>
%
% qr portion of Onyx Manual.
%
%%%%%%%%%%%%%%%%%%%%%%%%%%%%%%%%%%%%%%%%%%%%%%%%%%%%%%%%%%%%%%%%%%%%%%%%%%%%%

\subsection{qr}
\label{qr}
\index{qr@\classname{qr}{}}

The \classname{qr} macros implement operations on a ring.  The type of the ring
elements and which field of the elements to use are determined by arguments that
are passed into the macros.  The macros are optimized for speed and code size,
which means that there is minimal error checking built in.  As a result, care
must be taken to assure that these are used as intended, or strange things can
happen.

\subsubsection{API}
\begin{capi}
%\label{qr}
\index{qr@\cppmacro{qr}{}}
\citem{\cppmacro[]{qr}{{\lt}qr\_type{\gt} a\_type}}
	\begin{capilist}
	\item[Input(s): ]
		\begin{description}\item[]
		\item[a\_type: ]
			Data type for the \classname{qr}.
		\end{description}
	\item[Output(s): ]
		A data structure that can be used for a \classname{qr}.
	\item[Exception(s): ] None.
	\item[Description: ]
		Generate code for a \classname{qr} data structure.
	\end{capilist}
\label{qr_new}
\index{qr_new@\cppmacro{qr\_new}{}}
\citem{\cppmacro[void]{qr\_new}{{\lt}qr\_type{\gt} *a\_qr, {\lt}field\_name{\gt}
a\_field}}
	\begin{capilist}
	\item[Input(s): ]
		\begin{description}\item[]
		\item[a\_qr: ]
			Pointer to a \classname{qr}.
		\item[a\_field: ]
			Field within the \classname{qr} elements to use.
		\end{description}
	\item[Output(s): ] None.
	\item[Exception(s): ] None.
	\item[Description: ]
		Constructor.
	\end{capilist}
\label{qr_next}
\index{qr_next@\cppmacro{qr\_next}{}}
\citem{\cppmacro[{\lt}qr\_type{\gt} *]{qr\_next}{{\lt}qr\_type{\gt} *a\_qr,
{\lt}field\_name{\gt} a\_field}}
	\begin{capilist}
	\item[Input(s): ]
		\begin{description}\item[]
		\item[a\_qr: ]
			Pointer to a \classname{qr}.
		\item[a\_field: ]
			Field within the \classname{qr} elements to use.
		\end{description}
	\item[Output(s): ]
		\begin{description}\item[]
		\item[retval: ]
			Pointer to the next element in the \classname{qr}.
		\end{description}
	\item[Exception(s): ] None.
	\item[Description: ]
		Return a pointer to the next element in the \classname{qr}.
	\end{capilist}
\label{qr_prev}
\index{qr_prev@\cppmacro{qr\_prev}{}}
\citem{\cppmacro[{\lt}qr\_type{\gt} *]{qr\_prev}{{\lt}qr\_type{\gt} *a\_qr,
{\lt}field\_name{\gt} a\_field}}
	\begin{capilist}
	\item[Input(s): ]
		\begin{description}\item[]
		\item[a\_qr: ]
			Pointer to a \classname{qr}.
		\item[a\_field: ]
			Field within the \classname{qr} elements to use.
		\end{description}
	\item[Output(s): ]
		\begin{description}\item[]
		\item[retval: ]
			Pointer to the previous element in the \classname{qr}.
		\end{description}
	\item[Exception(s): ] None.
	\item[Description: ]
		Return a pointer to the previous element in the \classname{qr}.
	\end{capilist}
\label{qr_before_insert}
\index{qr_before_insert@\cppmacro{qr\_before\_insert}{}}
\citem{\cppmacro[void]{qr\_before\_insert}{{\lt}qr\_type{\gt} *a\_qrelm,
{\lt}qr\_type{\gt} *a\_qr, {\lt}field\_name{\gt} a\_field}}
	\begin{capilist}
	\item[Input(s): ]
		\begin{description}\item[]
		\item[a\_qrelm: ]
			Pointer to an element in a \classname{qr}.
		\item[a\_qr: ]
			Pointer to an element that is the only element in its
			ring.
		\item[a\_field: ]
			Field within the \classname{qr} elements to use.
		\end{description}
	\item[Output(s): ] None.
	\item[Exception(s): ] None.
	\item[Description: ]
		Insert \cvar{a\_qr} before \cvar{a\_qrelm}.
	\end{capilist}
\label{qr_after_insert}
\index{qr_after_insert@\cppmacro{qr\_after\_insert}{}}
\citem{\cppmacro[void]{qr\_after\_insert}{{\lt}qr\_type{\gt} *a\_qrelm,
{\lt}qr\_type{\gt} *a\_qr, {\lt}field\_name{\gt} a\_field}}
	\begin{capilist}
	\item[Input(s): ]
		\begin{description}\item[]
		\item[a\_qrelm: ]
			Pointer to an element in a \classname{qr}.
		\item[a\_qr: ]
			Pointer to an element that is the only element in its
			ring.
		\item[a\_field: ]
			Field within the \classname{qr} elements to use.
		\end{description}
	\item[Output(s): ] None.
	\item[Exception(s): ] None.
	\item[Description: ]
		Insert \cvar{a\_qr} after \cvar{a\_qrelm}.
	\end{capilist}
\label{qr_meld}
\index{qr_meld@\cppmacro{qr\_meld}{}}
\citem{\cppmacro[void]{qr\_meld}{{\lt}qr\_type{\gt} *a\_qr\_a,
{\lt}qr\_type{\gt} *a\_qr\_b, {\lt}qr\_type{\gt} a\_type, {\lt}field\_name{\gt}
a\_field}}
	\begin{capilist}
	\item[Input(s): ]
		\begin{description}\item[]
		\item[a\_qr\_a: ]
			Pointer to a \classname{qr}.
		\item[a\_qr\_b: ]
			Pointer to a \classname{qr}.
		\item[a\_type: ]
			Data type for the \classname{qr} elements.
		\item[a\_field: ]
			Field within the \classname{qr} elements to use.
		\end{description}
	\item[Output(s): ] None.
	\item[Exception(s): ] None.
	\item[Description: ]
		Meld \cvar{a\_qr\_a} and \cvar{a\_qr\_b} into one ring.
	\end{capilist}
\label{qr_split}
\index{qr_split@\cppmacro{qr\_split}{}}
\citem{\cppmacro[void]{qr\_split}{{\lt}qr\_type{\gt} *a\_qr\_a,
{\lt}qr\_type{\gt} *a\_qr\_b, {\lt}qr\_type{\gt} a\_type, {\lt}field\_name{\gt}
a\_field}}
	\begin{capilist}
	\item[Input(s): ]
		\begin{description}\item[]
		\item[a\_qr\_a: ]
			Pointer to a \classname{qr}.
		\item[a\_qr\_b: ]
			Pointer to a \classname{qr}.
		\item[a\_type: ]
			Data type for the \classname{qr} elements.
		\item[a\_field: ]
			Field within the \classname{qr} elements to use.
		\end{description}
	\item[Output(s): ] None.
	\item[Exception(s): ] None.
	\item[Description: ]
		Split a ring at \cvar{a\_qr\_a} and \cvar{a\_qr\_b}.
	\end{capilist}
\label{qr_remove}
\index{qr_remove@\cppmacro{qr\_remove}{}}
\citem{\cppmacro[void]{qr\_remove}{{\lt}qr\_type{\gt} *a\_qr,
{\lt}field\_name{\gt} a\_field}}
	\begin{capilist}
	\item[Input(s): ]
		\begin{description}\item[]
		\item[a\_qr: ]
			Pointer to a \classname{qr}.
		\item[a\_field: ]
			Field within the \classname{qr} elements to use.
		\end{description}
	\item[Output(s): ] None.
	\item[Exception(s): ] None.
	\item[Description: ]
		Remove \cvar{a\_qr} from the ring.
	\end{capilist}
\label{qr_foreach}
\index{qr_foreach@\cppmacro{qr\_foreach}{}}
\citem{\cppmacro[]{qr\_foreach}{{\lt}qr\_type{\gt} *a\_var, {\lt}qr\_type{\gt}
*a\_qr, {\lt}field\_name{\gt} a\_field}}
	\begin{capilist}
	\item[Input(s): ]
		\begin{description}\item[]
		\item[a\_var: ]
			The name of a temporary variable to use for iteration.
		\item[a\_qr: ]
			Pointer to a \classname{qr}.
		\item[a\_field: ]
			Field within the \classname{qr} elements to use.
		\end{description}
	\item[Output(s): ] None.
	\item[Exception(s): ] None.
	\item[Description: ]
		Iterate through the \classname{qr}, storing a pointer to each
		element in \cvar{a\_var} along the way.
	\end{capilist}
\label{qr_reverse_foreach}
\index{qr_reverse_foreach@\cppmacro{qr\_reverse\_foreach}{}}
\citem{\cppmacro[]{qr\_reverse\_foreach}{{\lt}qr\_type{\gt} *a\_var,
{\lt}qr\_type{\gt} *a\_qr, {\lt}field\_name{\gt} a\_field}}
	\begin{capilist}
	\item[Input(s): ]
		\begin{description}\item[]
		\item[a\_var: ]
			The name of a temporary variable to use for iteration.
		\item[a\_qr: ]
			Pointer to a \classname{qr}.
		\item[a\_field: ]
			Field within the \classname{qr} elements to use.
		\end{description}
	\item[Output(s): ] None.
	\item[Exception(s): ] None.
	\item[Description: ]
		Iterate through the \classname{qr} in the reverse direction,
		storing a pointer to each element in \cvar{a\_var} along the
		way.
	\end{capilist}
\end{capi}

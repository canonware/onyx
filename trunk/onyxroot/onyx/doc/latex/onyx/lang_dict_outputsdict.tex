%-*-mode:latex-*-
%%%%%%%%%%%%%%%%%%%%%%%%%%%%%%%%%%%%%%%%%%%%%%%%%%%%%%%%%%%%%%%%%%%%%%%%%%%%%
%
% <Copyright = jasone>
% <License>
%
%%%%%%%%%%%%%%%%%%%%%%%%%%%%%%%%%%%%%%%%%%%%%%%%%%%%%%%%%%%%%%%%%%%%%%%%%%%%%
%
% Version: Onyx <Version = onyx>
%
% outputsdict reference portion of Onyx Manual.
%
%%%%%%%%%%%%%%%%%%%%%%%%%%%%%%%%%%%%%%%%%%%%%%%%%%%%%%%%%%%%%%%%%%%%%%%%%%%%%

\subsection{outputsdict}
\label{sec:outputsdict}
\index{outputsdict@\onyxop{}{outputsdict}{}}

The outputsdict dictionary is primarily used to support
\htmlref{\onyxop{}{outputs}{}}{systemdict:outputs}, but its contents may be of
use to an application that wishes to extend or modify formatted printing.

There is an entry in outputsdict for each Onyx type.  Each entry renders objects
that correspond to its name using optional flags stored in a dictionary.  The
following flags are supported for all types:
\begin{description}
\item[{\tt \$n}: ]
	Maximum length, in bytes.  Default: disabled.
\item[{\tt \$w}: ]
	Minimum length, in bytes.  Default: disabled.
\item[{\tt \$j}: ]
	Justification.  Legal values:
	\begin{description}
	\item[{\tt \$l}: ]
		Left.
	\item[{\tt \$c}: ]
		Center.
	\item[{\tt \$r}: ]
		Right (default).
	\end{description}
\item[{\tt \$p}: ]
	Padding character.  Default: {\tt ` '}.
\item[{\tt \$r}: ]
	Syntactic rendering recursion depth.  Default: 1.
\end{description}

The following additional flags are supported for integers:
\begin{description}
\item[{\tt \$b}: ]
	Base, from 2 to 36.  Default: 10.
\item[{\tt \$s}: ]
	Sign.  Legal values:
	\begin{description}
	\item[{\tt \$-}: ]
		Only print sign if output is negative (default).
	\item[{\tt \$+}: ]
		Always print sign.
	\end{description}
\end{description}

The following additional flags are supported for reals:
\begin{description}
\item[{\tt \$d}: ]
	Digits of precision past decimal point.  Default: 6.
\item[{\tt \$e}: ]
	Exponential notation, if true.  Default: false.
\end{description}

\begin{longtable}{\optableformat{4.10}}
\caption{outputsdict summary}
\\
\hline
\optableent
	{Input(s)}
	{Op/Proc/Var}
	{Output(s)}
	{Description}
\hline \hline
%begin{latexonly}
\endfirsthead
\caption[]{\emph{continued}} \\
\hline
\optableent
	{Input(s)}
	{Op/Proc/Var}
	{Output(s)}
	{Description}
\hline \hline \endhead
\multicolumn{2}{r}{\emph{Continued on next page...}} \endfoot
\hline \endlastfoot
%end{latexonly}
\optableent
	{array flags}
	{{\bf \htmlref{arraytype}{outputsdict:arraytype}}}
	{string}
	{Create formatted string from array.}
\hline
\optableent
	{boolean flags}
	{{\bf \htmlref{booleantype}{outputsdict:booleantype}}}
	{string}
	{Create formatted string from boolean.}
\hline
\optableent
	{class flags}
	{{\bf \htmlref{classtype}{outputsdict:classtype}}}
	{string}
	{Create formatted string from class.}
\hline
\optableent
	{condition flags}
	{{\bf \htmlref{conditiontype}{outputsdict:conditiontype}}}
	{string}
	{Create formatted string from condition.}
\hline
\optableent
	{dict flags}
	{{\bf \htmlref{dicttype}{outputsdict:dicttype}}}
	{string}
	{Create formatted string from dict.}
\hline
\optableent
	{file flags}
	{{\bf \htmlref{filetype}{outputsdict:filetype}}}
	{string}
	{Create formatted string from file.}
\hline
\optableent
	{fino flags}
	{{\bf \htmlref{finotype}{outputsdict:finotype}}}
	{string}
	{Create formatted string from fino.}
\hline
\optableent
	{handle flags}
	{{\bf \htmlref{handletype}{outputsdict:handletype}}}
	{string}
	{Create formatted string from handle.}
\hline
\optableent
	{instance flags}
	{{\bf \htmlref{instancetype}{outputsdict:instancetype}}}
	{string}
	{Create formatted string from instance.}
\hline
\optableent
	{integer flags}
	{{\bf \htmlref{integertype}{outputsdict:integertype}}}
	{string}
	{Create formatted string from integer.}
\hline
\optableent
	{mark flags}
	{{\bf \htmlref{marktype}{outputsdict:marktype}}}
	{string}
	{Create formatted string from mark.}
\hline
\optableent
	{mutex flags}
	{{\bf \htmlref{mutextype}{outputsdict:mutextype}}}
	{string}
	{Create formatted string from mutex.}
\hline
\optableent
	{name flags}
	{{\bf \htmlref{nametype}{outputsdict:nametype}}}
	{string}
	{Create formatted string from name.}
\hline
\optableent
	{null flags}
	{{\bf \htmlref{nulltype}{outputsdict:nulltype}}}
	{string}
	{Create formatted string from null.}
\hline
\optableent
	{operator flags}
	{{\bf \htmlref{operatortype}{outputsdict:operatortype}}}
	{string}
	{Create formatted string from operator.}
\hline
\optableent
	{pmark flags}
	{{\bf \htmlref{pmarktype}{outputsdict:pmarktype}}}
	{string}
	{Create formatted string from pmark.}
\hline
\optableent
	{real flags}
	{{\bf \htmlref{realtype}{outputsdict:realtype}}}
	{string}
	{Create formatted string from real.}
\hline
\optableent
	{regex flags}
	{{\bf \htmlref{regextype}{outputsdict:regextype}}}
	{string}
	{Create formatted string from regex.}
\hline
\optableent
	{regsub flags}
	{{\bf \htmlref{regsubtype}{outputsdict:regsubtype}}}
	{string}
	{Create formatted string from regsub.}
\hline
\optableent
	{stack flags}
	{{\bf \htmlref{stacktype}{outputsdict:stacktype}}}
	{string}
	{Create formatted string from stack.}
\hline
\optableent
	{string flags}
	{{\bf \htmlref{stringtype}{outputsdict:stringtype}}}
	{string}
	{Create formatted string from string.}
\hline
\optableent
	{thread flags}
	{{\bf \htmlref{threadtype}{outputsdict:threadtype}}}
	{string}
	{Create formatted string from thread.}
\end{longtable}

\begin{description}
\label{outputsdict:arraytype}
\index{arraytype@\onyxop{}{arraytype}{}}
\item[{\onyxop{array flags}{arraytype}{string}}: ]
	\begin{description}\item[]
	\item[Input(s): ]
		\begin{description}\item[]
		\item[array: ]
			An array object.
		\item[flags: ]
			Formatting flags.
		\end{description}
	\item[Output(s): ]
		\begin{description}\item[]
		\item[string: ]
			Formatted string representation of \oparg{array}.
		\end{description}
	\item[Error(s): ]
		\begin{description}\item[]
		\item[\htmlref{stackunderflow}{stackunderflow}.]
		\item[\htmlref{typecheck}{typecheck}.]
		\end{description}
	\item[Description: ]
		Create a formatted string representation of \oparg{array}.
	\item[Example(s): ]\begin{verbatim}

onyx:0> outputsdict begin
onyx:0> [1 [2 3] 4]
onyx:1> dup <$w 9 $p `_' $r 0> arraytype print `\n' print flush
__-array-
onyx:1> dup <$w 9 $p `_' $r 1> arraytype print `\n' print flush
[1 -array- 4]
onyx:1>
		\end{verbatim}
	\end{description}
\label{outputsdict:booleantype}
\index{booleantype@\onyxop{}{booleantype}{}}
\item[{\onyxop{boolean flags}{booleantype}{string}}: ]
	\begin{description}\item[]
	\item[Input(s): ]
		\begin{description}\item[]
		\item[boolean: ]
			A boolean object.
		\item[flags: ]
			Formatting flags.
		\end{description}
	\item[Output(s): ]
		\begin{description}\item[]
		\item[string: ]
			Formatted string representation of \oparg{boolean}.
		\end{description}
	\item[Error(s): ]
		\begin{description}\item[]
		\item[\htmlref{stackunderflow}{stackunderflow}.]
		\item[\htmlref{typecheck}{typecheck}.]
		\end{description}
	\item[Description: ]
		Create a formatted string representation of \oparg{boolean}.
	\item[Example(s): ]\begin{verbatim}

onyx:0> outputsdict begin
onyx:0> false
onyx:1> dup <$n 3> booleantype print `\n' print flush
fal
onyx:1> dup <$n 7> booleantype print `\n' print flush
false
onyx:1>
		\end{verbatim}
	\end{description}
\label{outputsdict:classtype}
\index{classtype@\onyxop{}{classtype}{}}
\item[{\onyxop{class flags}{classtype}{string}}: ]
	\begin{description}\item[]
	\item[Input(s): ]
		\begin{description}\item[]
		\item[class: ]
			A class object.
		\item[flags: ]
			Formatting flags.
		\end{description}
	\item[Output(s): ]
		\begin{description}\item[]
		\item[string: ]
			Formatted string representation of \oparg{class}.
		\end{description}
	\item[Error(s): ]
		\begin{description}\item[]
		\item[\htmlref{stackunderflow}{stackunderflow}.]
		\item[\htmlref{typecheck}{typecheck}.]
		\end{description}
	\item[Description: ]
		Create a formatted string representation of \oparg{class}.
	\item[Example(s): ]\begin{verbatim}

onyx:0> outputsdict begin
onyx:0> vclass
onyx:1> <$w 30 $p `.' $j $c> classtype print `\n' print flush
...........-class-............
onyx:0>
		\end{verbatim}
	\end{description}
\label{outputsdict:conditiontype}
\index{conditiontype@\onyxop{}{conditiontype}{}}
\item[{\onyxop{condition flags}{conditiontype}{string}}: ]
	\begin{description}\item[]
	\item[Input(s): ]
		\begin{description}\item[]
		\item[condition: ]
			A condition object.
		\item[flags: ]
			Formatting flags.
		\end{description}
	\item[Output(s): ]
		\begin{description}\item[]
		\item[string: ]
			Formatted string representation of \oparg{condition}.
		\end{description}
	\item[Error(s): ]
		\begin{description}\item[]
		\item[\htmlref{stackunderflow}{stackunderflow}.]
		\item[\htmlref{typecheck}{typecheck}.]
		\end{description}
	\item[Description: ]
		Create a formatted string representation of \oparg{condition}.
	\item[Example(s): ]\begin{verbatim}

onyx:0> outputsdict begin
onyx:0> condition
onyx:1> <$w 15 $p `_' $j $c> booleantype print `\n' print flush
__-condition-__
onyx:0>
		\end{verbatim}
	\end{description}
\label{outputsdict:dicttype}
\index{dicttype@\onyxop{}{dicttype}{}}
\item[{\onyxop{dict flags}{dicttype}{string}}: ]
	\begin{description}\item[]
	\item[Input(s): ]
		\begin{description}\item[]
		\item[dict: ]
			A dict object.
		\item[flags: ]
			Formatting flags.
		\end{description}
	\item[Output(s): ]
		\begin{description}\item[]
		\item[string: ]
			Formatted string representation of \oparg{dict}.
		\end{description}
	\item[Error(s): ]
		\begin{description}\item[]
		\item[\htmlref{stackunderflow}{stackunderflow}.]
		\item[\htmlref{typecheck}{typecheck}.]
		\end{description}
	\item[Description: ]
		Create a formatted string representation of \oparg{dict}.
	\item[Example(s): ]\begin{verbatim}

onyx:0> outputsdict begin
onyx:0> <$foo `foo'> <$w 30 $p `.' $j $r> dicttype print `\n' print flush
..................<$foo `foo'>
onyx:0>
		\end{verbatim}
	\end{description}
\label{outputsdict:filetype}
\index{filetype@\onyxop{}{filetype}{}}
\item[{\onyxop{file flags}{filetype}{string}}: ]
	\begin{description}\item[]
	\item[Input(s): ]
		\begin{description}\item[]
		\item[file: ]
			A file object.
		\item[flags: ]
			Formatting flags.
		\end{description}
	\item[Output(s): ]
		\begin{description}\item[]
		\item[string: ]
			Formatted string representation of \oparg{file}.
		\end{description}
	\item[Error(s): ]
		\begin{description}\item[]
		\item[\htmlref{stackunderflow}{stackunderflow}.]
		\item[\htmlref{typecheck}{typecheck}.]
		\end{description}
	\item[Description: ]
		Create a formatted string representation of \oparg{file}.
	\item[Example(s): ]\begin{verbatim}

onyx:0> outputsdict begin
onyx:0> stdin
onyx:1> <$w 30 $p `.' $j $c> filetype print `\n' print flush
............-file-............
onyx:0>
		\end{verbatim}
	\end{description}
\label{outputsdict:finotype}
\index{finotype@\onyxop{}{finotype}{}}
\item[{\onyxop{fino flags}{finotype}{string}}: ]
	\begin{description}\item[]
	\item[Input(s): ]
		\begin{description}\item[]
		\item[fino: ]
			A fino object.
		\item[flags: ]
			Formatting flags.
		\end{description}
	\item[Output(s): ]
		\begin{description}\item[]
		\item[string: ]
			Formatted string representation of \oparg{fino}.
		\end{description}
	\item[Error(s): ]
		\begin{description}\item[]
		\item[\htmlref{stackunderflow}{stackunderflow}.]
		\item[\htmlref{typecheck}{typecheck}.]
		\end{description}
	\item[Description: ]
		Create a formatted string representation of \oparg{fino}.
	\item[Example(s): ]\begin{verbatim}

onyx:0> outputsdict begin
onyx:0> (
onyx:1> <$w 30 $p `.' $j $c> finotype print `\n' print flush
............-fino-............
onyx:0>
		\end{verbatim}
	\end{description}
\label{outputsdict:handletype}
\index{handletype@\onyxop{}{handletype}{}}
\item[{\onyxop{handle flags}{handletype}{string}}: ]
	\begin{description}\item[]
	\item[Input(s): ]
		\begin{description}\item[]
		\item[handle: ]
			A handle object.
		\item[flags: ]
			Formatting flags.
		\end{description}
	\item[Output(s): ]
		\begin{description}\item[]
		\item[string: ]
			Formatted string representation of \oparg{handle}.
		\end{description}
	\item[Error(s): ]
		\begin{description}\item[]
		\item[\htmlref{stackunderflow}{stackunderflow}.]
		\item[\htmlref{typecheck}{typecheck}.]
		\end{description}
	\item[Description: ]
		Create a formatted string representation of \oparg{handle}.
	\item[Example(s): ]
		The following example is a bit contrived, since there is no way
		to create a handle object with a stock onyx interpreter.
		Therefore, imagine that an operator named taggedhandle exists
		that creates a handle with a tag that is the name ``tagged''.
		\begin{verbatim}

onyx:0> outputsdict begin
onyx:0> taggedhandle
onyx:1> <$w 30 $p `.' $j $l handletype print `\n' print flush
=tagged=......................
onyx:0>
		\end{verbatim}
	\end{description}
\label{outputsdict:instancetype}
\index{instancetype@\onyxop{}{instancetype}{}}
\item[{\onyxop{instance flags}{instancetype}{string}}: ]
	\begin{description}\item[]
	\item[Input(s): ]
		\begin{description}\item[]
		\item[instance: ]
			An instance object.
		\item[flags: ]
			Formatting flags.
		\end{description}
	\item[Output(s): ]
		\begin{description}\item[]
		\item[string: ]
			Formatted string representation of \oparg{instance}.
		\end{description}
	\item[Error(s): ]
		\begin{description}\item[]
		\item[\htmlref{stackunderflow}{stackunderflow}.]
		\item[\htmlref{typecheck}{typecheck}.]
		\end{description}
	\item[Description: ]
		Create a formatted string representation of \oparg{instance}.
	\item[Example(s): ]\begin{verbatim}

onyx:0> outputsdict begin
onyx:0> vinstance
onyx:1> <$w 30 $p `.' $j $c> instancetype print `\n' print flush
..........-instance-..........
onyx:0>
		\end{verbatim}
	\end{description}
\label{outputsdict:integertype}
\index{integertype@\onyxop{}{integertype}{}}
\item[{\onyxop{integer flags}{integertype}{string}}: ]
	\begin{description}\item[]
	\item[Input(s): ]
		\begin{description}\item[]
		\item[integer: ]
			An integer object.
		\item[flags: ]
			Formatting flags.
		\end{description}
	\item[Output(s): ]
		\begin{description}\item[]
		\item[string: ]
			Formatted string representation of \oparg{integer}.
		\end{description}
	\item[Error(s): ]
		\begin{description}\item[]
		\item[\htmlref{stackunderflow}{stackunderflow}.]
		\item[\htmlref{typecheck}{typecheck}.]
		\end{description}
	\item[Description: ]
		Create a formatted string representation of \oparg{integer}.
	\item[Example(s): ]\begin{verbatim}

onyx:0> outputsdict begin
onyx:0> 42 <$w 6 $p `_' $j $c $s $-> integertype print `\n' print flush
__42__
onyx:0> 42 <$w 6 $p `_' $j $c $s $+> integertype print `\n' print flush
_+42__
onyx:0> `0x' print 42 <$w 6 $p `0' $b 16> integertype print `\n' print flush
0x00002a
onyx:0>
		\end{verbatim}
	\end{description}
\label{outputsdict:marktype}
\index{marktype@\onyxop{}{marktype}{}}
\item[{\onyxop{mark flags}{marktype}{string}}: ]
	\begin{description}\item[]
	\item[Input(s): ]
		\begin{description}\item[]
		\item[mark: ]
			A mark object.
		\item[flags: ]
			Formatting flags.
		\end{description}
	\item[Output(s): ]
		\begin{description}\item[]
		\item[string: ]
			Formatted string representation of \oparg{mark}.
		\end{description}
	\item[Error(s): ]
		\begin{description}\item[]
		\item[\htmlref{stackunderflow}{stackunderflow}.]
		\item[\htmlref{typecheck}{typecheck}.]
		\end{description}
	\item[Description: ]
		Create a formatted string representation of \oparg{mark}.
	\item[Example(s): ]\begin{verbatim}

onyx:0> outputsdict begin
onyx:0> mark
onyx:1> <$w 30 $p `.' $j $c> marktype print `\n' print flush
............-mark-............
onyx:0>
		\end{verbatim}
	\end{description}
\label{outputsdict:mutextype}
\index{mutextype@\onyxop{}{mutextype}{}}
\item[{\onyxop{mutex flags}{mutextype}{string}}: ]
	\begin{description}\item[]
	\item[Input(s): ]
		\begin{description}\item[]
		\item[mutex: ]
			A mutex object.
		\item[flags: ]
			Formatting flags.
		\end{description}
	\item[Output(s): ]
		\begin{description}\item[]
		\item[string: ]
			Formatted string representation of \oparg{mutex}.
		\end{description}
	\item[Error(s): ]
		\begin{description}\item[]
		\item[\htmlref{stackunderflow}{stackunderflow}.]
		\item[\htmlref{typecheck}{typecheck}.]
		\end{description}
	\item[Description: ]
		Create a formatted string representation of \oparg{mutex}.
	\item[Example(s): ]\begin{verbatim}

onyx:0> outputsdict begin
onyx:0> mutex
onyx:1> <$w 30 $p `.' $j $c> mutextype print `\n' print flush
...........-mutex-............
onyx:0>
		\end{verbatim}
	\end{description}
\label{outputsdict:nametype}
\index{nametype@\onyxop{}{nametype}{}}
\item[{\onyxop{name flags}{nametype}{string}}: ]
	\begin{description}\item[]
	\item[Input(s): ]
		\begin{description}\item[]
		\item[name: ]
			A name object.
		\item[flags: ]
			Formatting flags.
		\end{description}
	\item[Output(s): ]
		\begin{description}\item[]
		\item[string: ]
			Formatted string representation of \oparg{name}.
		\end{description}
	\item[Error(s): ]
		\begin{description}\item[]
		\item[\htmlref{stackunderflow}{stackunderflow}.]
		\item[\htmlref{typecheck}{typecheck}.]
		\end{description}
	\item[Description: ]
		Create a formatted string representation of \oparg{name}.
	\item[Example(s): ]\begin{verbatim}

onyx:0> outputsdict begin
onyx:0> $foo
onyx:1> <$w 30 $p `.' $j $c> nametype print `\n' print flush
.............$foo.............
onyx:0>
		\end{verbatim}
	\end{description}
\label{outputsdict:nulltype}
\index{nulltype@\onyxop{}{nulltype}{}}
\item[{\onyxop{null flags}{nulltype}{string}}: ]
	\begin{description}\item[]
	\item[Input(s): ]
		\begin{description}\item[]
		\item[null: ]
			A null object.
		\item[flags: ]
			Formatting flags.
		\end{description}
	\item[Output(s): ]
		\begin{description}\item[]
		\item[string: ]
			Formatted string representation of \oparg{null}.
		\end{description}
	\item[Error(s): ]
		\begin{description}\item[]
		\item[\htmlref{stackunderflow}{stackunderflow}.]
		\item[\htmlref{typecheck}{typecheck}.]
		\end{description}
	\item[Description: ]
		Create a formatted string representation of \oparg{null}.
	\item[Example(s): ]\begin{verbatim}

onyx:0> outputsdict begin
onyx:0> null
onyx:1> <$w 30 $p `.' $j $c> nulltype print `\n' print flush
.............null.............
onyx:0>
		\end{verbatim}
	\end{description}
\label{outputsdict:operatortype}
\index{operatortype@\onyxop{}{operatortype}{}}
\item[{\onyxop{operator flags}{operatortype}{string}}: ]
	\begin{description}\item[]
	\item[Input(s): ]
		\begin{description}\item[]
		\item[operator: ]
			An operator object.
		\item[flags: ]
			Formatting flags.
		\end{description}
	\item[Output(s): ]
		\begin{description}\item[]
		\item[string: ]
			Formatted string representation of \oparg{operator}.
		\end{description}
	\item[Error(s): ]
		\begin{description}\item[]
		\item[\htmlref{stackunderflow}{stackunderflow}.]
		\item[\htmlref{typecheck}{typecheck}.]
		\end{description}
	\item[Description: ]
		Create a formatted string representation of \oparg{operator}.
	\item[Example(s): ]
		The following example shows an operator printed out with two
		leading and trailing dashes.  If the interpreter cannot
		determine the name associated with an operator, as will be the
		case for custom operators, the operator will be printed as
		{\tt -operator-}.
		\begin{verbatim}

onyx:0> outputsdict begin
onyx:0> ~realtime
onyx:1> <$w 30 $p `.' $j $c> operatortype print `\n' print flush
.........--realtime--.........
onyx:0>
		\end{verbatim}
	\end{description}
\label{outputsdict:pmarktype}
\index{pmarktype@\onyxop{}{pmarktype}{}}
\item[{\onyxop{pmark flags}{pmarktype}{string}}: ]
	\begin{description}\item[]
	\item[Input(s): ]
		\begin{description}\item[]
		\item[pmark: ]
			A pmark object.
		\item[flags: ]
			Formatting flags.
		\end{description}
	\item[Output(s): ]
		\begin{description}\item[]
		\item[string: ]
			Formatted string representation of \oparg{pmark}.
		\end{description}
	\item[Error(s): ]
		\begin{description}\item[]
		\item[\htmlref{stackunderflow}{stackunderflow}.]
		\item[\htmlref{typecheck}{typecheck}.]
		\end{description}
	\item[Description: ]
		Create a formatted string representation of \oparg{pmark}.
	\item[Example(s): ]\begin{verbatim}

onyx:0> outputsdict begin
onyx:0> { ~x
Error $undefined
ostack: (-pmark- $x)
dstack: (-dict- -dict- -dict- -dict- -dict-)
cstack: ()
estack/istack trace (0..1):
0:      -file-
1:      --start--
onyx:3> pop pop resume
onyx:1> <$w 30 $p `.' $j $c> pmarktype print `\n' print flush
...........-pmark-............
onyx:0>
		\end{verbatim}
	\end{description}
\label{outputsdict:realtype}
\index{realtype@\onyxop{}{realtype}{}}
\item[{\onyxop{real flags}{realtype}{string}}: ]
	\begin{description}\item[]
	\item[Input(s): ]
		\begin{description}\item[]
		\item[real: ]
			A real object.
		\item[flags: ]
			Formatting flags.
		\end{description}
	\item[Output(s): ]
		\begin{description}\item[]
		\item[string: ]
			Formatted string representation of \oparg{real}.
		\end{description}
	\item[Error(s): ]
		\begin{description}\item[]
		\item[\htmlref{stackunderflow}{stackunderflow}.]
		\item[\htmlref{typecheck}{typecheck}.]
		\end{description}
	\item[Description: ]
		Create a formatted string representation of \oparg{real}.
	\item[Example(s): ]\begin{verbatim}

onyx:0> outputsdict begin
onyx:0> 6.022e23 <$d 4> realtype print `\n' print flush
602200000000000027262976.0000
onyx:0> 6.022e23 <$d 4 $e true> realtype print `\n' print flush
6.0220e+23
onyx:0> 6.022e23 <$d 0 $e true> realtype print `\n' print flush
6e+23
onyx:0> 6.022e23 <$d 4 $w 40 $p `0'> realtype print `\n' print flush
00000000000602200000000000027262976.0000
onyx:0>
		\end{verbatim}
	\end{description}
\label{outputsdict:regextype}
\index{regextype@\onyxop{}{regextype}{}}
\item[{\onyxop{regex flags}{regextype}{string}}: ]
	\begin{description}\item[]
	\item[Input(s): ]
		\begin{description}\item[]
		\item[regex: ]
			A regex object.
		\item[flags: ]
			Formatting flags.
		\end{description}
	\item[Output(s): ]
		\begin{description}\item[]
		\item[string: ]
			Formatted string representation of \oparg{regex}.
		\end{description}
	\item[Error(s): ]
		\begin{description}\item[]
		\item[\htmlref{stackunderflow}{stackunderflow}.]
		\item[\htmlref{typecheck}{typecheck}.]
		\end{description}
	\item[Description: ]
		Create a formatted string representation of \oparg{regex}.
	\item[Example(s): ]\begin{verbatim}

onyx:0> outputsdict begin
onyx:0> `' regex
onyx:1> <$w 30 $p `.' $j $c> regextype print `\n' print flush
...........-regex-............
onyx:0>
		\end{verbatim}
	\end{description}
\label{outputsdict:regsubtype}
\index{regsubtype@\onyxop{}{regsubtype}{}}
\item[{\onyxop{regsub flags}{regsubtype}{string}}: ]
	\begin{description}\item[]
	\item[Input(s): ]
		\begin{description}\item[]
		\item[regsub: ]
			A regsub object.
		\item[flags: ]
			Formatting flags.
		\end{description}
	\item[Output(s): ]
		\begin{description}\item[]
		\item[string: ]
			Formatted string representation of \oparg{regsub}.
		\end{description}
	\item[Error(s): ]
		\begin{description}\item[]
		\item[\htmlref{stackunderflow}{stackunderflow}.]
		\item[\htmlref{typecheck}{typecheck}.]
		\end{description}
	\item[Description: ]
		Create a formatted string representation of \oparg{regsub}.
	\item[Example(s): ]\begin{verbatim}

onyx:0> outputsdict begin
onyx:0> `' `' regsub
onyx:1> <$w 30 $p `.' $j $c> regsubtype print `\n' print flush
...........-regsub-...........
onyx:0>
		\end{verbatim}
	\end{description}
\label{outputsdict:stacktype}
\index{stacktype@\onyxop{}{stacktype}{}}
\item[{\onyxop{stack flags}{stacktype}{string}}: ]
	\begin{description}\item[]
	\item[Input(s): ]
		\begin{description}\item[]
		\item[stack: ]
			A stack object.
		\item[flags: ]
			Formatting flags.
		\end{description}
	\item[Output(s): ]
		\begin{description}\item[]
		\item[string: ]
			Formatted string representation of \oparg{stack}.
		\end{description}
	\item[Error(s): ]
		\begin{description}\item[]
		\item[\htmlref{stackunderflow}{stackunderflow}.]
		\item[\htmlref{typecheck}{typecheck}.]
		\end{description}
	\item[Description: ]
		Create a formatted string representation of \oparg{stack}.
	\item[Example(s): ]\begin{verbatim}

onyx:0> outputsdict begin
onyx:0> (1 (2 3) 4)
onyx:1> dup <$w 9 $p `_' $r 0> stacktype print `\n' print flush
__-stack-
onyx:1> <$w 9 $p `_' $r 1> stacktype print `\n' print flush
(1 -stack- 4)
onyx:0>
		\end{verbatim}
	\end{description}
\label{outputsdict:stringtype}
\index{stringtype@\onyxop{}{stringtype}{}}
\item[{\onyxop{string flags}{stringtype}{string}}: ]
	\begin{description}\item[]
	\item[Input(s): ]
		\begin{description}\item[]
		\item[string: ]
			A string object.
		\item[flags: ]
			Formatting flags.
		\end{description}
	\item[Output(s): ]
		\begin{description}\item[]
		\item[string: ]
			Formatted string representation of \oparg{string}.
		\end{description}
	\item[Error(s): ]
		\begin{description}\item[]
		\item[\htmlref{stackunderflow}{stackunderflow}.]
		\item[\htmlref{typecheck}{typecheck}.]
		\end{description}
	\item[Description: ]
		Create a formatted string representation of \oparg{string}.
	\item[Example(s): ]\begin{verbatim}

onyx:0> outputsdict begin
onyx:0> `A string'
onyx:1> <$w 30 $p `.' $j $c> stringtype print `\n' print flush
...........A string...........
onyx:0>
		\end{verbatim}
	\end{description}
\label{outputsdict:threadtype}
\index{threadtype@\onyxop{}{threadtype}{}}
\item[{\onyxop{thread flags}{threadtype}{string}}: ]
	\begin{description}\item[]
	\item[Input(s): ]
		\begin{description}\item[]
		\item[thread: ]
			A thread object.
		\item[flags: ]
			Formatting flags.
		\end{description}
	\item[Output(s): ]
		\begin{description}\item[]
		\item[string: ]
			Formatted string representation of \oparg{thread}.
		\end{description}
	\item[Error(s): ]
		\begin{description}\item[]
		\item[\htmlref{stackunderflow}{stackunderflow}.]
		\item[\htmlref{typecheck}{typecheck}.]
		\end{description}
	\item[Description: ]
		Create a formatted string representation of \oparg{thread}.
	\item[Example(s): ]\begin{verbatim}

onyx:0> outputsdict begin
onyx:0> () {} thread
onyx:1> <$w 30 $p `.' $j $c> threadtype print `\n' print flush
...........-thread-...........
onyx:0>
		\end{verbatim}
	\end{description}
\end{description}

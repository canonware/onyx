%-*-mode:latex-*-
%%%%%%%%%%%%%%%%%%%%%%%%%%%%%%%%%%%%%%%%%%%%%%%%%%%%%%%%%%%%%%%%%%%%%%%%%%%%%
%
% <Copyright = jasone>
% <License>
%
%%%%%%%%%%%%%%%%%%%%%%%%%%%%%%%%%%%%%%%%%%%%%%%%%%%%%%%%%%%%%%%%%%%%%%%%%%%%%
%
% Version: <Version>
%
% socks portion of Canonware Software Manual.
%              
%%%%%%%%%%%%%%%%%%%%%%%%%%%%%%%%%%%%%%%%%%%%%%%%%%%%%%%%%%%%%%%%%%%%%%%%%%%%%

\subsection{socks}
\label{socks}
\index{\classname{socks}{}}

The \classname{socks} class implements a socket listener (server).  The
\classname{socks} class is used to accept connections to a port and create a
\classname{sock} instance for each connection.

\subsubsection{API}
\begin{capi}
\label{socks_new}
\index{\cfunc{socks\_new}{}}
\citem{\cfunc[cw\_socks\_t *]{socks\_new}{void}}
	\begin{capilist}
	\item[Input(s): ] None.
	\item[Output(s): ]
		\begin{description}\item[]
		\item[retval: ]
			Pointer to a \classname{socks}.
		\end{description}
	\item[Exception(s): ]
		\begin{description}\item[]
		\item[\htmlref{\_CW\_STASHX\_OOM}{_CW_STASHX_OOM}.]
		\end{description}
	\item[Description: ]
		Constructor.
	\end{capilist}
\label{socks_delete}
\index{\cfunc{socks\_delete}{}}
\citem{\cfunc[void]{socks\_delete}{cw\_socks\_t *a\_socks}}
	\begin{capilist}
	\item[Input(s): ]
		\begin{description}\item[]
		\item[a\_socks: ]
			Pointer to a \classname{socks}.
		\end{description}
	\item[Output(s): ] None.
	\item[Exception(s): ] None.
	\item[Description: ]
		Destructor.
	\end{capilist}
\label{socks_listen}
\index{\cfunc{socks\_listen}{}}
\citem{\cfunc[cw\_bool\_t]{socks\_listen}{cw\_socks\_t *a\_socks,
cw\_uint32\_a\_mask, int *r\_port}}
	\begin{capilist}
	\item[Input(s): ]
		\begin{description}\item[]
		\item[a\_socks: ]
			Pointer to a \classname{socks}.
		\item[a\_mask: ]
			Mask of client addresses to listen to (INADDR\_ANY,
			INADDR\_LOOPBACK, etc.).
		\item[r\_port: ]
			Pointer to a port number to listen on, or 0.
		\end{description}
	\item[Output(s): ]
		\begin{description}\item[]
		\item[retval: ]
			\begin{description}\item[]
			\item[FALSE: ]
				Success.
			\item[TRUE: ]
				\cfunc{socket}{} error, \cfunc{bind}{} error, or
				\cfunc{getsockname}{} error.
			\end{description}
		\item[*r\_port: ]
			If \cvar{retval} is FALSE, port number that
			\cvar{a\_socks} is listening on.  Otherwise, undefined.
		\end{description}
	\item[Exception(s): ] None.
	\item[Description: ]
		Do setup and start accepting connections on \cvar{*r\_port}.  If
		\cvar{*r\_port} is 0, let the operating system choose what port
		number to use, and assign the number to \cvar{*r\_port} before
		returning.
	\end{capilist}
\label{socks_accept}
\index{\cfunc{socks\_accept}{}}
\citem{\cfunc[cw\_sock\_t *]{socks\_accept}{cw\_socks\_t *a\_socks, struct
timespec *a\_timeout, cw\_sock\_t *r\_sock}}
	\begin{capilist}
	\item[Input(s): ]
		\begin{description}\item[]
		\item[a\_socks: ]
			Pointer to a \classname{socks}.
		\item[a\_timeout: ]
			Pointer to a timeout value, specified as an absolute
			time interval, or NULL.  A NULL value will cause this
			function to block indefinitely.
		\item[r\_sock: ]
			Pointer to a \classname{sock} that is not connected.
		\end{description}
	\item[Output(s): ]
		\begin{description}\item[]
		\item[retval: ]
			\begin{description}\item[]
			\item[non-NULL: ]
				\cvar{r\_sock}.
			\item[NULL: ]
				Not listening, cannot allocate a file
				descriptor, \cfunc{accept}{} error, or
				\cfunc{sock\_wrap}{} error.
			\end{description}
		\end{description}
	\item[Exception(s): ] None.
	\item[Description: ]
		Accept a connection.  Don't return until someone connects, or
		the timeout expires.
	\end{capilist}
\end{capi}

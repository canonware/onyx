%-*-mode:latex-*-
%%%%%%%%%%%%%%%%%%%%%%%%%%%%%%%%%%%%%%%%%%%%%%%%%%%%%%%%%%%%%%%%%%%%%%%%%%%%%
%
% <Copyright = jasone>
% <License>
%
%%%%%%%%%%%%%%%%%%%%%%%%%%%%%%%%%%%%%%%%%%%%%%%%%%%%%%%%%%%%%%%%%%%%%%%%%%%%%
%
% Version: <Version>
%
% thd portion of Canonware Software Manual.
%              
%%%%%%%%%%%%%%%%%%%%%%%%%%%%%%%%%%%%%%%%%%%%%%%%%%%%%%%%%%%%%%%%%%%%%%%%%%%%%

\subsection{thd}
\label{thd}
\index{\classname{thd}{}}

The \classname{thd} class implements a wrapper around the system threads library
(POSIX threads only, so far).  In most regards, this is a thin wrapper around
the normal threads functionality provided by the system, but some extra
information is kept in order to allow implmentation of thread
suspension/resumption, ``critical sections'', and ``single sections''.

In most cases, the additional functionality is implemented with the aid of
signals.  As a result, system calls may be interrupted by signals.  The system
calls will be automaticalaly restarted if they have made no progress at the time
of interruption, but will return a partial result otherwise.  Therefore, if any
of the additional functionality is utilized, the application must be careful to
handle partial system call results.  At least the following system calls can be
interrupted: \cfunc{read}{}, \cfunc{write}{}, \cfunc{sendto}{},
\cfunc{recvfrom}{}, \cfunc{sendmsg}{} and \cfunc{recvmsg}{}, \cfunc{ioctl}{},
and \cfunc{wait}{}.  See the system documentation for additional information.

\subsubsection{API}
\begin{description}
\label{thd_}
\index{\cfunc{thd\_}{}}
\item[{\cfunc[]{thd\_}{}}: ]
	\begin{description}\item[]
	\item[Input(s): ]
		\begin{description}\item[]
		\item[: ]
		\end{description}
	\item[Output(s): ]
		\begin{description}\item[]
		\item[: ]
		\end{description}
	\item[Exception(s): ]
		\begin{description}\item[]
		\item[\htmlref{\_CW\_STASHX\_}{_CW_STASHX_}.]
		\end{description}
	\item[Description: ]
	\end{description}
\end{description}

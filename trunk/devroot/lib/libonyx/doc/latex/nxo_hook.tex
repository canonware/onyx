%-*-mode:latex-*-
%%%%%%%%%%%%%%%%%%%%%%%%%%%%%%%%%%%%%%%%%%%%%%%%%%%%%%%%%%%%%%%%%%%%%%%%%%%%%
%
% <Copyright = jasone>
% <License>
%
%%%%%%%%%%%%%%%%%%%%%%%%%%%%%%%%%%%%%%%%%%%%%%%%%%%%%%%%%%%%%%%%%%%%%%%%%%%%%
%
% Version: <Version>
%
% nxo_hook portion of Canonware Software Manual.
%              
%%%%%%%%%%%%%%%%%%%%%%%%%%%%%%%%%%%%%%%%%%%%%%%%%%%%%%%%%%%%%%%%%%%%%%%%%%%%%

\subsection{nxo\_hook}
\label{nxo_hook}
\index{nxo_hook@\classname{nxo\_hook}{}}

The \classname{nxo\_hook} class is a subclass of the \classname{nxo} class.

\subsubsection{API}
\begin{capi}
\label{cw_nxo_hook_eval_t}
\index{cw_nxo_hook_eval_t@\cfunc{cw\_nxo\_hook\_eval\_t}{}}
\citem{\cfunc[void]{cw\_nxo\_hook\_eval\_t}{void *a\_data,
cw\_nxo\_t *a\_thread}}
	\begin{capilist}
	\item[Input(s): ]
		\begin{description}\item[]
		\item[a\_data: ]
			Opaque data pointer.
		\item[a\_thread: ]
			Pointer to a thread \classname{nxo}.
		\end{description}
	\item[Output(s): ] None.
	\item[Exception(s): ] Hook-dependent.
	\item[Description: ]
		Evaluation function typedef.
	\end{capilist}
\label{cw_nxo_hook_ref_iter_t}
\index{cw_nxo_hook_ref_iter_t@\cfunc{cw\_nxo\_hook\_ref\_iter\_t}{}}
\citem{\cfunc[cw\_nxoe\_t *]{cw\_nxo\_hook\_ref\_iter\_t}{void *a\_data,
cw\_bool\_t a\_reset}}
	\begin{capilist}
	\item[Input(s): ]
		\begin{description}\item[]
		\item[a\_data: ]
			Opaque data pointer.
		\item[a\_reset: ]
			\begin{description}\item[]
			\item[FALSE: ]
				At least one iteration has already occurred.
			\item[TRUE: ]
				First iteration.
			\end{description}
		\end{description}
	\item[Output(s): ]
		\begin{description}\item[]
		\item[retval: ]
			\begin{description}\item[]
			\item[non-NULL: ]
				Pointer to an \classname{nxoe}.
			\item[NULL: ]
				No more references.
			\end{description}
		\end{description}
	\item[Exception(s): ] None.
	\item[Description: ]
		Reference iterator function typedef.
	\end{capilist}
\label{cw_nxo_hook_delete_t}
\index{cw_nxo_hook_delete_t@\cfunc{cw\_nxo\_hook\_delete\_t}{}}
\citem{\cfunc[void]{cw\_nxo\_hook\_delete\_t}{void *a\_data, cw\_nx\_t *a\_nx}}
	\begin{capilist}
	\item[Input(s): ]
		\begin{description}\item[]
		\item[a\_data: ]
			Opaque data pointer.
		\item[a\_nx: ]
			Pointer to an \classname{nx}.
		\end{description}
	\item[Output(s): ] None.
	\item[Exception(s): ] None.
	\item[Description: ]
		Destructor function typedef.
	\end{capilist}
\label{nxo_hook_new}
\index{nxo_hook_new@\cfunc{nxo\_hook\_new}{}}
\citem{\cfunc[void]{nxo\_hook\_new}{cw\_nxo\_t *a\_nxo, cw\_nx\_t *a\_nx, void
*a\_data, cw\_nxo\_hook\_eval\_t *a\_eval\_f, cw\_nxo\_hook\_ref\_iter\_t
*a\_ref\_iter\_f, cw\_nxo\_hook\_delete\_t *a\_delete\_f}}
	\begin{capilist}
	\item[Input(s): ]
		\begin{description}\item[]
		\item[a\_nxo: ]
			Pointer to a hook \classname{nxo}.
		\item[a\_nx: ]
			Pointer to an \classname{nx}.
		\item[a\_data: ]
			Opaque data pointer to be passed to \cvar{a\_eval\_f},
			\cvar{a\_ref\_iter\_f}, and \cvar{a\_delete\_f}.
		\item[a\_eval\_f: ]
			Pointer to an evaluation function.
		\item[a\_ref\_iter\_f: ]
			Pointer to a reference iterator function.
		\item[a\_delete\_f: ]
			Pointer to a destructor function.
		\end{description}
	\item[Output(s): ] None.
	\item[Exception(s): ]
		\begin{description}\item[]
		\item[\htmlref{\_CW\_STASHX\_OOM}{_CW_STASHX_OOM}.]
		\end{description}
	\item[Description: ]
		Constructor.
	\end{capilist}
\label{nxo_hook_tag_get}
\index{nxo_hook_tag_get@\cfunc{nxo\_hook\_tag\_get}{}}
\citem{\cfunc[cw\_nxo\_t *]{nxo\_hook\_tag\_get}{cw\_nxo\_t *a\_nxo}}
	\begin{capilist}
	\item[Input(s): ]
		\begin{description}\item[]
		\item[a\_nxo: ]
			Pointer to a hook \classname{nxo}.
		\end{description}
	\item[Output(s): ]
		\begin{description}\item[]
		\item[retval: ]
			Pointer to the tag object associated with \cvar{a\_nxo}.
		\end{description}
	\item[Exception(s): ] None.
	\item[Description: ]
		Return a pointer to the tag object associated with
		\cvar{a\_nxo}.  This object pointer can safely be used for
		modifying the tag object.
	\end{capilist}
\label{nxo_hook_data_get}
\index{nxo_hook_data_get@\cfunc{nxo\_hook\_data\_get}{}}
\citem{\cfunc[void *]{nxo\_hook\_data\_get}{cw\_nxo\_t *a\_nxo}}
	\begin{capilist}
	\item[Input(s): ]
		\begin{description}\item[]
		\item[a\_nxo: ]
			Pointer to a hook \classname{nxo}.
		\end{description}
	\item[Output(s): ]
		\begin{description}\item[]
		\item[retval: ]
			Opaque data pointer.
		\end{description}
	\item[Exception(s): ] None.
	\item[Description: ]
		Return the opaque data pointer that was specified during
		construction.
	\end{capilist}
\label{nxo_hook_eval}
\index{nxo_hook_eval@\cfunc{nxo\_hook\_eval}{}}
\citem{\cfunc[void]{nxo\_hook\_eval}{cw\_nxo\_t *a\_nxo, cw\_nxo\_t *a\_thread}}
	\begin{capilist}
	\item[Input(s): ]
		\begin{description}\item[]
		\item[a\_nxo: ]
			Pointer to a hook \classname{nxo}.
		\item[a\_thread: ]
			Pointer to a thread \classname{nxo}.
		\end{description}
	\item[Output(s): ] None.
	\item[Exception(s): ] Hook-specific.
	\item[Description: ]
		Evaluate the \cvar{a\_nxo}.  If there is no evaluation function
		associated with \cvar{a\_nxo}, it is pushed onto ostack.
	\end{capilist}
\end{capi}

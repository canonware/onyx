%-*-mode:latex-*-
%%%%%%%%%%%%%%%%%%%%%%%%%%%%%%%%%%%%%%%%%%%%%%%%%%%%%%%%%%%%%%%%%%%%%%%%%%%%%
%
% <Copyright = jasone>
% <License>
%
%%%%%%%%%%%%%%%%%%%%%%%%%%%%%%%%%%%%%%%%%%%%%%%%%%%%%%%%%%%%%%%%%%%%%%%%%%%%%
%
% Version: Onyx <Version = onyx>
%
% nxo_class portion of Onyx Manual.
%              
%%%%%%%%%%%%%%%%%%%%%%%%%%%%%%%%%%%%%%%%%%%%%%%%%%%%%%%%%%%%%%%%%%%%%%%%%%%%%

\subsection{nxo\_class}
\label{nxo_class}
\index{nxo_class@\classname{nxo\_class}{}}

The \classname{nxo\_class} class is a subclass of the \classname{nxo} class.

\subsubsection{API}
\begin{capi}
\label{cw_nxo_class_ref_iter_t}
\index{cw_nxo_class_ref_iter_t@\cfunc{cw\_nxo\_class\_ref\_iter\_t}{}}
\citem{\cfunc[cw\_nxoe\_t *]{cw\_nxo\_class\_ref\_iter\_t}{void *a\_opaque,
cw\_bool\_t a\_reset}}
	\begin{capilist}
	\item[Input(s): ]
		\begin{description}\item[]
		\item[a\_opaque: ]
			Opaque data pointer.
		\item[a\_reset: ]
			\begin{description}\item[]
			\item[FALSE: ]
				At least one iteration has already occurred.
			\item[TRUE: ]
				First iteration.
			\end{description}
		\end{description}
	\item[Output(s): ]
		\begin{description}\item[]
		\item[retval: ]
			\begin{description}\item[]
			\item[non-NULL: ]
				Pointer to an \classname{nxoe}.
			\item[NULL: ]
				No more references.
			\end{description}
		\end{description}
	\item[Exception(s): ] None.
	\item[Description: ]
		Reference iterator function typedef.
	\end{capilist}
\label{cw_nxo_class_delete_t}
\index{cw_nxo_class_delete_t@\cfunc{cw\_nxo\_class\_delete\_t}{}}
\citem{\cfunc[cw\_bool\_t]{cw\_nxo\_class\_delete\_t}{void *a\_opaque,
cw\_uint32\_t a\_iter}}
	\begin{capilist}
	\item[Input(s): ]
		\begin{description}\item[]
		\item[a\_opaque: ]
			Opaque data pointer.
		\item[a\_iter: ]
			Garbage collector sweep iteration count (starts at 0).
			This value can be used to impose ordering of dependent
			object deletions.
		\end{description}
	\item[Output(s): ]
		\begin{description}\item[]
		\item[retval: ]
			\begin{description}\item[]
			\item[FALSE: ] Success.
			\item[TRUE: ] Defer deletion until a later garbage
			collector sweep iteration.
			\end{description}
		\end{description}
	\item[Exception(s): ] None.
	\item[Description: ]
		Destructor function typedef.
	\end{capilist}
\label{nxo_class_new}
\index{nxo_class_new@\cfunc{nxo\_class\_new}{}}
\citem{\cfunc[void]{nxo\_class\_new}{cw\_nxo\_t *a\_nxo, void *a\_opaque,
cw\_nxo\_class\_ref\_iter\_t *a\_ref\_iter\_f, cw\_nxo\_class\_delete\_t
*a\_delete\_f}}
	\begin{capilist}
	\item[Input(s): ]
		\begin{description}\item[]
		\item[a\_nxo: ]
			Pointer to a class \classname{nxo}.
		\item[a\_opaque: ]
			Opaque data pointer to be passed to
			\cvar{a\_ref\_iter\_f} and \cvar{a\_delete\_f}.
		\item[a\_ref\_iter\_f: ]
			Pointer to a reference iterator function.
		\item[a\_delete\_f: ]
			Pointer to a destructor function.
		\end{description}
	\item[Output(s): ] None.
	\item[Exception(s): ]
		\begin{description}\item[]
		\item[\htmlref{CW\_ONYXX\_OOM}{CW_ONYXX_OOM}.]
		\end{description}
	\item[Description: ]
		Constructor.
	\end{capilist}
\label{nxo_class_name_get}
\index{nxo_class_name_get@\cfunc{nxo\_class\_name\_get}{}}
\citem{\cfunc[cw\_nxo\_t *]{nxo\_class\_name\_get}{const cw\_nxo\_t *a\_nxo}}
	\begin{capilist}
	\item[Input(s): ]
		\begin{description}\item[]
		\item[a\_nxo: ]
			Pointer to a class \classname{nxo}.
		\end{description}
	\item[Output(s): ]
		\begin{description}\item[]
		\item[retval: ]
			Pointer to the name object associated with \cvar{a\_nxo}
			(may be of any type).
		\end{description}
	\item[Exception(s): ] None.
	\item[Description: ]
		Return a pointer to the name object associated with
		\cvar{a\_nxo}.  This object pointer can safely be used for
		modifying the name object.
	\end{capilist}
\label{nxo_class_super_get}
\index{nxo_class_super_get@\cfunc{nxo\_class\_super\_get}{}}
\citem{\cfunc[cw\_nxo\_t *]{nxo\_class\_super\_get}{const cw\_nxo\_t *a\_nxo}}
	\begin{capilist}
	\item[Input(s): ]
		\begin{description}\item[]
		\item[a\_nxo: ]
			Pointer to a class \classname{nxo}.
		\end{description}
	\item[Output(s): ]
		\begin{description}\item[]
		\item[retval: ]
			Pointer to the superclass object associated with
			\cvar{a\_nxo}.
		\end{description}
	\item[Exception(s): ] None.
	\item[Description: ]
		Return a pointer to the super object associated with
		\cvar{a\_nxo}.  This object pointer can safely be used for
		modifying the super object.
	\end{capilist}
\label{nxo_class_methods_get}
\index{nxo_class_methods_get@\cfunc{nxo\_class\_methods\_get}{}}
\citem{\cfunc[cw\_nxo\_t *]{nxo\_class\_methods\_get}{const cw\_nxo\_t *a\_nxo}}
	\begin{capilist}
	\item[Input(s): ]
		\begin{description}\item[]
		\item[a\_nxo: ]
			Pointer to a class \classname{nxo}.
		\end{description}
	\item[Output(s): ]
		\begin{description}\item[]
		\item[retval: ]
			Pointer to the methods object associated with
			\cvar{a\_nxo}.
		\end{description}
	\item[Exception(s): ] None.
	\item[Description: ]
		Return a pointer to the methods object associated with
		\cvar{a\_nxo}.  This object pointer can safely be used for
		modifying the methods object.
	\end{capilist}
\label{nxo_class_data_get}
\index{nxo_class_data_get@\cfunc{nxo\_class\_data\_get}{}}
\citem{\cfunc[cw\_nxo\_t *]{nxo\_class\_data\_get}{const cw\_nxo\_t *a\_nxo}}
	\begin{capilist}
	\item[Input(s): ]
		\begin{description}\item[]
		\item[a\_nxo: ]
			Pointer to a class \classname{nxo}.
		\end{description}
	\item[Output(s): ]
		\begin{description}\item[]
		\item[retval: ]
			Pointer to the data object associated with
			\cvar{a\_nxo}.
		\end{description}
	\item[Exception(s): ] None.
	\item[Description: ]
		Return a pointer to the data object associated with
		\cvar{a\_nxo}.  This object pointer can safely be used for
		modifying the data object.
	\end{capilist}
\label{nxo_class_opaque_get}
\index{nxo_class_opaque_get@\cfunc{nxo\_class\_opaque\_get}{}}
\citem{\cfunc[void *]{nxo\_class\_opaque\_get}{const cw\_nxo\_t *a\_nxo}}
	\begin{capilist}
	\item[Input(s): ]
		\begin{description}\item[]
		\item[a\_nxo: ]
			Pointer to a class \classname{nxo}.
		\end{description}
	\item[Output(s): ]
		\begin{description}\item[]
		\item[retval: ]
			Opaque data pointer.
		\end{description}
	\item[Exception(s): ] None.
	\item[Description: ]
		Return the opaque data pointer associated with \cvar{a\_nxo}.
	\end{capilist}
\label{nxo_class_opaque_set}
\index{nxo_class_opaque_set@\cfunc{nxo\_class\_opaque\_set}{}}
\citem{\cfunc[void]{nxo\_class\_opaque\_set}{cw\_nxo\_t *a\_nxo, void
*a\_opaque}}
	\begin{capilist}
	\item[Input(s): ]
		\begin{description}\item[]
		\item[a\_nxo: ]
			Pointer to a class \classname{nxo}.
		\item[a\_opaque: ]
			Opaque data pointer.
		\end{description}
	\item[Output(s): ] None.
	\item[Exception(s): ] None.
	\item[Description: ]
		Set the opaque data pointer associated with \cvar{a\_nxo}.
	\end{capilist}
\label{nxo_class_eval}
\index{nxo_class_eval@\cfunc{nxo\_class\_eval}{}}
\citem{\cfunc[void]{nxo\_class\_eval}{cw\_nxo\_t *a\_nxo, cw\_nxo\_t
*a\_thread}}
	\begin{capilist}
	\item[Input(s): ]
		\begin{description}\item[]
		\item[a\_nxo: ]
			Pointer to a class \classname{nxo}.
		\item[a\_thread: ]
			Pointer to a thread \classname{nxo}.
		\end{description}
	\item[Output(s): ] None.
	\item[Exception(s): ] Class-specific.
	\item[Description: ]
		Evaluate the \cvar{a\_nxo}.  If there is no evaluation function
		associated with \cvar{a\_nxo}, it is pushed onto ostack.
	\end{capilist}
\end{capi}

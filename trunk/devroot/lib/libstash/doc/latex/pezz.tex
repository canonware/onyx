%-*-mode:latex-*-
%%%%%%%%%%%%%%%%%%%%%%%%%%%%%%%%%%%%%%%%%%%%%%%%%%%%%%%%%%%%%%%%%%%%%%%%%%%%%
%
% <Copyright = jasone>
% <License>
%
%%%%%%%%%%%%%%%%%%%%%%%%%%%%%%%%%%%%%%%%%%%%%%%%%%%%%%%%%%%%%%%%%%%%%%%%%%%%%
%
% Version: <Version>
%
% pezz portion of Canonware Software Manual.
%              
%%%%%%%%%%%%%%%%%%%%%%%%%%%%%%%%%%%%%%%%%%%%%%%%%%%%%%%%%%%%%%%%%%%%%%%%%%%%%

\subsection{pezz}
\label{pezz}
\index{\classname{pezz}{}}

The \classname{pezz} class provides cached allocation for equal-size buffers.
It does incremental block allocation, then carves buffers from those blocks.  No
memory is freed until \cfunc{pezz\_delete}{} is called.

In the debug version of \libname{libstash}, tracking information is kept so that
memory leaks and other memory allocation errors can be detected and printed to
\cvar{out\_err}.

\subsubsection{API}
\begin{description}
\label{pezz_new}
\index{\cfunc{pezz\_new}{}}
\item[{\cfunc[cw\_pezz\_t *]{pezz\_new}{cw\_pezz\_t *a\_pezz, cw\_mem\_t
*a\_mem, cw\_uint32\_t a\_buffer\_size, cw\_uint32\_t a\_num\_buffers}}: ]
	\begin{description}\item[]
	\item[Input(s): ]
		\begin{description}\item[]
		\item[a\_pezz: ]
			Pointer to space for a \classname{pezz}, or NULL.
		\item[a\_mem: ]
			Pointer to the allocator to use internally, or NULL.
		\item[a\_buffer\_size: ]
			Size of buffers to allocate and return from
			\cfunc{pezz\_get}{}.
		\item[a\_num\_buffers: ]
			Number of buffers to allocate space for each time a new
			memory block is allocated.
		\end{description}
	\item[Output(s): ]
		\begin{description}\item[]
		\item[retval: ]
			Pointer to a \classname{pezz}.
		\end{description}
	\item[Exception(s): ]
		\begin{description}\item[]
		\item[\htmlref{\_CW\_STASHX\_OOM}{_CW_STASHX_OOM}.]
		\end{description}
	\item[Description: ]
		Constructor.
	\end{description}
\label{pezz_delete}
\index{\cfunc{pezz\_delete}{}}
\item[{\cfunc[void]{pezz\_delete}{cw\_pezz\_t *a\_pezz}}: ]
	\begin{description}\item[]
	\item[Input(s): ]
		\begin{description}\item[]
		\item[a\_pezz: ]
			Pointer to a \classname{pezz}.
		\end{description}
	\item[Output(s): ] None.
	\item[Exception(s): ] None.
	\item[Description: ]
		Destructor.
	\end{description}
\label{pezz_buffer_size_get}
\index{\cfunc{pezz\_buffer\_size\_get}{}}
\item[{\cfunc[cw\_uint32\_t]{pezz\_}{cw\_pezz\_t *a\_pezz}}: ]
	\begin{description}\item[]
	\item[Input(s): ]
		\begin{description}\item[]
		\item[a\_pezz: ]
			Pointer to a \classname{pezz}.
		\end{description}
	\item[Output(s): ]
		\begin{description}\item[]
		\item[retval: ]
			Size of buffers that \cvar{a\_pezz} is using.
		\end{description}
	\item[Exception(s): ] None.
	\item[Description: ]
		Return the size of the buffers that \cvar{a\_pezz} is returning.
	\end{description}
\label{pezz_get_e}
\index{\cfunc{pezz\_get\_e}{}}
\item[{\cfunc[void *]{pezz\_get\_e}{cw\_pezz\_t *a\_pezz, const char
*a\_filename, cw\_uint32\_t a\_line\_num}}: ]
\label{pezz_get}
\index{\cppmacro{pezz\_get}{}}
\item[{\cppmacro[void *]{pezz\_get}{cw\_pezz\_t *a\_pezz}}: ]
	\begin{description}\item[]
	\item[Input(s): ]
		\begin{description}\item[]
		\item[a\_pezz: ]
			Pointer to a \classname{pezz}.
		\item[a\_filename: ]
			Should be \_\_FILE\_\_.
		\item[a\_line\_num: ]
			Should be \_\_LINE\_\_.
		\end{description}
	\item[Output(s): ]
		\begin{description}\item[]
		\item[retval: ]
			Pointer to a memory buffer.
		\end{description}
	\item[Exception(s): ]
		\begin{description}\item[]
		\item[\htmlref{\_CW\_STASHX\_OOM}{_CW_STASHX_OOM}.]
		\end{description}
	\item[Description: ]
		Allocate a memory buffer and return a pointer to it.
	\end{description}
\label{pezz_put_e}
\index{\cfunc{pezz\_put_e}{}}
\item[{\cfunc[void]{pezz\_put\_e}{cw\_pezz\_t *a\_pezz, void *a\_buffer, const
char *a\_filename, cw\_uint32\_t a\_line\_num}}: ]
\label{pezz_put}
\index{\cppmacro{pezz\_put}{}}
\item[{\cppmacro[void]{pezz\_put}{cw\_pezz\_t *a\_pezz, void *a\_buffer}}: ]
	\begin{description}\item[]
	\item[Input(s): ]
		\begin{description}\item[]
		\item[a\_pezz: ]
			Pointer to a \classname{pezz}.
		\item[a\_buffer: ]
			Pointer to a memory buffer.
		\item[a\_filename: ]
			Should be \_\_FILE\_\_.
		\item[a\_line\_num: ]
			Should be \_\_LINE\_\_.
		\end{description}
	\item[Output(s): ] None.
	\item[Exception(s): ] None.
	\item[Description: ]
		Put back (deallocate) \cvar{a\_buffer}.
	\end{description}
\label{pezz_dump}
\index{\cfunc{pezz\_dump}{}}
\item[{\cfunc[void]{pezz\_dump}{cw\_pezz\_t *a\_pezz, const char *a\_prefix}}: ]
	\begin{description}\item[]
	\item[Input(s): ]
		\begin{description}\item[]
		\item[a\_pezz: ]
			Pointer to a \classname{pezz}.
		\item[a\_prefix: ]
			Pointer to a string which is used as a prefix for all
			lines of output.
		\end{description}
	\item[Output(s): ]
		Output printed to \cvar{out\_err}.
	\item[Exception(s): ]
		\begin{description}\item[]
		\item[\htmlref{\_CW\_STASHX\_OOM}{_CW_STASHX_OOM}.]
		\end{description}
	\item[Description: ]
		Dump the internal state of \cvar{a\_pezz} to \cvar{out\_err}.
	\end{description}
\end{description}

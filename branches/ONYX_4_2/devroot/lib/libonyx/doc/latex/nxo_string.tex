%-*-mode:latex-*-
%%%%%%%%%%%%%%%%%%%%%%%%%%%%%%%%%%%%%%%%%%%%%%%%%%%%%%%%%%%%%%%%%%%%%%%%%%%%%
%
% <Copyright = jasone>
% <License>
%
%%%%%%%%%%%%%%%%%%%%%%%%%%%%%%%%%%%%%%%%%%%%%%%%%%%%%%%%%%%%%%%%%%%%%%%%%%%%%
%
% Version: Onyx <Version = onyx>
%
% nxo_string portion of Onyx Manual.
%              
%%%%%%%%%%%%%%%%%%%%%%%%%%%%%%%%%%%%%%%%%%%%%%%%%%%%%%%%%%%%%%%%%%%%%%%%%%%%%

\subsection{nxo\_string}
\label{nxo_string}
\index{nxo_string@\classname{nxo\_string}{}}

The \classname{nxo\_string} class is a subclass of the \classname{nxo} class.
Strings are not `{\bs}0'-terminated, mainly since substrings are references to
other strings, and the termination character wouldn't be consistently useful.
\cfunc{nxo\_string\_cstring}{} is useful for creating `{\bs}0'-terminated copies
of strings for situations where other C functions expect terminated strings.

\subsubsection{API}
\begin{capi}
\label{nxo_string_new}
\index{nxo_string_new@\cfunc{nxo\_string\_new}{}}
\citem{\cfunc[void]{nxo\_string\_new}{cw\_nxo\_t *a\_nxo, cw\_nx\_t *a\_nx,
cw\_bool\_t a\_locking, cw\_uint32\_t a\_len}}
	\begin{capilist}
	\item[Input(s): ]
		\begin{description}\item[]
		\item[a\_nxo: ]
			Pointer to a string \classname{nxo}.
		\item[a\_nx: ]
			Pointer to an \classname{nx}.
		\item[a\_locking: ]
			Implicit locking mode.
		\item[a\_len: ]
			Length in bytes of string to create.
		\end{description}
	\item[Output(s): ] None.
	\item[Exception(s): ]
		\begin{description}\item[]
		\item[\htmlref{CW\_ONYXX\_OOM}{CW_ONYXX_OOM}.]
		\end{description}
	\item[Description: ]
		Constructor.
	\end{capilist}
\label{nxo_string_substring_new}
\index{nxo_string_substring_new@\cfunc{nxo\_string\_substring\_new}{}}
\citem{\cfunc[void]{nxo\_string\_substring\_new}{cw\_nxo\_t *a\_nxo, cw\_nxo\_t
*a\_string, cw\_nx\_t *a\_nx, cw\_uint32\_t a\_offset, cw\_uint32\_t a\_len}}
	\begin{capilist}
	\item[Input(s): ]
		\begin{description}\item[]
		\item[a\_nxo: ]
			Pointer to a string \classname{nxo}.
		\item[a\_string: ]
			Pointer to a string \classname{nxo} to create a
			substring of.
		\item[a\_nx: ]
			Pointer to an \classname{nx}.
		\item[a\_offset: ]
			Offset into \cvar{a\_string}.
		\item[a\_len: ]
			Length in bytes of substring to create.
		\end{description}
	\item[Output(s): ] None.
	\item[Exception(s): ]
		\begin{description}\item[]
		\item[\htmlref{CW\_ONYXX\_OOM}{CW_ONYXX_OOM}.]
		\end{description}
	\item[Description: ]
		Substring constructor.
	\end{capilist}
\label{nxo_string_copy}
\index{nxo_string_copy@\cfunc{nxo\_string\_copy}{}}
\citem{\cfunc[void]{nxo\_string\_copy}{cw\_nxo\_t *a\_to, cw\_nxo\_t *a\_from}}
	\begin{capilist}
	\item[Input(s): ]
		\begin{description}\item[]
		\item[a\_to: ]
			Pointer to a string \classname{nxo}.
		\item[a\_from: ]
			Pointer to a string \classname{nxo}.
		\end{description}
	\item[Output(s): ] None.
	\item[Exception(s): ] None.
	\item[Description: ]
		Copy the contents of \cvar{a\_from} to \cvar{a\_to}.  The length
		of \cvar{a\_to} must be at least that of \cvar{a\_from}.
	\end{capilist}
\label{nxo_string_cstring}
\index{nxo_string_cstring@\cfunc{nxo\_string\_cstring}{}}
\citem{\cfunc[void]{nxo\_string\_cstring}{cw\_nxo\_t *a\_to, cw\_nxo\_t
*a\_from, cw\_nxo\_t *a\_thread}}
	\begin{capilist}
	\item[Input(s): ]
		\begin{description}\item[]
		\item[a\_to: ]
			Pointer to an \classname{nxo}.
		\item[a\_from: ]
			Pointer to a string or name \classname{nxo}.
		\item[a\_thread: ]
			Pointer to a thread \classname{nxo}.
		\end{description}
	\item[Output(s): ] None.
	\item[Exception(s): ]
		\begin{description}\item[]
		\item[\htmlref{CW\_ONYXX\_OOM}{CW_ONYXX_OOM}.]
		\end{description}
	\item[Description: ]
		Create a copy of \cvar{a\_from}, but append a `{\bs}0' character
		to make it usable in calls to typical C functions that expect a
		terminated string.
	\end{capilist}
\label{nxo_string_len_get}
\index{nxo_string_len_get@\cfunc{nxo\_string\_len\_get}{}}
\citem{\cfunc[cw\_uint32\_t]{nxo\_string\_len\_get}{const cw\_nxo\_t *a\_nxo}}
	\begin{capilist}
	\item[Input(s): ]
		\begin{description}\item[]
		\item[a\_nxo: ]
			Pointer to a string \classname{nxo}.
		\end{description}
	\item[Output(s): ]
		\begin{description}\item[]
		\item[retval: ]
			Length of \cvar{a\_nxo}.
		\end{description}
	\item[Exception(s): ] None.
	\item[Description: ]
		Return the length of \cvar{a\_nxo}.
	\end{capilist}
\label{nxo_string_el_get}
\index{nxo_string_el_get@\cfunc{nxo\_string\_el\_get}{}}
\citem{\cfunc[void]{nxo\_string\_el\_get}{const cw\_nxo\_t *a\_nxo, cw\_nxoi\_t
a\_offset, cw\_uint8\_t *r\_el}}
	\begin{capilist}
	\item[Input(s): ]
		\begin{description}\item[]
		\item[a\_nxo: ]
			Pointer to a string \classname{nxo}.
		\item[a\_offset: ]
			Offset of character to get.
		\item[r\_el: ]
			Pointer to space to copy a character to.
		\end{description}
	\item[Output(s): ]
		\begin{description}\item[]
		\item[*r\_el: ]
			A copy of the character of \cvar{a\_nxo} at offset
			\cvar{a\_offset}.
		\end{description}
	\item[Exception(s): ] None.
	\item[Description: ]
		Get a copy of the character of \cvar{a\_nxo} at offset
		\cvar{a\_offset}.
	\end{capilist}
\label{nxo_string_el_set}
\index{nxo_string_el_set@\cfunc{nxo\_string\_el\_set}{}}
\citem{\cfunc[void]{nxo\_string\_el\_set}{cw\_nxo\_t *a\_nxo, cw\_uint8\_t
a\_el, cw\_nxoi\_t a\_offset}}
	\begin{capilist}
	\item[Input(s): ]
		\begin{description}\item[]
		\item[a\_nxo: ]
			Pointer to a string \classname{nxo}.
		\item[a\_el: ]
			A character.
		\item[a\_offset: ]
			Offset of character in \cvar{a\_nxo} to replace with
			\cvar{a\_el}.
		\end{description}
	\item[Output(s): ] None.
	\item[Exception(s): ] None.
	\item[Description: ]
		Copy \cvar{a\_el} into the element of \cvar{a\_nxo} at offset
		\cvar{a\_offset}.
	\end{capilist}
\label{nxo_string_lock}
\index{nxo_string_lock@\cfunc{nxo\_string\_lock}{}}
\citem{\cfunc[void]{nxo\_string\_lock}{cw\_nxo\_t *a\_nxo}}
	\begin{capilist}
	\item[Input(s): ]
		\begin{description}\item[]
		\item[a\_nxo: ]
			Pointer to a string \classname{nxo}.
		\end{description}
	\item[Output(s): ] None.
	\item[Exception(s): ] None.
	\item[Description: ]
		If implicit locking is activated for \cvar{a\_nxo}, lock it.
	\end{capilist}
\label{nxo_string_unlock}
\index{nxo_string_unlock@\cfunc{nxo\_string\_unlock}{}}
\citem{\cfunc[void]{nxo\_string\_unlock}{cw\_nxo\_t *a\_nxo}}
	\begin{capilist}
	\item[Input(s): ]
		\begin{description}\item[]
		\item[a\_nxo: ]
			Pointer to a string \classname{nxo}.
		\end{description}
	\item[Output(s): ] None.
	\item[Exception(s): ] None.
	\item[Description: ]
		If implicit locking is activated for \cvar{a\_nxo}, unlock it.
	\end{capilist}
\label{nxo_string_get}
\index{nxo_string_get@\cfunc{nxo\_string\_get}{}}
\citem{\cfunc[cw\_uint8\_t *]{nxo\_string\_get}{const cw\_nxo\_t *a\_nxo}}
	\begin{capilist}
	\item[Input(s): ]
		\begin{description}\item[]
		\item[a\_nxo: ]
			Pointer to a string \classname{nxo}.
		\end{description}
	\item[Output(s): ]
		\begin{description}\item[]
		\item[retval: ]
			Pointer to the string internal to \cvar{a\_nxo}.
		\end{description}
	\item[Exception(s): ] None.
	\item[Description: ]
		Return a pointer to the string internal to \cvar{a\_nxo}.
	\end{capilist}
\label{nxo_string_set}
\index{nxo_string_set@\cfunc{nxo\_string\_set}{}}
\citem{\cfunc[void]{nxo\_string\_set}{cw\_nxo\_t *a\_nxo, cw\_uint32\_t
a\_offset, const cw\_uint8\_t *a\_str, cw\_uint32\_t a\_len}}
	\begin{capilist}
	\item[Input(s): ]
		\begin{description}\item[]
		\item[a\_nxo: ]
			Pointer to a string \classname{nxo}.
		\item[a\_offset: ]
			Offset into \cvar{a\_nxo} to replace.
		\item[a\_str: ]
			String to replace a range of \cvar{a\_nxo} with.
		\item[a\_len: ]
			Length in bytes of \cvar{a\_str}.
		\end{description}
	\item[Output(s): ] None.
	\item[Exception(s): ] None.
	\item[Description: ]
		Replace \cvar{a\_len} bytes of \cvar{a\_nxo} at offset
		\cvar{a\_offset} with \cvar{a\_str}.
	\end{capilist}
\end{capi}

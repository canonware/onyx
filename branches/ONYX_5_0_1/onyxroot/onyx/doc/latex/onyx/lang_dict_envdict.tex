%-*-mode:latex-*-
%%%%%%%%%%%%%%%%%%%%%%%%%%%%%%%%%%%%%%%%%%%%%%%%%%%%%%%%%%%%%%%%%%%%%%%%%%%%%
%
% <Copyright = jasone>
% <License>
%
%%%%%%%%%%%%%%%%%%%%%%%%%%%%%%%%%%%%%%%%%%%%%%%%%%%%%%%%%%%%%%%%%%%%%%%%%%%%%
%
% Version: Onyx <Version = onyx>
%
% envdict reference portion of Onyx Manual.
%
%%%%%%%%%%%%%%%%%%%%%%%%%%%%%%%%%%%%%%%%%%%%%%%%%%%%%%%%%%%%%%%%%%%%%%%%%%%%%

\subsection{envdict}
\label{sec:envdict}
\index{envdict@\onyxop{}{envdict}{}}

The envdict dictionary contains keys of type name and values of type string that
correspond to the environment passed into the program.  All threads share the
same envdict, which is implicitly locked.  Modifications to envdict should be
made via the \htmlref{\onyxop{}{setenv}{}}{systemdict:setenv} and
\htmlref{\onyxop{}{unsetenv}{}}{systemdict:unsetenv} operators.  If envdict is
modified directly, the changes will not be visible to programs such as
\binname{ps}.

%-*-mode:latex-*-
%%%%%%%%%%%%%%%%%%%%%%%%%%%%%%%%%%%%%%%%%%%%%%%%%%%%%%%%%%%%%%%%%%%%%%%%%%%%%
%
% <Copyright = jasone>
% <License>
%
%%%%%%%%%%%%%%%%%%%%%%%%%%%%%%%%%%%%%%%%%%%%%%%%%%%%%%%%%%%%%%%%%%%%%%%%%%%%%
%
% Version: Onyx <Version = onyx>
%
% nxo portion of Onyx Manual.
%              
%%%%%%%%%%%%%%%%%%%%%%%%%%%%%%%%%%%%%%%%%%%%%%%%%%%%%%%%%%%%%%%%%%%%%%%%%%%%%

\subsection{nxo}
\label{nxo}
\index{nxo@\classname{nxo}{}}

The \classname{nxo} class is the basis for the Onyx type system.
\classname{nxo} objects can be any of the following types, as determined by the
\ctype{cw\_nxot\_t} type:
\begin{description}
\item[NXOT\_NO: ] \htmlref{\classname{nxo\_no}}{nxo_no}
\item[NXOT\_ARRAY: ] \htmlref{\classname{nxo\_array}}{nxo_array}
\item[NXOT\_BOOLEAN: ] \htmlref{\classname{nxo\_boolean}}{nxo_boolean}
\item[NXOT\_CONDITION: ] \htmlref{\classname{nxo\_condition}}{nxo_condition}
\item[NXOT\_DICT: ] \htmlref{\classname{nxo\_dict}}{nxo_dict}
\item[NXOT\_FILE: ] \htmlref{\classname{nxo\_file}}{nxo_file}
\item[NXOT\_FINO: ] \htmlref{\classname{nxo\_fino}}{nxo_fino}
\item[NXOT\_HOOK: ] \htmlref{\classname{nxo\_hook}}{nxo_hook}
\item[NXOT\_INTEGER: ] \htmlref{\classname{nxo\_integer}}{nxo_integer}
\item[NXOT\_MARK: ] \htmlref{\classname{nxo\_mark}}{nxo_mark}
\item[NXOT\_MUTEX: ] \htmlref{\classname{nxo\_mutex}}{nxo_mutex}
\item[NXOT\_NAME: ] \htmlref{\classname{nxo\_name}}{nxo_name}
\item[NXOT\_NULL: ] \htmlref{\classname{nxo\_null}}{nxo_null}
\item[NXOT\_OPERATOR: ] \htmlref{\classname{nxo\_operator}}{nxo_operator}
\item[NXOT\_PMARK: ] \htmlref{\classname{nxo\_pmark}}{nxo_pmark}
\item[NXOT\_REAL: ] \htmlref{\classname{nxo\_real}}{nxo_real}
\item[NXOT\_STACK: ] \htmlref{\classname{nxo\_stack}}{nxo_stack}
\item[NXOT\_STRING: ] \htmlref{\classname{nxo\_string}}{nxo_string}
\item[NXOT\_THREAD: ] \htmlref{\classname{nxo\_thread}}{nxo_thread}
\end{description}

Due to limitations of the C programming language, it is the responsibility of
the application to do type checking to assure that an incompatible
\classname{nxo} object is not passed to a type-specific function.  For example,
passing a file \classname{nxo} to \cfunc{nxo\_string\_get}{} is prohibited, and
will result in undefined behaviour (including crashes).

Composite objects contain a reference to an \classname{nxoe} object.  For the
most part, the application does not need to be aware of this.  The only
exception is when writing extensions with the hook type.  Hook objects need to
be able to iterate over the objects they reference internally, and return
\classname{nxoe} references to the garbage collector.

The following functions are applicable to all types of \classname{nxo} objects.

\subsubsection{API}
\begin{capi}
\label{nxo_compare}
\index{nxo_compare@\cfunc{nxo\_compare}{}}
\citem{\cfunc[cw\_sint32\_t]{nxo\_compare}{const cw\_nxo\_t *a\_a, const
cw\_nxo\_t *a\_b}}
	\begin{capilist}
	\item[Input(s): ]
		\begin{description}\item[]
		\item[a\_a: ]
			Pointer to an \classname{nxo}.
		\item[a\_b: ]
			Pointer to an \classname{nxo}.
		\end{description}
	\item[Output(s): ]
		\begin{description}\item[]
		\item[retval: ]
			\begin{description}\item[]
			\item[-1: ]
				For types which it is meaningful (integer,
				string), \cvar{a\_a} is less than \cvar{a\_b}.
			\item[0: ]
				\cvar{a\_a} and \cvar{a\_b} are equal.
			\item[1: ]
				For types which it is meaningful (integer,
				string), \cvar{a\_a} is greater than
				\cvar{a\_b}.
			\item[2: ]
				Incompatible types, or not the same composite
				object.
			\end{description}
		\end{description}
	\item[Exception(s): ] None.
	\item[Description: ]
		Compare \cvar{a\_a} and \cvar{a\_b}.
	\end{capilist}
\label{nxo_dup}
\index{nxo_dup@\cfunc{nxo\_dup}{}}
\citem{\cfunc[void]{nxo\_dup}{cw\_nxo\_t *a\_to, cw\_nxo\_t *a\_from}}
	\begin{capilist}
	\item[Input(s): ]
		\begin{description}\item[]
		\item[a\_to: ]
			Pointer to an \classname{nxo}.
		\item[a\_from: ]
			Pointer to an \classname{nxo}.
		\end{description}
	\item[Output(s): ] None.
	\item[Exception(s): ] None.
	\item[Description: ]
		Duplicate \cvar{a\_from} to \cvar{a\_to}.  This does not do a
		copy of composite objects; rather it creates a new reference to
		the value of a composite object.
	\end{capilist}
\label{nxo_type_get}
\index{nxo_type_get@\cfunc{nxo\_type\_get}{}}
\citem{\cfunc[cw\_nxot\_t]{nxo\_type\_get}{const cw\_nxo\_t *a\_nxo}}
	\begin{capilist}
	\item[Input(s): ]
		\begin{description}\item[]
		\item[a\_nxo: ]
			Pointer to an \classname{nxo}.
		\end{description}
	\item[Output(s): ]
		\begin{description}\item[]
		\item[retval: ]
			\begin{description}\item[]
				\item[NXOT\_NO: ]
					\htmlref{\classname{nxo\_no}}{nxo_no}
				\item[NXOT\_ARRAY: ]
					\htmlref{\classname{nxo\_array}}
					{nxo_array}
				\item[NXOT\_BOOLEAN: ]
					\htmlref{\classname{nxo\_boolean}}
					{nxo_boolean}
				\item[NXOT\_CONDITION: ]
					\htmlref{\classname{nxo\_condition}}
					{nxo_condition}
				\item[NXOT\_DICT: ]
					\htmlref{\classname{nxo\_dict}}
					{nxo_dict}
				\item[NXOT\_FILE: ]
					\htmlref{\classname{nxo\_file}}
					{nxo_file}
				\item[NXOT\_FINO: ]
					\htmlref{\classname{nxo\_fino}}
					{nxo_fino}
				\item[NXOT\_HOOK: ]
					\htmlref{\classname{nxo\_hook}}
					{nxo_hook}
				\item[NXOT\_INTEGER: ]
					\htmlref{\classname{nxo\_integer}}
					{nxo_integer}
				\item[NXOT\_MARK: ]
					\htmlref{\classname{nxo\_mark}}
					{nxo_mark}
				\item[NXOT\_MUTEX: ]
					\htmlref{\classname{nxo\_mutex}}
					{nxo_mutex}
				\item[NXOT\_NAME: ]
					\htmlref{\classname{nxo\_name}}
					{nxo_name}
				\item[NXOT\_NULL: ]
					\htmlref{\classname{nxo\_null}}
					{nxo_null}
				\item[NXOT\_OPERATOR: ]
					\htmlref{\classname{nxo\_operator}}
					{nxo_operator}
				\item[NXOT\_PMARK: ]
					\htmlref{\classname{nxo\_pmark}}
					{nxo_pmark}
				\item[NXOT\_REAL: ]
					\htmlref{\classname{nxo\_real}}
					{nxo_real}
				\item[NXOT\_STACK: ]
					\htmlref{\classname{nxo\_stack}}
					{nxo_stack}
				\item[NXOT\_STRING: ]
					\htmlref{\classname{nxo\_string}}
					{nxo_string}
				\item[NXOT\_THREAD: ]
					\htmlref{\classname{nxo\_thread}}
					{nxo_thread}
			\end{description}
		\end{description}
	\item[Exception(s): ] None.
	\item[Description: ]
		Return the type of \cvar{a\_nxo}.
	\end{capilist}
\label{nxo_nxoe_get}
\index{nxo_nxoe_get@\cfunc{nxo\_nxoe\_get}{}}
\citem{\cfunc[cw\_nxoe\_t *]{nxo\_nxoe\_get}{const cw\_nxo\_t *a\_nxo}}
	\begin{capilist}
	\item[Input(s): ]
		\begin{description}\item[]
		\item[a\_nxo: ]
			Pointer to an \classname{nxo}.
		\end{description}
	\item[Output(s): ]
		\begin{description}\item[]
		\item[retval: ]
			Pointer to the \classname{nxoe} associated with
			\cvar{a\_nxo}, or NULL if \cvar{a\_nxo} is not
			composite.
		\end{description}
	\item[Exception(s): ] None.
	\item[Description: ]
		Return a pointer to the \classname{nxoe} associated with
		\cvar{a\_nxo}.
	\end{capilist}
\label{nxo_lcheck}
\index{nxo_lcheck@\cfunc{nxo\_lcheck}{}}
\citem{\cfunc[cw\_bool\_t]{nxo\_lcheck}{}}
	\begin{capilist}
	\item[Input(s): ]
		\begin{description}\item[]
		\item[a\_nxo: ]
			Pointer to an array, dict, file, stack, or string
			\classname{nxo}.
		\end{description}
	\item[Output(s): ]
		\begin{description}\item[]
		\item[retval: ]
			\begin{description}\item[]
			\item[FALSE: ]
				\cvar{a\_nxo} is not implicitly locked.
			\item[TRUE: ]
				\cvar{a\_nxo} is implicitly locked.
			\end{description}
		\end{description}
	\item[Exception(s): ] None.
	\item[Description: ]
		For array, dict, file, stack, or string \classname{nxo\/}s,
		return whether \cvar{a\_nxo} is implicitly locked.
	\end{capilist}
\label{nxo_attr_get}
\index{nxo_attr_get@\cfunc{nxo\_attr\_get}{}}
\citem{\cfunc[cw\_nxoa\_t]{nxo\_attr\_get}{const cw\_nxo\_t *a\_nxo}}
	\begin{capilist}
	\item[Input(s): ]
		\begin{description}\item[]
		\item[a\_nxo: ]
			Pointer to an \classname{nxo}.
		\end{description}
	\item[Output(s): ]
		\begin{description}\item[]
		\item[retval: ]
			\begin{description}\item[]
			\item[NXOA\_LITERAL: ]
				\cvar{a\_nxo} is literal.
			\item[NXOA\_EXECUTABLE: ]
				\cvar{a\_nxo} is executable.
			\end{description}
		\end{description}
	\item[Exception(s): ] None.
	\item[Description: ]
		Return the attribute for \cvar{a\_nxo}.
	\end{capilist}
\label{nxo_attr_set}
\index{nxo_attr_set@\cfunc{nxo\_attr\_set}{}}
\citem{\cfunc[void]{nxo\_attr\_set}{cw\_nxo\_t *a\_nxo, cw\_nxoa\_t a\_attr}}
	\begin{capilist}
	\item[Input(s): ]
		\begin{description}\item[]
		\item[a\_nxo: ]
			Pointer to an \classname{nxo}.
		\item[a\_attr: ]
			Value of attribute to set for \cvar{a\_nxo}.
		\end{description}
	\item[Output(s): ] None.
	\item[Exception(s): ] None.
	\item[Description: ]
		Set the attribute for \cvar{a\_nxo} to \cvar{a\_attr}.
	\end{capilist}
\end{capi}

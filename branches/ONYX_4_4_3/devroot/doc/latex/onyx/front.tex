%-*-mode:latex-*-
%%%%%%%%%%%%%%%%%%%%%%%%%%%%%%%%%%%%%%%%%%%%%%%%%%%%%%%%%%%%%%%%%%%%%%%%%%%%%
%
% <Copyright = jasone>
% <License>
%
%%%%%%%%%%%%%%%%%%%%%%%%%%%%%%%%%%%%%%%%%%%%%%%%%%%%%%%%%%%%%%%%%%%%%%%%%%%%%
%
% Version: Onyx <Version = onyx>
%
% Definitions and front matter of Onyx Manual.
%              
%%%%%%%%%%%%%%%%%%%%%%%%%%%%%%%%%%%%%%%%%%%%%%%%%%%%%%%%%%%%%%%%%%%%%%%%%%%%%

\documentclass[10pt,titlepage]{book}
\usepackage[dvips]{graphicx}
\usepackage{longtable}
\usepackage{float}
\usepackage{html}
\usepackage{makeidx}
%begin{latexonly}
\usepackage{newcent}
\usepackage{fancyheadings}
\usepackage{ifthen}
%end{latexonly}

%begin{latexonly}
%
% Configure margins, line spacing, etc.
%
\setlength{\topmargin}{0in}
%\setlength{\headheight}{36pt}
\setlength{\headheight}{54pt}
\setlength{\headsep}{12pt}
\setlength{\topskip}{0in}
\setlength{\textheight}{8.5in}
\setlength{\oddsidemargin}{0in}
\setlength{\evensidemargin}{0in}
\setlength{\textwidth}{6.5in}
\setlength{\parindent}{0pt}
%\setlength{\parskip}{10pt}
\setlength{\parskip}{8pt}
%\setlength{\topsep}{0pt}
%\setlength{\partopsep}{0pt}
%\renewcommand{\baselinestretch}{1.3}
\renewcommand{\baselinestretch}{1.0}
%end{latexonly}

%
% Title page setup
%
\title{Onyx Manual, Version <Version = onyx>}
\author{Jason Evans}
\date{\today}

% Control what all ends up in the table of contents.
%\addtocounter{secnumdepth}{4}

%begin{latexonly}
% Use this to make sure chapter breaks look right.
\newcommand{\clearemptydoublepage}
	{\newpage \thispagestyle{empty} \cleardoublepage}

%
% Set up the standard page headers.
%
%\renewcommand{\chaptermark}[1]{\markboth{#1}{}}
%\renewcommand{\sectionmark}[1]{\markright{\thesection\ #1}}
%\lhead[\bfseries\thepage]{\bfseries\rightmark}
%\rhead[\bfseries{Chapter \thechapter}]{\bfseries\thepage}
%\cfoot{}
%\setlength{\headrulewidth}{2pt}
%\chead[\bfseries{Onyx Manual}]{\bfseries{Jason Evans}}
%end{latexonly}

%begin{latexonly}
\newenvironment{capilist}
	{
		\raggedright
		\begin{list}{}
		{
			\listparindent 0pt
			\itemindent \listparindent
			\item\relax
		}
		\begin{description}
	}
	{
		\end{description}
		\end{list}
		\vspace{8pt}
	}
%end{latexonly}
\begin{htmlonly}
\newenvironment{capilist}
	{
		\begin{description}
		\begin{description}
	}
	{
		\end{description}
		\end{description}
	}
\end{htmlonly}

\newcommand{\citem}[1]{\bfseries #1: \normalfont

}
\newenvironment{capi}
	{\setlength{\parskip}{0pt}}
	{\setlength{\parskip}{8pt}}

\providecommand{\ifthenelse}[3]{ } % For latex2html.
\newcommand{\cfunc}[3][]{#1\ifthenelse{\equal{}{#1}}{}{ }{\em #2}(#3)}
\newcommand{\cvar}[1]{{\em #1}}
\newcommand{\ctype}[1]{#1}
\newcommand{\cppmacro}[3][]{#1\ifthenelse{\equal{}{#1}}{}{ }{\em #2}(#3)}
\newcommand{\cppdef}[1]{{\em #1}}
\newcommand{\binname}[1]{{\em #1}}
\newcommand{\libname}[1]{{\em #1}}
\newcommand{\filename}[1]{{\em #1}}
\newcommand{\classname}[1]{{\em #1}}
\newcommand{\dbgsym}[1]{{``#1''}}
\newcommand{\onyxop}[3]{{\em #1}\ifthenelse{\equal{}{#1}}{}{ }{\bf #2}\ifthenelse{\equal{}{#3}}{}{ }{\em #3}}
\newcommand{\oparg}[1]{{\em #1}}
\newcommand{\commas}{,,,}

% \optableent{inputs}{operator}{outputs}{description}
\newcommand{\optableent}[4]{\parbox[t]{2.0in}{#1 \\
\hspace*{0.25in} \parbox[t]{1.75in}{#2} \\
\hspace*{0.50in} \parbox[t]{1.50in}{#3 \vspace*{2pt}}} & #4 \\
}

% For printed versions, we need to limit the width of tables, but for html
% there is no need to.
\newcommand{\optableformat}[1]{|l|p{#1in}|}
\newcommand{\rxtableformat}[1]{|r|p{#1in}|}

% TeX doesn't have '\', '^', '<', '>', '|', and '~' in its text fonts.
\newcommand{\bs}{$\backslash$}
\newcommand{\carat}{\^{ }}
\newcommand{\lt}{$<$}
\newcommand{\gt}{$>$}
\newcommand{\pipe}{$|$}
\newcommand{\twid}{$\sim$}
\newcommand{\lb}{[}
\newcommand{\rb}{]}

% The build system does substitution of the YYY..ZZZ strings after the html
% is generated.
\begin{htmlonly}
\newcommand{\optableformat}[1]{|l|l|}
\newcommand{\rxtableformat}[1]{|r|l|}
\newcommand{\bs}{YYYbsZZZ}
\newcommand{\carat}{^}
\newcommand{\lt}{<}
\newcommand{\gt}{>}
\newcommand{\pipe}{|}
\newcommand{\twid}{\~}
\newcommand{\lb}{YYYlbZZZ}
\newcommand{\rb}{YYYrbZZZ}
\end{htmlonly}

% Generate an index.
\makeindex

\begin{document}
\frontmatter
\pagestyle{plain}
\maketitle

\clearemptydoublepage
\begin{htmlonly}
\part*{Preface}
\end{htmlonly}
\begin{latexonly}
\chapter*{Preface}
\end{latexonly}

This manual primarily documents the Onyx programming language.  However, Onyx is
designed to be run either as a stand alone program or as an embeddable
interpreter, so the manual also documents different aspects of the
implementation that are important when embedding Onyx into another program.

Onyx came in to existence when the author started working on a text editor named
slate (still in development) that was meant to be extensible, much in the same
way as \htmladdnormallink{GNU
emacs}{http://www.gnu.org/software/emacs/emacs.html},
\htmladdnormallink{JED}{http://space.mit.edu/~davis/jed/}, and
\htmladdnormallink{Jade}{http://www.dcs.warwick.ac.uk/~john/sw/jade/}.  One of
the goals was to provide robust multi-threading in slate in order to make it
simple to avoid the long pauses that afflict, for example, users of the
\htmladdnormallink{gnus}{http://www.gnus.org/} news/mail reader, which is part
of emacs.  Unfortunately, when work began on slate in 1999, the author was
unable to find any embeddable scripting languages that provided adequate support
for threads.  Thus Onyx was born.  The author was familiar and enamored with
Adobe's PostScript$_{TM}$ language, which has basic threading support when used
in a Display PostScript$_{TM}$ environment, so Onyx started off looking very
similar.  As Onyx matured, it deviated to the point that it is now a truly
different language, with different syntax, additional and more powerful data
types, better debugging capabilities, POSIX-related functionality, more powerful
threading, regular expressions, etc.

As this project grew far beyond what was originally expected, it became clear
that in order to justify the effort being put into Onyx's design and
implementation, Onyx would have to be usable for more than just slate, or else
slate would have to become {\em very} popular, which seems unlikely, given the
plethora of text editors.  Therefore, Onyx has been structured such that it can
be configured in a myriad of ways, with the hope that others will be able to
easily make it fit their needs.  This manual documents Onyx in its full glory
without mention that features may be disabled, so there are portions that do not
apply to Onyx interpreters that have been configured without Onyx's full feature
set.

For software distributions, news, and additional project information, see
\htmlurl{http://www.canonware.com/}.

\clearemptydoublepage
\tableofcontents
\begin{latexonly}
\listoftables
\end{latexonly}

%
% Setup for main body of document.
%
\mainmatter

%
% Set up the standard page headers.
%
%begin{latexonly}
\renewcommand{\chaptermark}[1]{\markboth{#1}{}}
\renewcommand{\sectionmark}[1]{\markright{\thesection\ #1}}
\lhead[\bfseries\thepage]{\bfseries\rightmark}
\rhead[\bfseries{Chapter \thechapter}]{\bfseries\thepage}
\cfoot{}
\setlength{\headrulewidth}{2pt}
\chead[\bfseries{Onyx Manual}]{\bfseries{Jason Evans}}
\pagestyle{fancy}
%end{latexonly}

%-*-mode:latex-*-
%%%%%%%%%%%%%%%%%%%%%%%%%%%%%%%%%%%%%%%%%%%%%%%%%%%%%%%%%%%%%%%%%%%%%%%%%%%%%
%
% <Copyright = jasone>
% <License>
%
%%%%%%%%%%%%%%%%%%%%%%%%%%%%%%%%%%%%%%%%%%%%%%%%%%%%%%%%%%%%%%%%%%%%%%%%%%%%%
%
% Version: <Version>
%
% nxn portion of Onyx Manual.
%              
%%%%%%%%%%%%%%%%%%%%%%%%%%%%%%%%%%%%%%%%%%%%%%%%%%%%%%%%%%%%%%%%%%%%%%%%%%%%%

\subsection{nxn}
\label{nxn}
\index{nxn@\classname{nxn}{}}

The \classname{nxn} class provides access to a table of string constants.  The
main reason for this class's existence is that multiple C files often use
identical string constants, and this saves memory by allowing all to refer to a
single string.

\subsubsection{API}
\begin{capi}
\label{nxn_str}
\index{nxn_str@\cfunc{nxn\_str}{}}
\citem{\cfunc[const cw\_uint8\_t *]{nxn\_str}{cw\_nxn\_t a\_nxn}}
	\begin{capilist}
	\item[Input(s): ]
		\begin{description}\item[]
		\item[a\_nxn: ]
			A number that corresponds to an entry in the string
			table.
		\end{description}
	\item[Output(s): ]
		\begin{description}\item[]
		\item[retval: ]
			Pointer to a string constant.
		\end{description}
	\item[Exception(s): ] None.
	\item[Description: ]
		Return a pointer to the string constant associated with
		\cvar{a\_nxn}.
	\end{capilist}
\label{nxn_len}
\index{nxn_len@\cfunc{nxn\_len}{}}
\citem{\cfunc[cw\_uint32\_t]{nxn\_len}{cw\_nxn\_t a\_nxn}}
	\begin{capilist}
	\item[Input(s): ]
		\begin{description}\item[]
		\item[a\_nxn: ]
			A number that corresponds to an entry in the string
			table.
		\end{description}
	\item[Output(s): ]
		\begin{description}\item[]
		\item[retval: ]
			String length of a string constant.
		\end{description}
	\item[Exception(s): ] None.
	\item[Description: ]
		Return the string length of the string constant associated with
		\cvar{a\_nxn}.
	\end{capilist}
\end{capi}

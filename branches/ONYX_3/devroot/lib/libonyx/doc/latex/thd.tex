%-*-mode:latex-*-
%%%%%%%%%%%%%%%%%%%%%%%%%%%%%%%%%%%%%%%%%%%%%%%%%%%%%%%%%%%%%%%%%%%%%%%%%%%%%
%
% <Copyright = jasone>
% <License>
%
%%%%%%%%%%%%%%%%%%%%%%%%%%%%%%%%%%%%%%%%%%%%%%%%%%%%%%%%%%%%%%%%%%%%%%%%%%%%%
%
% Version: Onyx <Version = onyx>
%
% thd portion of Onyx Manual.
%              
%%%%%%%%%%%%%%%%%%%%%%%%%%%%%%%%%%%%%%%%%%%%%%%%%%%%%%%%%%%%%%%%%%%%%%%%%%%%%

\subsection{thd}
\label{thd}
\index{thd@\classname{thd}{}}

The \classname{thd} class implements a wrapper around the system threads library
(POSIX threads only, so far).  In most regards, this is a thin wrapper around
the normal threads functionality provided by the system, but some extra
information is kept in order to allow implmentation of thread
suspension/resumption, ``critical sections'', and ``single sections''.

In most cases, the additional functionality is implemented with the aid of
signals.  As a result, system calls may be interrupted by signals.  The system
calls will be automaticalaly restarted if they have made no progress at the time
of interruption, but will return a partial result otherwise.  Therefore, if any
of the additional functionality is utilized, the application must be careful to
handle partial system call results.  At least the following system calls can be
interrupted: \cfunc{read}{}, \cfunc{write}{}, \cfunc{sendto}{},
\cfunc{recvfrom}{}, \cfunc{sendmsg}{}, \cfunc{recvmsg}{}, \cfunc{ioctl}{}, and
\cfunc{wait}{}.  See the system documentation for additional information.

\subsubsection{API}
\begin{capi}
\label{thd_new}
\index{thd_new@\cfunc{thd\_new}{}}
\citem{\cfunc[cw\_thd\_t *]{thd\_new}{void *(*a\_start\_func)(void *), void
*a\_arg, cw\_bool\_t a\_suspendible}}
	\begin{capilist}
	\item[Input(s): ]
		\begin{description}\item[]
		\item[a\_start\_func: ] Pointer to a start function.
		\item[a\_arg: ] Argument passed to \cfunc{a\_start\_func}{}.
		\item[a\_suspendible: ]
		\begin{description}\item[]
			\item[FALSE: ] Not suspendible.
			\item[TRUE: ] Suspendible.
		\end{description}
		\end{description}
	\item[Output(s): ]
		\begin{description}\item[]
		\item[retval: ] Pointer to a \classname{thd}.
		\end{description}
	\item[Exception(s): ]
		\begin{description}\item[]
		\item[\htmlref{CW\_ONYXX\_OOM}{CW_ONYXX_OOM}.]
		\end{description}
	\item[Description: ]
		Constructor (creates a new thread).
	\end{capilist}
\label{thd_delete}
\index{thd_delete@\cfunc{thd\_delete}{}}
\citem{\cfunc[void]{thd\_delete}{cw\_thd\_t *a\_thd}}
	\begin{capilist}
	\item[Input(s): ]
		\begin{description}\item[]
		\item[a\_thd: ] Pointer to a \classname{thd}.
		\end{description}
	\item[Output(s): ] None.
	\item[Exception(s): ] None.
	\item[Description: ]
		Destructor.
	\end{capilist}
\label{thd_join}
\index{thd_join@\cfunc{thd\_join}{}}
\citem{\cfunc[void *]{thd\_join}{cw\_thd\_t *a\_thd}}
	\begin{capilist}
	\item[Input(s): ]
		\begin{description}\item[]
		\item[a\_thd: ] Pointer to a \classname{thd}.
		\end{description}
	\item[Output(s): ]
		\begin{description}\item[]
		\item[retval: ]
			Return value from thread entry function.
		\end{description}
	\item[Exception(s): ] None.
	\item[Description: ]
		Join (wait for) the thread associated with \cvar{a\_thd}.
	\end{capilist}
\label{thd_self}
\index{thd_self@\cfunc{thd\_self}{}}
\citem{\cfunc[cw\_thd\_t *]{thd\_self}{void}}
	\begin{capilist}
	\item[Input(s): ] None.
	\item[Output(s): ]
		\begin{description}\item[]
		\item[retval: ]
			Pointer to the calling thread's \classname{thd}
			structure.
		\end{description}
	\item[Exception(s): ] None.
	\item[Description: ]
		Return a pointer to the \classname{thd} structure that
		corresponds to the calling thread.
	\end{capilist}
\label{thd_yield}
\index{thd_yield@\cfunc{thd\_yield}{}}
\citem{\cfunc[void]{thd\_yield}{void}}
	\begin{capilist}
	\item[Input(s): ] None.
	\item[Output(s): ] None.
	\item[Exception(s): ] None.
	\item[Description: ]
		Give up the rest of the calling thread's time slice.
	\end{capilist}
\label{thd_sigmask}
\index{thd_sigmask@\cfunc{thd\_sigmask}{}}
\citem{\cfunc[int]{thd\_sigmask}{int a\_how, const sigset\_t *a\_set, sigset\_t
*r\_oset}}
	\begin{capilist}
	\item[Input(s): ]
		\begin{description}\item[]
		\item[a\_how: ]
			\begin{description}\item[]
			\item[SIG\_BLOCK: ]
				Block signals in \cvar{a\_set}.
			\item[SIG\_UNBLOCK: ]
				Unblock signals in \cvar{a\_set}.
			\item[SIG\_SETMASK: ]
				Set signal mask to \cvar{a\_set}.
			\end{description}
		\item[a\_set: ]
			Pointer to a signal set.
		\item[r\_oset: ]
			\begin{description}\item[]
			\item[non-NULL: ]
				Pointer space to store the old signal mask.
			\item[NULL: ]
				Ignored.
			\end{description}
		\end{description}
	\item[Output(s): ]
		\begin{description}\item[]
		\item[retval: ]
			Always zero, unless the arguments are invalid.
		\item[*r\_oset: ]
			Old signal set.
		\end{description}
	\item[Exception(s): ] None.
	\item[Description: ]
		Set the calling thread's signal mask.
	\end{capilist}
\label{thd_crit_enter}
\index{thd_crit_enter@\cfunc{thd\_crit\_enter}{}}
\citem{\cfunc[void]{thd\_crit\_enter}{void}}
	\begin{capilist}
	\item[Input(s): ] None.
	\item[Output(s): ] None.
	\item[Exception(s): ] None.
	\item[Description: ]
		Enter a critical region where the calling thread may not be
		suspended by \cfunc{thd\_suspend}{}, \cfunc{thd\_trysuspend}{},
		or \cfunc{thd\_single\_enter}{}.
	\end{capilist}
\label{thd_crit_leave}
\index{thd_crit_leave@\cfunc{thd\_crit\_leave}{}}
\citem{\cfunc[void]{thd\_crit\_leave}{void}}
	\begin{capilist}
	\item[Input(s): ] None.
	\item[Output(s): ] None.
	\item[Exception(s): ] None.
	\item[Description: ]
		Leave a critical section; the calling thread may once again be
		suspended.
	\end{capilist}
\label{thd_single_enter}
\index{thd_single_enter@\cfunc{thd\_single\_enter}{}}
\citem{\cfunc[void]{thd\_single\_enter}{void}}
	\begin{capilist}
	\item[Input(s): ] None.
	\item[Output(s): ] None.
	\item[Exception(s): ] None.
	\item[Description: ]
		Enter a critical region where all other suspendible threads must
		be suspended.
	\end{capilist}
\label{thd_single_leave}
\index{thd_single_leave@\cfunc{thd\_single\_leave}{}}
\citem{\cfunc[void]{thd\_single\_leave}{void}}
	\begin{capilist}
	\item[Input(s): ] None.
	\item[Output(s): ] None.
	\item[Exception(s): ] None.
	\item[Description: ]
		Leave a critical section where all other threads must be
		suspended.  All threads that were suspended in
		\cfunc{thd\_single\_enter}{} are resumed.
	\end{capilist}
\label{thd_suspend}
\index{thd_suspend@\cfunc{thd\_suspend}{}}
\citem{\cfunc[void]{thd\_suspend}{cw\_thd\_t *a\_thd}}
	\begin{capilist}
	\item[Input(s): ]
		\begin{description}\item[]
		\item[a\_thd: ]
			Pointer to a \classname{thd}.
		\end{description}
	\item[Output(s): ] None.
	\item[Exception(s): ] None.
	\item[Description: ]
		Suspend \cvar{a\_thd}.
	\end{capilist}
\label{thd_trysuspend}
\index{thd_trysuspend@\cfunc{thd\_trysuspend}{}}
\citem{\cfunc[cw\_bool\_t]{thd\_trysuspend}{cw\_thd\_t *a\_thd}}
	\begin{capilist}
	\item[Input(s): ]
		\begin{description}\item[]
		\item[a\_thd: ]
			Pointer to a \classname{thd}.
		\end{description}
	\item[Output(s): ]
		\begin{description}\item[]
		\item[retval: ]
			\begin{description}\item[]
			\item[FALSE: ]
				Success.
			\item[TRUE: ]
				Failure.
			\end{description}
		\end{description}
	\item[Exception(s): ] None.
	\item[Description: ]
		Try to suspend \cvar{a\_thd}, but fail if it is in a critical
		section.
	\end{capilist}
\label{thd_resume}
\index{thd_resume@\cfunc{thd\_resume}{}}
\citem{\cfunc[void]{thd\_resume}{cw\_thd\_t *a\_thd}}
	\begin{capilist}
	\item[Input(s): ]
		\begin{description}\item[]
		\item[a\_thd: ]
			Pointer to a \classname{thd}.
		\end{description}
	\item[Output(s): ] None.
	\item[Exception(s): ] None.
	\item[Description: ]
		Resume (make runnable) \classname{a\_thd}.
	\end{capilist}
\end{capi}

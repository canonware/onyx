%-*-mode:latex-*-
%%%%%%%%%%%%%%%%%%%%%%%%%%%%%%%%%%%%%%%%%%%%%%%%%%%%%%%%%%%%%%%%%%%%%%%%%%%%%
%
% <Copyright = jasone>
% <License>
%
%%%%%%%%%%%%%%%%%%%%%%%%%%%%%%%%%%%%%%%%%%%%%%%%%%%%%%%%%%%%%%%%%%%%%%%%%%%%%
%
% Version: Onyx <Version = onyx>
%
% errordict reference portion of Onyx Manual.
%
%%%%%%%%%%%%%%%%%%%%%%%%%%%%%%%%%%%%%%%%%%%%%%%%%%%%%%%%%%%%%%%%%%%%%%%%%%%%%

\subsection{errordict}
\label{sec:errordict}
\index{errordict@\onyxop{}{errordict}{}}

Each thread has its own errordict, which is used by default by the error
handling machinery.

\begin{longtable}{\optableformat{4.10}}
\caption{errordict summary} \\
\hline
\optableent
	{Input(s)}
	{Op/Proc/Var}
	{Output(s)}
	{Description}
\hline \hline
%begin{latexonly}
\endfirsthead
\caption[]{\emph{continued}} \\
\hline
\optableent
	{Input(s)}
	{Op/Proc/Var}
	{Output(s)}
	{Description}
\hline \hline \endhead
\multicolumn{2}{r}{\emph{Continued on next page...}} \endfoot
\hline \endlastfoot
%end{latexonly}
\optableent
	{--}
	{{\bf \htmlref{handleerror}{errordict:handleerror}}}
	{--}
	{Print a state dump.}
\hline
\optableent
	{--}
	{{\bf \htmlref{stop}{errordict:stop}}}
	{--}
	{Last operation during error handling.}
\end{longtable}

\begin{description}
\label{handleerror}
\label{errordict:handleerror}
\index{handleerror@\onyxop{}{handleerror}{}}
\item[{\onyxop{--}{handleerror}{--}}: ]
	\begin{description}\item[]
	\item[Input(s): ] None.
	\item[Output(s): ] None.
	\item[Error(s): ]
		Under normal conditions, no errors occur.  However, it is
		possible for the application to corrupt the error handling
		machinery to the point that an error will occur.  If that
		happens, the result is possible infinite recursion, and program
		crashes are a real possibility.
	\item[Description: ]
		Print a dump of the most recent error recorded in the
		currenterror dictionary.
	\item[Example(s): ]\begin{verbatim}

onyx:0> {true {true 1 sprint x y} if} eval
true
Error $undefined
ostack: ()
dstack: (-dict- -dict- -dict- -dict-)
cstack: ()
estack/istack trace (0..5):
0:      x
1: {
        true
        1
        sprint
 3:-->  x
        y
}
2:      --if--
3:      --eval--
4:      -file-
5:      --start--
onyx:1> errordict begin handleerror end
Error $undefined
ostack: ()
dstack: (-dict- -dict- -dict- -dict-)
cstack: ()
estack/istack trace (0..5):
0:      x
1: {
        true
        1
        sprint
 3:-->  x
        y
}
2:      --if--
3:      --eval--
4:      -file-
5:      --start--
onyx:1>
		\end{verbatim}
	\end{description}
\label{errordict:stop}
\index{stop@\onyxop{}{stop}{}}
\item[{\onyxop{--}{stop}{--}}: ]
	\begin{description}\item[]
	\item[Input(s): ] None.
	\item[Output(s): ] None.
	\item[Error(s): ] None.
	\item[Description: ]
		This is called as the very last operation when an error occurs.
		Initially, its value is the same as that for the
		\htmlref{\onyxop{}{stop}{}}{systemdict:stop} operator in
		systemdict.
	\item[Example(s): ]\begin{verbatim}

onyx:0> errordict begin
onyx:0> $stop {`Custom stop\n' print flush quit} def
onyx:0> x
Error $undefined
ostack: ()
dstack: (-dict- -dict- -dict- -dict- -dict-)
cstack: ()
estack/istack trace (0..2):
0:      x
1:      -file-
2:      --start--
Custom stop
		\end{verbatim}
	\end{description}
\end{description}

%-*-mode:latex-*-
%%%%%%%%%%%%%%%%%%%%%%%%%%%%%%%%%%%%%%%%%%%%%%%%%%%%%%%%%%%%%%%%%%%%%%%%%%%%%
%
% <Copyright = jasone>
% <License>
%
%%%%%%%%%%%%%%%%%%%%%%%%%%%%%%%%%%%%%%%%%%%%%%%%%%%%%%%%%%%%%%%%%%%%%%%%%%%%%
%
% Version: Onyx <Version = onyx>
%
% Back matter of Onyx Manual.
%
%%%%%%%%%%%%%%%%%%%%%%%%%%%%%%%%%%%%%%%%%%%%%%%%%%%%%%%%%%%%%%%%%%%%%%%%%%%%%

\clearemptydoublepage
\backmatter
\chapter{LICENSES}
%
% Don't print the chapter number in the header.
%
%begin{latexonly}
\rhead[\bfseries{LICENSE}]{\bfseries\thepage}
%end{latexonly}

Onyx is licensed according to the following terms:

\begin{verbatim}
<Copyright = jasone>
<Copyright = mordor>
<License>
\end{verbatim}

Onyx's regular expression support is provided by the PCRE library package, which
is open source software, written by Philip Hazel, and copyright by the
University of Cambridge, England.  The official PCRE distribution site is
\htmlurl{ftp://ftp.csx.cam.ac.uk/pub/software/programming/pcre/}.  Inclusion of
the previous two sentences in this documentation meets the requirements of the
PCRE license, both for source and binary distributions of Onyx.  See the PCRE
source distribution for additional details.

The \modulename{modclopt} module optionally uses the libedit library.  Following
is the license text.  Note that the Regents of the University of California have
approved the retroactive removal of clause 3 from the license, which means that
the advertising clause no longer applies to libedit.

\begin{verbatim}
Copyright (c) 1992, 1993
     The Regents of the University of California.  All rights reserved.

This code is derived from software contributed to Berkeley by
Christos Zoulas of Cornell University.

Redistribution and use in source and binary forms, with or without
modification, are permitted provided that the following conditions
are met:
1. Redistributions of source code must retain the above copyright
   notice, this list of conditions and the following disclaimer.
2. Redistributions in binary form must reproduce the above copyright
   notice, this list of conditions and the following disclaimer in the
   documentation and/or other materials provided with the distribution.
3. All advertising materials mentioning features or use of this software
   must display the following acknowledgement:
     This product includes software developed by the University of
     California, Berkeley and its contributors.
4. Neither the name of the University nor the names of its contributors
   may be used to endorse or promote products derived from this software
   without specific prior written permission.

THIS SOFTWARE IS PROVIDED BY THE REGENTS AND CONTRIBUTORS ``AS IS'' AND
ANY EXPRESS OR IMPLIED WARRANTIES, INCLUDING, BUT NOT LIMITED TO, THE
IMPLIED WARRANTIES OF MERCHANTABILITY AND FITNESS FOR A PARTICULAR PURPOSE
ARE DISCLAIMED.  IN NO EVENT SHALL THE REGENTS OR CONTRIBUTORS BE LIABLE
FOR ANY DIRECT, INDIRECT, INCIDENTAL, SPECIAL, EXEMPLARY, OR CONSEQUENTIAL
DAMAGES (INCLUDING, BUT NOT LIMITED TO, PROCUREMENT OF SUBSTITUTE GOODS
OR SERVICES; LOSS OF USE, DATA, OR PROFITS; OR BUSINESS INTERRUPTION)
HOWEVER CAUSED AND ON ANY THEORY OF LIABILITY, WHETHER IN CONTRACT, STRICT
LIABILITY, OR TORT (INCLUDING NEGLIGENCE OR OTHERWISE) ARISING IN ANY WAY
OUT OF THE USE OF THIS SOFTWARE, EVEN IF ADVISED OF THE POSSIBILITY OF
SUCH DAMAGE.
\end{verbatim}

\clearemptydoublepage
%
% Don't print the chapter number in the header for the index.
%
%begin{latexonly}
\rhead[\bfseries{INDEX}]{\bfseries\thepage}
%end{latexonly}

\addcontentsline{toc}{chapter}{Index}
\begin{theindex}
% latex2html apparently generates and inserts its own index, so don't actually
% include the index for the html version.
%begin{latexonly}
\printindex
%end{latexonly}
\end{theindex}

\end{document}

%-*-mode:latex-*-
%%%%%%%%%%%%%%%%%%%%%%%%%%%%%%%%%%%%%%%%%%%%%%%%%%%%%%%%%%%%%%%%%%%%%%%%%%%%%
%
% <Copyright = jasone>
% <License>
%
%%%%%%%%%%%%%%%%%%%%%%%%%%%%%%%%%%%%%%%%%%%%%%%%%%%%%%%%%%%%%%%%%%%%%%%%%%%%%
%
% Version: Onyx <Version = onyx>
%
% mb portion of Onyx Manual.
%
%%%%%%%%%%%%%%%%%%%%%%%%%%%%%%%%%%%%%%%%%%%%%%%%%%%%%%%%%%%%%%%%%%%%%%%%%%%%%

\subsection{mb}
\label{mb}
\index{mb@\classname{mb}{}}

The \classname{mb} class implements memory barriers.  A memory barrier is a low
level construct that is sometimes useful for guaranteeing the order in which
memory operations take place, even when multiple microprocessors are involved.
In most cases, mutexes are the best choice for synchronizing data access, but
sometimes it is convenient (and critical to performance) to use memory barriers
where weaker access constraints are adequate.

\subsubsection{API}
\begin{capi}
\label{mb_write}
\index{mb_write@\cfunc{mb\_write}{}}
\citem{\cfunc[void]{mb\_write}{void}}
	\begin{capilist}
	\item[Input(s): ] None.
	\item[Output(s): ] None.
	\item[Exception(s): ] None.
	\item[Description: ]
		Create a write barrier, so that any memory writes done before
		the memory barrier are guaranteed to be visible by the time any
		memory writes after the memory barrier become visible.
	\end{capilist}
\end{capi}

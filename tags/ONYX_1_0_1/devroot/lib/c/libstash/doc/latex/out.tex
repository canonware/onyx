%-*-mode:latex-*-
%%%%%%%%%%%%%%%%%%%%%%%%%%%%%%%%%%%%%%%%%%%%%%%%%%%%%%%%%%%%%%%%%%%%%%%%%%%%%
%
% <Copyright = jasone>
% <License>
%
%%%%%%%%%%%%%%%%%%%%%%%%%%%%%%%%%%%%%%%%%%%%%%%%%%%%%%%%%%%%%%%%%%%%%%%%%%%%%
%
% Version: <Version>
%
% out portion of Onyx Manual.
%              
%%%%%%%%%%%%%%%%%%%%%%%%%%%%%%%%%%%%%%%%%%%%%%%%%%%%%%%%%%%%%%%%%%%%%%%%%%%%%

\subsection{out}
\label{out}
\index{out@\classname{out}{}}

The \classname{out} class provides functionality similar to the printf family of
functions, with the additional capability of dynamically adding handlers for
non-builtin types.

The syntax of formatting specifiers is significantly different than what printf
uses, since arbitrary flags must be supported.  Formatting specifiers are
delimited by `{\lb}' and `{\rb}'.  Following are some examples of formatting
strings, followed by explanations.

\begin{description}
\item[{\tt "{\lb}{\lb}"}: ]
	Print a `{\lb}' character.  Due to the way C strings and `{\bs}'
	protection work, it is necessary to use a different escape character
	(`{\lb}').
\item[{\tt "[s{\rb}"}: ]
	Print a string.
\item[{\tt "{\lb}s{\pipe}w:10{\rb}"}: ]
	Print a string, padded to be at least 10 bytes long.
\item[{\tt "{\lb}i{\pipe}b:16{\pipe}p:0{\pipe}w:8{\rb}"}: ]
	Print a 32 bit integer in base 16.  Pad the output to 8 bytes, using `0'
	for the padding character.
\end{description}

As can be seen above, flags are specified as name/value pairs.  Each name and
value is separated by `:', and name/value pairs are separated from each other
(and the type specifier) by `{\pipe}'.  Names and values can be of arbitrary
length (nonzero for names), and can contain any characters except `{\rb}',
`{\pipe}', `:', and ` '.  The parser may let these characters slip through in
some cases, but such behavior should not be relied on.  Some or all of the
following flags are supported for the builtin types:

\begin{description}
\item[{\tt "w"}: ]
	Minimum number of bytes of output.
\item[{\tt "j"}: ]
	Justification.  Legal values:
	\begin{description}
	\item[{\tt "r"}: ]
		Right.
	\item[{\tt "l"}: ]
		Left.
	\item[{\tt "c"}: ]
		Center.
	\end{description}
\item[{\tt "p"}: ]
	 Padding character.
\item[{\tt "b"}: ]
	Numerical base.  Legal bases are (2 {\lt}= base {\lt}= 36).
\item[{\tt "s"}: ]
	Signed/unsigned.  Legal values:
	\begin{description}
	\item[{\tt "u"}: ]
		 Unsigned.
	\item[{\tt "s"}: ]
		Signed.
	\end{description}
\item[{\tt "+"}: ]
	Show sign.  Legal values:
	\begin{description}
	\item[{\tt "-"}: ]
		Only print sign if output is negative.
	\item[{\tt "+"}: ]
		Always print sign.
	\end{description}
\end{description}

The following is a matrix of flag defaults for the builtin output types.  A
blank cell indicates an unsupported type/flag combination:
\begin{table}[htb]
\begin{center}
\begin{tabular}{|l|l||l|l|l|l|l|l|}
\hline
\multicolumn{2}{|c||}{Type} & \multicolumn{6}{c|}{Flag} \\
\hline
C type		& Symbol & w	& j	& p	& b	& s	& +	\\
\hline \hline
cw\_*int32\_t	& i	& Fit	& r	& ` '	& 10	& u	& --	\\
\hline
cw\_*int64\_t	& q	& Fit	& r	& ` '	& 10	& u	& --	\\
\hline
cw\_*int8\_t	& c	& Fit	& r	& ` '	&	&	&	\\
\hline
cw\_*int8\_t *	& s	& Fit	& r	& ` '	&	&	&	\\
\hline
void *		& p	& Fit	& r	& ` '	& 16	& u	& --	\\
\hline
cw\_buf\_t *	& b	& Fit	& r	& ` '	&	&	&	\\
\hline
\end{tabular}
\end{center}
\caption{Flag defaults for builtin output types}
\end{table}

Note that \classname{buf} printing is not provided if \libname{libstash} is
compiled without the buf class.  Whether \classname{buf} printing is supported
can be determined via the
\htmlref{\cppdef{\_CW\_HAVE\_LIBSTASH\_BUF}}{_CW_HAVE_LIBSTASH_BUF} cpp macro.

\subsubsection{API}
\begin{capi}
\label{cw_out_render_t}
\index{cw_out_render_t@\cfunc{cw\_out\_render\_t}{}}
\citem{\cfunc[typedef cw\_uint32\_t]{out\_render\_t}{const char *a\_format,
cw\_uint32\_t a\_format\_len, const void *a\_arg, cw\_uint32\_t a\_max\_len,
cw\_uint8\_t *r\_str}}
	\begin{capilist}
	\item[Input(s): ]
		\begin{description}\item[]
		\item[a\_format: ]
			Pointer to a format specifier.
		\item[a\_format\_len: ]
			Length in bytes of \cvar{a\_format}.
		\item[a\_arg: ]
			Pointer to object to be output.
		\item[a\_max\_len: ]
			Maximum number of bytes to output.
		\item[r\_str: ]
			Pointer to space to render to.
		\end{description}
	\item[Output(s): ]
		\begin{description}\item[]
		\item[retval: ]
			Number of bytes of output spoce that the rendering
			function needs.
		\item[*r\_str: ]
			Rendered version of \cvar{a\_arg}.
		\end{description}
	\item[Exception(s): ]
		\begin{description}\item[]
		\item[\htmlref{\_CW\_STASHX\_OOM}{_CW_STASHX_OOM}.]
		\end{description}
	\item[Description: ]
		Print \cvar{a\_arg} to \cvar{*r\_str}, given the specification
		pointed to by \cvar{a\_format}.
	\end{capilist}
\label{out_new}
\index{out_new@\cfunc{out\_new}{}}
\citem{\cfunc[cw\_out\_t *]{out\_new}{cw\_out\_t *a\_out, cw\_mem\_t
*a\_mem}}
	\begin{capilist}
	\item[Input(s): ]
		\begin{description}\item[]
		\item[a\_out: ]
			Pointer to space for an \classname{out}, or NULL.
		\item[a\_mem: ]
			Pointer to the allocator to use internally.
		\end{description}
	\item[Output(s): ]
		\begin{description}\item[]
		\item[retval: ]
			Pointer to an \classname{out}.
		\end{description}
	\item[Exception(s): ]
		\begin{description}\item[]
		\item[\htmlref{\_CW\_STASHX\_OOM}{_CW_STASHX_OOM}.]
		\end{description}
	\item[Description: ]
		Constructor.
	\end{capilist}
\label{out_delete}
\index{out_delete@\cfunc{out\_delete}{}}
\citem{\cfunc[void]{out\_delete}{cw\_out\_t *a\_out}}
	\begin{capilist}
	\item[Input(s): ]
		\begin{description}\item[]
		\item[a\_out: ]
			Pointer to an \classname{out}.
		\end{description}
	\item[Output(s): ] None.
	\item[Exception(s): ] None.
	\item[Description: ]
		Destructor.
	\end{capilist}
\label{out_register}
\index{out_register@\cfunc{out\_register}{}}
\citem{\cfunc[cw\_bool\_t]{out\_register}{cw\_out\_t *a\_out, const char
*a\_type, cw\_uint32\_t a\_size, cw\_out\_render\_t *a\_render\_func}}
	\begin{capilist}
	\item[Input(s): ]
		\begin{description}\item[]
		\item[a\_out: ]
			Pointer to an \classname{out}.
		\item[a\_type: ]
			Pointer to a NULL-terminated string that represents a
			data type specifier.  The string length (not including
			the NULL terminator) must be
			\cppdef{\_CW\_OUT\_MAX\_TYPE} bytes or less.
		\item[a\_size: ]
			Size of the argument that will be passed to
			\cfunc{*out\_put*}{}.  In almost all cases this will be
			\cfunc{sizeof}{void *}.  \cvar{a\_size} must be 1, 2, 4,
			or 8.
		\item[a\_render\_func: ]
			Pointer to a rendering function.
		\end{description}
	\item[Output(s): ] None.
	\item[Exception(s): ]
		\begin{description}\item[]
		\item[\htmlref{\_CW\_STASHX\_OOM}{_CW_STASHX_OOM}.]
		\end{description}
	\item[Description: ]
		Register a new type with \cvar{a\_out}, so that non-builtin type
		specifiers can be embedded in formatting strings.
	\end{capilist}
\label{out_merge}
\index{out_merge@\cfunc{out\_merge}{}}
\citem{\cfunc[void]{out\_merge}{cw\_out\_t *a\_a, cw\_out\_t *a\_b}}
	\begin{capilist}
	\item[Input(s): ]
		\begin{description}\item[]
		\item[a\_a: ]
			Pointer to an \classname{out}.
		\item[a\_b: ]
			Pointer to an \classname{out}.
		\end{description}
	\item[Output(s): ] None.
	\item[Exception(s): ]
		\begin{description}\item[]
		\item[\htmlref{\_CW\_STASHX\_OOM}{_CW_STASHX_OOM}.]
		\end{description}
	\item[Description: ]
		Merge the extended type handlers for \cvar{a\_b} into
		\cvar{a\_a}.
	\end{capilist}
\label{out_default_fd_get}
\index{out_default_fd_get@\cfunc{out\_default\_fd\_get}{}}
\citem{\cfunc[cw\_sint32\_t]{out\_default\_fd\_get}{cw\_out\_t *a\_out}}
	\begin{capilist}
	\item[Input(s): ]
		\begin{description}\item[]
		\item[a\_out: ]
			Pointer to an \classname{out}.
		\end{description}
	\item[Output(s): ]
		\begin{description}\item[]
		\item[retval: ]
			File descriptor number.
		\end{description}
	\item[Exception(s): ] None.
	\item[Description: ]
		Return the default file descriptor for \cvar{a\_out}.
	\end{capilist}
\label{out_default_fd_set}
\index{out_default_fd_set@\cfunc{out\_default\_fd\_set}{}}
\citem{\cfunc[void]{out\_default\_fd\_set}{cw\_out\_t *a\_out, cw\_sint32\_t
a\_fd}}
	\begin{capilist}
	\item[Input(s): ]
		\begin{description}\item[]
		\item[a\_out: ]
			Pointer to an \classname{out}.
		\item[a\_fd: ]
			File descriptor number.
		\end{description}
	\item[Output(s): ] None.
	\item[Exception(s): ] None.
	\item[Description: ]
		Set the default file descriptor number for \cvar{a\_out}.
	\end{capilist}
\label{out_put}
\index{out_put@\cfunc{out\_put}{}}
\citem{\cfunc[cw\_sint32\_t]{out\_put}{cw\_out\_t *a\_out, const char
*a\_format, ...}}
\label{_cw_out_put}
\index{_cw_out_put@\cppmacro{\_cw\_out\_put}{}}
\citem{\cppmacro[cw\_sint32\_t]{\_cw\_out\_put}{const char *a\_format, ...}}
	\begin{capilist}
	\item[Input(s): ]
		\begin{description}\item[]
		\item[a\_out: ]
			Pointer to an \classname{out}.
			\cppmacro{\_cw\_out\_put}{} implicitly uses
			\cvar{out\_std}.
		\item[a\_format: ]
			Pointer to a formatting specifier string.
		\item[...: ]
			Arguments that correspond to the specifiers in
			\cvar{a\_format}.
		\end{description}
	\item[Output(s): ]
		\begin{description}\item[]
		\item[retval: ]
			\begin{description}\item[]
			\item[{\gt}= 0: ]
				Number of bytes output.
			\item[-1: ]
				\cfunc{write}{} error.
			\end{description}
		\item{Output printed to the default file descriptor for
			\cvar{a\_out}.}
		\end{description}
	\item[Exception(s): ]
		\begin{description}\item[]
		\item[\htmlref{\_CW\_STASHX\_OOM}{_CW_STASHX_OOM}.]
		\item[\htmlref{\_CW\_STASHX\_OUT\_PARSE}{_CW_STASHX_OUT_PARSE}.]
		\end{description}
	\item[Description: ]
		Print formatted output.
	\end{capilist}
\label{out_put_e}
\index{out_put_e@\cfunc{out\_put\_e}{}}
\citem{\cfunc[cw\_sint32\_t]{out\_put\_e}{cw\_out\_t *a\_out, const char
*a\_file\_name, cw\_uint32\_t a\_line\_num, const char *a\_func\_name, const
char *a\_format, ...}}
\label{_cw_out_put_e}
\index{_cw_out_put_e@\cppmacro{\_cw\_out\_put\_e}{}}
\citem{\cppmacro[cw\_sint32\_t]{\_cw\_out\_put\_e}{const char *a\_file\_name,
cw\_uint32\_t a\_line\_num, const char *a\_func\_name, const char *a\_format,
...}}
	\begin{capilist}
	\item[Input(s): ]
		\begin{description}\item[]
		\item[a\_out: ]
			Pointer to an \classname{out}.
			\cppmacro{\_cw\_out\_put\_e}{} implicitly uses
			\cvar{out\_err}.
		\item[a\_file\_name: ]
			Pointer to a string that represents the source file
			name, or NULL.
		\item[a\_line\_num: ]
			Source file line number.  Ignored if
			\cvar{a\_file\_name} is NULL.
		\item[a\_func\_name: ]
			Pointer to a string that represents the sourc function
			name, or NULL.
		\item[a\_format: ]
			Pointer to a formatting specifier string.
		\item[...: ]
			Arguments that correspond to the specifiers in
			\cvar{a\_format}.
		\end{description}
	\item[Output(s): ]
		\item[retval: ]
			\begin{description}\item[]
			\item[{\gt}= 0: ]
				Number of bytes output.
			\item[-1: ]
				\cfunc{write}{} error.
			\end{description}
		\item{Output printed to the default file descriptor for
			\cvar{a\_out}.}
	\item[Exception(s): ]
		\begin{description}\item[]
		\item[\htmlref{\_CW\_STASHX\_OOM}{_CW_STASHX_OOM}.]
		\item[\htmlref{\_CW\_STASHX\_OUT\_PARSE}{_CW_STASHX_OUT_PARSE}.]
		\end{description}
	\item[Description: ]
		Print formatted output, with optional {\tt "At {\lt}file{\gt},
		line {\lt}line{\gt}: {\lt}function{\gt}(): "} prepended to the
		output.
	\end{capilist}
\label{out_put_n}
\index{out_put_n@\cfunc{out\_put\_n}{}}
\citem{\cfunc[cw\_sint32\_t]{out\_put\_n}{cw\_out\_t *a\_out, cw\_uint32\_t
a\_size, const char *a\_format, ...}}
\label{_cw_out_put_n}
\index{_cw_out_put_n@\cppmacro{\_cw\_out\_put\_n}{}}
\citem{\cppmacro[cw\_sint32\_t]{\_cw\_out\_put\_n}{cw\_uint32\_t a\_size, const
char *a\_format, ...}}
	\begin{capilist}
	\item[Input(s): ]
		\begin{description}\item[]
		\item[a\_out: ]
			Pointer to an \classname{out}.
			\cppmacro{\_cw\_out\_put\_n}{} implicitly uses
			\cvar{out\_std}.
		\item[a\_size: ]
			Maximum number of characters to output.
		\item[a\_format: ]
			Pointer to a formatting specifier string.
		\item[...: ]
			Arguments that correspond to the specifiers in
			\cvar{a\_format}.
		\end{description}
	\item[Output(s): ]
		\begin{description}\item[]
		\item[retval: ]
			\begin{description}\item[]
			\item[{\gt}= 0: ]
				Number of bytes output.
			\item[-1: ]
				\cfunc{write}{} error.
			\end{description}
		\item{Output printed to the default file descriptor for
			\cvar{a\_out}.}
		\end{description}
	\item[Exception(s): ]
		\begin{description}\item[]
		\item[\htmlref{\_CW\_STASHX\_OOM}{_CW_STASHX_OOM}.]
		\item[\htmlref{\_CW\_STASHX\_OUT\_PARSE}{_CW_STASHX_OUT_PARSE}.]
		\end{description}
	\item[Description: ]
		Print at most \cvar{a\_size} bytes of formatted output.
	\end{capilist}
\label{out_put_f}
\index{out_put_f@\cfunc{out\_put\_f}{}}
\citem{\cfunc[cw\_sint32\_t]{out\_put\_f}{cw\_out\_t *a\_out, cw\_sint32\_t
a\_fd, const char *a\_format, ...}}
\label{_cw_out_put_f}
\index{_cw_out_put_f@\cppmacro{\_cw\_out\_put\_f}{}}
\citem{\cppmacro[cw\_sint32\_t]{\_cw\_out\_put\_f}{cw\_sint32\_t a\_fd, const
char *a\_format, ...}}
	\begin{capilist}
	\item[Input(s): ]
		\begin{description}\item[]
		\item[a\_out: ]
			Pointer to an \classname{out}.
		\item[a\_fd: ]
			File descriptor number.
		\item[a\_format: ]
			Pointer to a formatting specifier string.
		\item[...: ]
			Arguments that correspond to the specifiers in
			\cvar{a\_format}.
		\end{description}
	\item[Output(s): ]
		\begin{description}\item[]
		\item[retval: ]
			\begin{description}\item[]
			\item[{\gt}= 0: ]
				Number of bytes output.
			\item[-1: ]
				\cfunc{write}{} error.
			\end{description}
		\item{Output printed to \cvar{a\_fd}.}
		\end{description}
	\item[Exception(s): ]
		\begin{description}\item[]
		\item[\htmlref{\_CW\_STASHX\_OOM}{_CW_STASHX_OOM}.]
		\item[\htmlref{\_CW\_STASHX\_OUT\_PARSE}{_CW_STASHX_OUT_PARSE}.]
		\end{description}
	\item[Description: ]
		Print formatted output.
	\end{capilist}
\label{out_put_fn}
\index{out_put_fn@\cfunc{out\_put\_fn}{}}
\citem{\cfunc[cw\_sint32\_t]{out\_put\_fn}{cw\_out\_t *a\_out, cw\_sint32\_t
a\_fd, cw\_uint32\_t a\_size, const char *a\_format, ...}}
\label{_cw_out_put_fn}
\index{_cw_out_put_fn@\cppmacro{\_cw\_out\_put\_fn}{}}
\citem{\cppmacro[cw\_sint32\_t]{\_cw\_out\_put\_fn}{cw\_sint32\_t a\_fd,
cw\_uint32\_t a\_size, const char *a\_format, ...}}
	\begin{capilist}
	\item[Input(s): ]
		\begin{description}\item[]
		\item[a\_out: ]
			Pointer to an \classname{out}.
		\item[a\_fd: ]
			File descriptor number.
		\item[a\_size: ]
			Maximum number of characters to output.
		\item[a\_format: ]
			Pointer to a formatting specifier string.
		\item[...: ]
			Arguments that correspond to the specifiers in
			\cvar{a\_format}.
		\end{description}
	\item[Output(s): ]
		\begin{description}\item[]
		\item[retval: ]
			\begin{description}\item[]
			\item[{\gt}= 0: ]
				Number of bytes output.
			\item[-1: ]
				\cfunc{write}{} error.
			\end{description}
		\item{Output printed to \cvar{a\_fd}.}
		\end{description}
	\item[Exception(s): ]
		\begin{description}\item[]
		\item[\htmlref{\_CW\_STASHX\_OOM}{_CW_STASHX_OOM}.]
		\item[\htmlref{\_CW\_STASHX\_OUT\_PARSE}{_CW_STASHX_OUT_PARSE}.]
		\end{description}
	\item[Description: ]
		Print at most \cvar{a\_size} bytes of formatted output.
	\end{capilist}
\label{out_put_fv}
\index{out_put_fv@\cfunc{out\_put\_fv}{}}
\citem{\cfunc[cw\_sint32\_t]{out\_put\_fv}{cw\_out\_t *a\_out, cw\_sint32\_t
a\_fd, const char *a\_format, va\_list a\_p}}
	\begin{capilist}
	\item[Input(s): ]
		\begin{description}\item[]
		\item[a\_out: ]
			Pointer to an \classname{out}.
		\item[a\_fd: ]
			File descriptor number.
		\item[a\_format: ]
			Pointer to a formatting specifier string.
		\item[a\_p: ]
			Variable argument list.
		\end{description}
	\item[Output(s): ]
		\begin{description}\item[]
		\item[retval: ]
			\begin{description}\item[]
			\item[{\gt}= 0: ]
				Number of bytes output.
			\item[-1: ]
				\cfunc{write}{} error.
			\end{description}
		\item{Output printed to \cvar{a\_fd}.}
		\end{description}
	\item[Exception(s): ]
		\begin{description}\item[]
		\item[\htmlref{\_CW\_STASHX\_OOM}{_CW_STASHX_OOM}.]
		\item[\htmlref{\_CW\_STASHX\_OUT\_PARSE}{_CW_STASHX_OUT_PARSE}.]
		\end{description}
	\item[Description: ]
		Print formatted output.
	\end{capilist}
\label{out_put_fvn}
\index{out_put_fvn@\cfunc{out\_put\_fvn}{}}
\citem{\cfunc[cw\_sint32\_t]{out\_put\_fvn}{cw\_out\_t *a\_out, cw\_sint32\_t
a\_fd, cw\_uint32\_t a\_size, const char *a\_format, va\_list a\_p}}
	\begin{capilist}
	\item[Input(s): ]
		\begin{description}\item[]
		\item[a\_out: ]
			Pointer to an \classname{out}.
		\item[a\_fd: ]
			File descriptor number.
		\item[a\_size: ]
			Maximum number of characters to output.
		\item[a\_format: ]
			Pointer to a formatting specifier string.
		\item[a\_p: ]
			Variable argument list.
		\end{description}
	\item[Output(s): ]
		\begin{description}\item[]
		\item[retval: ]
			\begin{description}\item[]
			\item[{\gt}= 0: ]
				Number of bytes output.
			\item[-1: ]
				\cfunc{write}{} error.
			\end{description}
		\item{Output printed to \cvar{a\_fd}.}
		\end{description}
	\item[Exception(s): ]
		\begin{description}\item[]
		\item[\htmlref{\_CW\_STASHX\_OOM}{_CW_STASHX_OOM}.]
		\item[\htmlref{\_CW\_STASHX\_OUT\_PARSE}{_CW_STASHX_OUT_PARSE}.]
		\end{description}
	\item[Description: ]
		Print at most \cvar{a\_size} bytes of formatted output.
	\end{capilist}
\label{out_put_s}
\index{out_put_s@\cfunc{out\_put\_s}{}}
\citem{\cfunc[cw\_sint32\_t]{out\_put\_s}{cw\_out\_t *a\_out, char *a\_str,
const char *a\_format, ...}}
\label{_cw_out_put_s}
\index{_cw_out_put_s@\cppmacro{\_cw\_out\_put\_s}{}}
\citem{\cppmacro[cw\_sint32\_t]{\_cw\_out\_put\_s}{char *a\_str, const char
*a\_format, ...}}
	\begin{capilist}
	\item[Input(s): ]
		\begin{description}\item[]
		\item[a\_out: ]
			Pointer to an \classname{out}.
			\cppmacro{\_cw\_out\_put\_s}{} implicitly uses NULL.
		\item[a\_str: ]
			Pointer to an output string.
		\item[a\_format: ]
			Pointer to a formatting specifier string.
		\item[...: ]
			Arguments that correspond to the specifiers in
			\cvar{a\_format}.
		\end{description}
	\item[Output(s): ]
		\begin{description}\item[]
		\item[retval: ]
			Number of bytes output, not including the terminating
			'{\bs}0'.
		\item[*a\_str: ]
			Formatted output.
		\end{description}
	\item[Exception(s): ]
		\begin{description}\item[]
		\item[\htmlref{\_CW\_STASHX\_OOM}{_CW_STASHX_OOM}.]
		\item[\htmlref{\_CW\_STASHX\_OUT\_PARSE}{_CW_STASHX_OUT_PARSE}.]
		\end{description}
	\item[Description: ]
		Print formatted output.
	\end{capilist}
\label{out_put_sa}
\index{out_put_sa@\cfunc{out\_put\_sa}{}}
\citem{\cfunc[cw\_sint32\_t]{out\_put\_sa}{cw\_out\_t *a\_out, char **r\_str,
const char *a\_format, ...}}
	\begin{capilist}
	\item[Input(s): ]
		\begin{description}\item[]
		\item[a\_out: ]
			Pointer to an \classname{out}.
			\cppmacro{\_cw\_out\_put\_s}{} implicitly uses NULL.
		\item[r\_str: ]
			Pointer to a string pointer.
		\item[a\_format: ]
			Pointer to a formatting specifier string.
		\item[...: ]
			Arguments that correspond to the specifiers in
			\cvar{a\_format}.
		\end{description}
	\item[Output(s): ]
		\begin{description}\item[]
		\item[retval: ]
			Number of bytes output, not including the terminating
			'{\bs}0'.
		\item[**r\_str: ]
			Formatted output.
		\end{description}
	\item[Exception(s): ]
		\begin{description}\item[]
		\item[\htmlref{\_CW\_STASHX\_OOM}{_CW_STASHX_OOM}.]
		\item[\htmlref{\_CW\_STASHX\_OUT\_PARSE}{_CW_STASHX_OUT_PARSE}.]
		\end{description}
	\item[Description: ]
		Allocate space, then print formatted output to it.
	\end{capilist}
\label{out_put_sn}
\index{out_put_sn@\cfunc{out\_put\_sn}{}}
\citem{\cfunc[cw\_sint32\_t]{out\_put\_sn}{cw\_out\_t *a\_out, char *a\_str,
cw\_uint32\_t a\_size, const char *a\_format, ...}}
\label{_cw_out_put_sn}
\index{_cw_out_put_sn@\cppmacro{\_cw\_out\_put\_sn}{}}
\citem{\cppmacro[cw\_sint32\_t]{\_cw\_out\_put\_sn}{char *a\_str, cw\_uint32\_t
a\_size, const char *a\_format, ...}}
	\begin{capilist}
	\item[Input(s): ]
		\begin{description}\item[]
		\item[a\_out: ]
			Pointer to an \classname{out}.
			\cppmacro{\_cw\_out\_put\_sn}{} implicitly uses NULL.
		\item[a\_str: ]
			Pointer to an output string.
		\item[a\_size: ]
			Maximum number of characters to output.  If
			\cvar{a\_size} is less than or equal to the number of
			bytes that could be output, were there no limit, then no
			'{\bs}0' string termination byte is written.  Therefore,
			if the caller is depending on such string termination,
			care should be taken to specify \cvar{a\_size} as one
			less than the total string length, and to make sure that
			if the string is completely filled with output, that the
			last byte is set to '{\bs}0'.
		\item[a\_format: ]
			Pointer to a formatting specifier string.
		\item[...: ]
			Arguments that correspond to the specifiers in
			\cvar{a\_format}.
		\end{description}
	\item[Output(s): ]
		\begin{description}\item[]
		\item[retval: ]
			Number of bytes output, not including the terminating
			'{\bs}0'.
		\item[*a\_str: ]
			Formatted output.
		\end{description}
	\item[Exception(s): ]
		\begin{description}\item[]
		\item[\htmlref{\_CW\_STASHX\_OOM}{_CW_STASHX_OOM}.]
		\item[\htmlref{\_CW\_STASHX\_OUT\_PARSE}{_CW_STASHX_OUT_PARSE}.]
		\end{description}
	\item[Description: ]
		Print at most \cvar{a\_size} bytes of formatted output.
	\end{capilist}
\label{out_put_sv}
\index{out_put_sv@\cfunc{out\_put\_sv}{}}
\citem{\cfunc[cw\_sint32\_t]{out\_put\_sv}{cw\_out\_t *a\_out, char *a\_str,
const char *a\_format, va\_list a\_p}}
	\begin{capilist}
	\item[Input(s): ]
		\begin{description}\item[]
		\item[a\_out: ]
			Pointer to an \classname{out}.
		\item[a\_str: ]
			Pointer to an output string.
		\item[a\_format: ]
			Pointer to a formatting specifier string.
		\item[a\_p: ]
			Variable argument list.
		\end{description}
	\item[Output(s): ]
		\begin{description}\item[]
		\item[retval: ]
			Number of bytes output, not including the terminating
			'{\bs}0'.
		\item[*a\_str: ]
			Formatted output.
		\end{description}
	\item[Exception(s): ]
		\begin{description}\item[]
		\item[\htmlref{\_CW\_STASHX\_OOM}{_CW_STASHX_OOM}.]
		\item[\htmlref{\_CW\_STASHX\_OUT\_PARSE}{_CW_STASHX_OUT_PARSE}.]
		\end{description}
	\item[Description: ]
		Print formatted output.
	\end{capilist}
\label{out_put_sva}
\index{out_put_sva@\cfunc{out\_put\_sva}{}}
\citem{\cfunc[cw\_sint32\_t]{out\_put\_sva}{cw\_out\_t *a\_out, char **r\_str,
const char *a\_format, va\_list a\_p}}
	\begin{capilist}
	\item[Input(s): ]
		\begin{description}\item[]
		\item[a\_out: ]
			Pointer to an \classname{out}.
		\item[r\_str: ]
			Pointer to an output string pointer.
		\item[a\_format: ]
			Pointer to a formatting specifier string.
		\item[a\_p: ]
			Variable argument list.
		\end{description}
	\item[Output(s): ]
		\begin{description}\item[]
		\item[retval: ]
			Number of bytes output, not including the terminating
			'{\bs}0'.
		\item[**r\_str: ]
			Formatted output.
		\end{description}
	\item[Exception(s): ]
		\begin{description}\item[]
		\item[\htmlref{\_CW\_STASHX\_OOM}{_CW_STASHX_OOM}.]
		\item[\htmlref{\_CW\_STASHX\_OUT\_PARSE}{_CW_STASHX_OUT_PARSE}.]
		\end{description}
	\item[Description: ]
		Allocate space, then print formatted output to it.
	\end{capilist}
\label{out_put_svn}
\index{out_put_svn@\cfunc{out\_put\_svn}{}}
\citem{\cfunc[cw\_sint32\_t]{out\_put\_svn}{cw\_out\_t *a\_out, char *a\_str,
cw\_uint32\_t a\_size, const char *a\_format, va\_list a\_p}}
	\begin{capilist}
	\item[Input(s): ]
		\begin{description}\item[]
		\item[a\_out: ]
			Pointer to an \classname{out}.
		\item[a\_str: ]
			Pointer to an output string.
		\item[a\_size: ]
			Maximum number of characters to output.  If
			\cvar{a\_size} is less than or equal to the number of
			bytes that could be output, were there no limit, then no
			'{\bs}0' string termination byte is written.  Therefore,
			if the caller is depending on such string termination,
			care should be taken to specify \cvar{a\_size} as one
			less than the total string length, and to make sure that
			if the string is completely filled with output, that the
			last byte is set to '{\bs}0'.
		\item[a\_format: ]
			Pointer to a formatting specifier string.
		\item[a\_p: ]
			Variable argument list.
		\end{description}
	\item[Output(s): ]
		\begin{description}\item[]
		\item[retval: ]
			\begin{description}\item[]
			\item[{\gt}= 0: ]
				Number of bytes output.
			\item[-1: ]
				\cfunc{write}{} error.
			\end{description}
		\item[*a\_str: ]
			Formatted output.
		\end{description}
	\item[Exception(s): ]
		\begin{description}\item[]
		\item[\htmlref{\_CW\_STASHX\_OOM}{_CW_STASHX_OOM}.]
		\item[\htmlref{\_CW\_STASHX\_OUT\_PARSE}{_CW_STASHX_OUT_PARSE}.]
		\end{description}
	\item[Description: ]
		Print at most \cvar{a\_size} bytes of formatted output.
	\end{capilist}
\label{spec_type_get}
\index{spec_type_get@\cfunc{spec\_type\_get}{}}
\citem{\cfunc[cw\_sint32\_t]{spec\_type\_get}{const char *a\_spec, cw\_uint32\_t
a\_spec\_len, const char **r\_val}}
	\begin{capilist}
	\item[Input(s): ]
		\begin{description}\item[]
		\item[a\_spec: ]
			Pointer to a specifier string.
		\item[a\_spec\_len: ]
			Length of \cvar{a\_spec}.
		\item[r\_val: ]
			Pointer to a string pointer.
		\end{description}
	\item[Output(s): ]
		\begin{description}\item[]
		\item[retval: ]
			Length of type value string.
		\item[*r\_val: ]
			Pointer to type value string.
		\end{description}
	\item[Exception(s): ] None.
	\item[Description: ]
		Return the length of the type value string, and set
		\cvar{*r\_val} to point to it.
	\end{capilist}
\label{spec_val_get}
\index{spec_val_get@\cfunc{spec\_val\_get}{}}
\citem{\cfunc[cw\_sint32\_t]{spec\_val\_get}{const char *a\_spec, cw\_uint32\_t
a\_spec\_len, const char *a\_name, cw\_uint32\_t a\_name\_len, const
cw\_uint8\_t **r\_val}}
	\begin{capilist}
	\item[Input(s): ]
		\begin{description}\item[]
		\item[a\_spec: ]
			Pointer to a specifier string.
		\item[a\_spec\_len: ]
			Length of \cvar{a\_spec}.
		\item[a\_name: ]
			Pointer to a NULL-terminated name.
		\item[a\_name\_len: ]
			Length of \cvar{a\_name}.
		\item[r\_val: ]
			Pointer to a string pointer.
		\end{description}
	\item[Output(s): ]
		\begin{description}\item[]
		\item[retval: ]
			\begin{description}\item[]
			\item[{\gt}= 0: ]
				Length of value string.
			\item[-1: ]
				No such name.
			\end{description}
		\item[*r\_val: ]
			Pointer to value string.
		\end{description}
	\item[Exception(s): ] None.
	\item[Description: ]
		Return the length of the value string that corresponds to
		\cvar{a\_spec}, and set \cvar{*r\_val} to point to it.
	\end{capilist}
\end{capi}

%-*-mode:latex-*-
%%%%%%%%%%%%%%%%%%%%%%%%%%%%%%%%%%%%%%%%%%%%%%%%%%%%%%%%%%%%%%%%%%%%%%%%%%%%%
%
% <Copyright = jasone>
% <License>
%
%%%%%%%%%%%%%%%%%%%%%%%%%%%%%%%%%%%%%%%%%%%%%%%%%%%%%%%%%%%%%%%%%%%%%%%%%%%%%
%
% Version: <Version>
%
% pool portion of Onyx Manual.
%              
%%%%%%%%%%%%%%%%%%%%%%%%%%%%%%%%%%%%%%%%%%%%%%%%%%%%%%%%%%%%%%%%%%%%%%%%%%%%%

\subsection{pool}
\label{pool}
\index{pool@\classname{pool}{}}

The \classname{pool} class provides cached allocation for equal-size buffers.
Each allocation is done via \cfunc{mem\_malloc}{}, and deallocated buffers are
cached.  The cache can be discarded via \cfunc{pool\_drain}{}.

In the debug version of \libname{libstash}, tracking information is kept so that
memory leaks and other memory allocation errors can be detected and printed to
\cvar{out\_err}.

\subsubsection{API}
\begin{capi}
\label{pool_new}
\index{pool_new@\cfunc{pool\_new}{}}
\citem{\cfunc[cw\_pool\_t *]{pool\_new}{cw\_pool\_t *a\_pool, cw\_mem\_t
*a\_mem, cw\_uint32\_t a\_buffer\_size}}
	\begin{capilist}
	\item[Input(s): ]
		\begin{description}\item[]
		\item[a\_pool: ]
			Pointer to space for a \classname{pool}, or NULL.
		\item[a\_mem: ]
			Pointer to the allocator to use internally, or NULL.
		\item[a\_buffer\_size: ]
			Size of buffers to allocate and return from
			\cfunc{pool\_get}{}.
		\end{description}
	\item[Output(s): ]
		\begin{description}\item[]
		\item[retval: ]
			Pointer to a \classname{pool}.
		\end{description}
	\item[Exception(s): ]
		\begin{description}\item[]
		\item[\htmlref{\_CW\_STASHX\_OOM}{_CW_STASHX_OOM}.]
		\end{description}
	\item[Description: ]
		Constructor.
	\end{capilist}
\label{pool_delete}
\index{pool_delete@\cfunc{pool\_delete}{}}
\citem{\cfunc[void]{pool\_delete}{cw\_pool\_t *a\_pool}}
	\begin{capilist}
	\item[Input(s): ]
		\begin{description}\item[]
		\item[a\_pool: ]
			Pointer to a \classname{pool}.
		\end{description}
	\item[Output(s): ] None.
	\item[Exception(s): ] None.
	\item[Description: ]
		Destructor.
	\end{capilist}
\label{pool_buffer_size_get}
\index{pool_buffer_size_get@\cfunc{pool\_buffer\_size\_get}{}}
\citem{\cfunc[cw\_uint32\_t]{pool\_size\_get}{cw\_pool\_t *a\_pool}}
	\begin{capilist}
	\item[Input(s): ]
		\begin{description}\item[]
		\item[a\_pool: ]
			Pointer to a \classname{pool}.
		\end{description}
	\item[Output(s): ]
		\begin{description}\item[]
		\item[retval: ]
			Size of buffers that \cvar{a\_pool} is using.
		\end{description}
	\item[Exception(s): ] None.
	\item[Description: ]
		Return the size of the buffers that \cvar{a\_pool} is using.
	\end{capilist}
\label{pool_drain}
\index{pool_drain@\cfunc{pool\_drain}{}}
\citem{\cfunc[void]{pool\_drain}{cw\_pool\_t *a\_pool}}
	\begin{capilist}
	\item[Input(s): ]
		\begin{description}\item[]
		\item[a\_pool: ]
			Pointer to a \classname{pool}.
		\end{description}
	\item[Output(s): ] None.
	\item[Exception(s): ] None.
	\item[Description: ]
		Discard internally cached buffers, if any.
	\end{capilist}
\label{pool_get_e}
\index{pool_get_e@\cfunc{pool\_get\_e}{}}
\citem{\cfunc[void *]{pool\_get\_e}{cw\_pool\_t *a\_pool, const char
*a\_filename, cw\_uint32\_t a\_line\_num}}
\label{pool_get}
\index{pool_get@\cppmacro{pool\_get}{}}
\citem{\cppmacro[void *]{pool\_get}{cw\_pool\_t *a\_pool}}
	\begin{capilist}
	\item[Input(s): ]
		\begin{description}\item[]
		\item[a\_pool: ]
			Pointer to a \classname{pool}.
		\item[a\_filename: ]
			Should be \_\_FILE\_\_.
		\item[a\_line\_num: ]
			Should be \_\_LINE\_\_.
		\end{description}
	\item[Output(s): ]
		\begin{description}\item[]
		\item[retval: ]
			Pointer to a memory buffer.
		\end{description}
	\item[Exception(s): ]
		\begin{description}\item[]
		\item[\htmlref{\_CW\_STASHX\_OOM}{_CW_STASHX_OOM}.]
		\end{description}
	\item[Description: ]
		Allocate a memory buffer and return a pointer to it.
	\end{capilist}
\label{pool_put_e}
\index{pool_put_e@\cfunc{pool\_put\_e}{}}
\citem{\cfunc[void]{pool\_put\_e}{cw\_pool\_t *a\_pool, void *a\_buffer, const
char *a\_filename, cw\_uint32\_t a\_line\_num}}
\label{pool_put}
\index{pool_put@\cppmacro{pool\_put}{}}
\citem{\cppmacro[void]{pool\_put}{cw\_pool\_t *a\_pool, void *a\_buffer}}
	\begin{capilist}
	\item[Input(s): ]
		\begin{description}\item[]
		\item[a\_pool: ]
			Pointer to a \classname{pool}.
		\item[a\_buffer: ]
			Pointer to a memory buffer.
		\item[a\_filename: ]
			Should be \_\_FILE\_\_.
		\item[a\_line\_num: ]
			Should be \_\_LINE\_\_.
		\end{description}
	\item[Output(s): ] None.
	\item[Exception(s): ] None.
	\item[Description: ]
		Put back (deallocate) \cvar{a\_buffer}.
	\end{capilist}
\label{pool_dump}
\index{pool_dump@\cfunc{pool\_dump}{}}
\citem{\cfunc[void]{pool\_dump}{cw\_pool\_t *a\_pool, const char *a\_prefix}}
	\begin{capilist}
	\item[Input(s): ]
		\begin{description}\item[]
		\item[a\_pool: ]
			Pointer to a \classname{pool}.
		\item[a\_prefix: ]
			Pointer to a string which is used as a prefix for all
			lines of output.
		\end{description}
	\item[Output(s): ]
		Output printed to \cvar{out\_err}.
	\item[Exception(s): ]
		\begin{description}\item[]
		\item[\htmlref{\_CW\_STASHX\_OOM}{_CW_STASHX_OOM}.]
		\end{description}
	\item[Description: ]
		Dump the internal state of \cvar{a\_pool} to \cvar{out\_err}.
	\end{capilist}
\end{capi}

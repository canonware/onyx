%-*-mode:latex-*-
%%%%%%%%%%%%%%%%%%%%%%%%%%%%%%%%%%%%%%%%%%%%%%%%%%%%%%%%%%%%%%%%%%%%%%%%%%%%%
%
% <Copyright = jasone>
% <License>
%
%%%%%%%%%%%%%%%%%%%%%%%%%%%%%%%%%%%%%%%%%%%%%%%%%%%%%%%%%%%%%%%%%%%%%%%%%%%%%
%
% Version: Onyx <Version = onyx>
%
% qs portion of Onyx Manual.
%              
%%%%%%%%%%%%%%%%%%%%%%%%%%%%%%%%%%%%%%%%%%%%%%%%%%%%%%%%%%%%%%%%%%%%%%%%%%%%%

\subsection{qs}
\label{qs}
\index{qs@\classname{qs}{}}

The \classname{qs} macros implement operations on a stack.  The type of the
stack elements and which field of the elements to use are determined by
arguments that are passed into the macros.  The macros are optimized for speed
and code size, which means that there is minimal error checking built in.  As a
result, care must be taken to assure that these macros are used as intended, or
strange things can happen.

\subsubsection{API}
\begin{capi}
\label{qs_head}
\index{qs_head@\cppmacro{qs\_head}{}}
\citem{\cppmacro[]{qs\_head}{{\lt}qs\_type{\gt} a\_type}}
	\begin{capilist}
	\item[Input(s): ]
		\begin{description}\item[]
		\item[a\_type: ]
			Data type for the \classname{qs}.
		\end{description}
	\item[Output(s): ]
		A data structure that can be used as a \classname{qs} head.
	\item[Exception(s): ] None.
	\item[Description: ]
		Generate code for a \classname{qs} head data structure.
	\end{capilist}
\label{qs_head_initializer}
\index{qs_head_initializer@\cppmacro{qs\_head\_initializer}{}}
\citem{\cppmacro[]{qs\_head\_initializer}{{\lt}qs\_type{\gt} *a\_head}}
	\begin{capilist}
	\item[Input(s): ]
		\begin{description}\item[]
		\item[a\_head: ]
			Pointer to a \classname{qs} head.
		\end{description}
	\item[Output(s): ] None.
	\item[Exception(s): ] None.
	\item[Description: ]
		Statically initialize a \classname{qs} head.
	\end{capilist}
\label{qs_elm}
\index{qs_elm@\cppmacro{qs\_elm}{}}
\citem{\cppmacro[]{qs\_elm}{{\lt}qs\_elm\_type{\gt} a\_type}}
	\begin{capilist}
	\item[Input(s): ]
		\begin{description}\item[]
		\item[a\_type: ]
			Data type for the \classname{qs} elements.
		\end{description}
	\item[Output(s): ]
		A data structure that can be used as a qs element.
	\item[Exception(s): ] None.
	\item[Description: ]
		Generate code for a \classname{qs} element data structure.
	\end{capilist}
\label{qs_new}
\index{qs_new@\cppmacro{qs\_new}{}}
\citem{\cppmacro[void]{qs\_new}{{\lt}qs\_type{\gt} *a\_head}}
	\begin{capilist}
	\item[Input(s): ]
		\begin{description}\item[]
		\item[a\_head: ]
			Pointer to a \classname{qs} head.
		\end{description}
	\item[Output(s): ] None.
	\item[Exception(s): ] None.
	\item[Description: ]
		Constructor.
	\end{capilist}
\label{qs_elm_new}
\index{qs_elm_new@\cppmacro{qs\_elm\_new}{}}
\citem{\cppmacro[void]{qs\_elm\_new}{{\lt}qs\_elm\_type{\gt} *a\_elm,
{\lt}field\_name{\gt} a\_field}}
	\begin{capilist}
	\item[Input(s): ]
		\begin{description}\item[]
		\item[a\_head: ]
			Pointer to a \classname{qs} element.
		\item[a\_field: ]
			Field within the \classname{qs} elements to use.
		\end{description}
	\item[Output(s): ] None.
	\item[Exception(s): ] None.
	\item[Description: ]
		Constructor.
	\end{capilist}
\label{qs_top}
\index{qs_top@\cppmacro{qs\_top}{}}
\citem{\cppmacro[{\lt}qs\_type{\gt} *]{qs\_top}{{\lt}qs\_type{\gt} *a\_head}}
	\begin{capilist}
	\item[Input(s): ]
		\begin{description}\item[]
		\item[a\_head: ]
			Pointer to a \classname{qs} head.
		\end{description}
	\item[Output(s): ]
		\begin{description}\item[]
		\item[retval: ]
			Pointer to the top element in the \classname{qs}.
		\end{description}
	\item[Exception(s): ] None.
	\item[Description: ]
		Return a pointer to the top element in the \classname{qs}.
	\end{capilist}
\label{qs_down}
\index{qs_down@\cppmacro{qs\_down}{}}
\citem{\cppmacro[{\lt}qs\_type{\gt} *]{qs\_down}{{\lt}qs\_elm\_type{\gt}
*a\_elm, {\lt}field\_name{\gt} a\_field}}
	\begin{capilist}
	\item[Input(s): ]
		\begin{description}\item[]
		\item[a\_elm: ]
			Pointer to a \classname{qs} element.
		\item[a\_field: ]
			Field within the \classname{qs} elements to use.
		\end{description}
	\item[Output(s): ]
		\begin{description}\item[]
		\item[retval: ]
			\begin{description}\item[]
			\item[non-NULL: ]
				Pointer to the next element in the
				\classname{qs}.
			\item[NULL: ]
				\cvar{a\_elm} is the bottom element in the
				\classname{qs}.
			\end{description}
		\end{description}
	\item[Exception(s): ] None.
	\item[Description: ]
		Return a pointer to the next element in the \classname{qs} below
		\cvar{a\_elm}.
	\end{capilist}
\label{qs_push}
\index{qs_push@\cppmacro{qs\_push}{}}
\citem{\cppmacro[void]{qs\_push}{{\lt}qs\_type{\gt} *a\_head,
{\lt}qs\_elm\_type{\gt} *a\_elm, {\lt}field\_name{\gt} a\_field}}
	\begin{capilist}
	\item[Input(s): ]
		\begin{description}\item[]
		\item[a\_head: ]
			Pointer to a \classname{qs} head.
		\item[a\_elm: ]
			Pointer to an element.
		\item[a\_field: ]
			Field within the \classname{qs} elements to use.
		\end{description}
	\item[Output(s): ] None.
	\item[Exception(s): ] None.
	\item[Description: ]
		Push \cvar{a\_elm} onto the \classname{qs}.
	\end{capilist}
\label{qs_under_push}
\index{qs_under_push@\cppmacro{qs\_under\_push}{}}
\citem{\cppmacro[void]{qs\_under\_push}{{\lt}qs\_elm\_type{\gt} *a\_qselm,
{\lt}qs\_elm\_type{\gt} *a\_elm, {\lt}field\_name{\gt} a\_field}}
	\begin{capilist}
	\item[Input(s): ]
		\begin{description}\item[]
		\item[a\_qselm: ]
			Pointer to a \classname{qs} element.
		\item[a\_elm: ]
			Pointer to an element.
		\item[a\_field: ]
			Field within the \classname{qs} elements to use.
		\end{description}
	\item[Output(s): ] None.
	\item[Exception(s): ] None.
	\item[Description: ]
		Push \cvar{a\_elm} under \cvar{a\_qselm}.
	\end{capilist}
\label{qs_pop}
\index{qs_pop@\cppmacro{qs\_pop}{}}
\citem{\cppmacro[void]{qs\_pop}{{\lt}qs\_type{\gt} *a\_head,
{\lt}field\_name{\gt} a\_field}}
	\begin{capilist}
	\item[Input(s): ]
		\begin{description}\item[]
		\item[a\_head: ]
			Pointer to a \classname{qs} head.
		\item[a\_field: ]
			Field within the \classname{qs} elements to use.
		\end{description}
	\item[Output(s): ] None.
	\item[Exception(s): ] None.
	\item[Description: ]
		Pop an element off of \cvar{a\_head}.
	\end{capilist}
\label{qs_foreach}
\index{qs_foreach@\cppmacro{qs\_foreach}{}}
\citem{\cppmacro[]{qs\_foreach}{{\lt}qs\_elm\_type{\gt} *a\_var,
{\lt}qs\_type{\gt} *a\_head, {\lt}field\_name{\gt} a\_field}}
	\begin{capilist}
	\item[Input(s): ]
		\begin{description}\item[]
		\item[a\_var: ]
			The name of a temporary variable to use for iteration.
		\item[a\_head: ]
			Pointer to a \classname{qs} head.
		\item[a\_field: ]
			Field within the \classname{qs} elements to use.
		\end{description}
	\item[Output(s): ] None.
	\item[Exception(s): ] None.
	\item[Description: ]
		Iterate down the \classname{qs}, storing a pointer to each
		element in \cvar{a\_var} along the way.
	\end{capilist}
\end{capi}

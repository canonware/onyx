%-*-mode:latex-*-
%%%%%%%%%%%%%%%%%%%%%%%%%%%%%%%%%%%%%%%%%%%%%%%%%%%%%%%%%%%%%%%%%%%%%%%%%%%%%
%
% <Copyright = jasone>
% <License>
%
%%%%%%%%%%%%%%%%%%%%%%%%%%%%%%%%%%%%%%%%%%%%%%%%%%%%%%%%%%%%%%%%%%%%%%%%%%%%%
%
% Version: Onyx <Version = onyx>
%
% nxo_thread portion of Onyx Manual.
%              
%%%%%%%%%%%%%%%%%%%%%%%%%%%%%%%%%%%%%%%%%%%%%%%%%%%%%%%%%%%%%%%%%%%%%%%%%%%%%

\subsection{nxo\_thread}
\label{nxo_thread}
\index{nxo_thread@\classname{nxo\_thread}{}}

The \classname{nxo\_thread} class is a subclass of the \classname{nxo} class.

The \classname{threadp} class is a helper class that contains scanner position
information.  The \classname{threadp} state is used when recording syntax
errors.

\subsubsection{API}
\begin{capi}
\label{nxo_threadp_new}
\index{nxo_threadp_new@\cfunc{nxo\_threadp\_new}{}}
\citem{\cfunc[void]{nxo\_threadp\_new}{cw\_nxo\_threadp\_t *a\_threadp}}
	\begin{capilist}
	\item[Input(s): ]
		\begin{description}\item[]
		\item[a\_threadp: ]
			Pointer to space for a \classname{threadp}.
		\end{description}
	\item[Output(s): ] None.
	\item[Exception(s): ] None.
	\item[Description: ]
		Constructor.
	\end{capilist}
\label{nxo_threadp_delete}
\index{nxo_threadp_delete@\cfunc{nxo\_threadp\_delete}{}}
\citem{\cfunc[void]{nxo\_threadp\_delete}{cw\_nxo\_threadp\_t *a\_threadp,
cw\_nxo\_t *a\_thread}}
	\begin{capilist}
	\item[Input(s): ]
		\begin{description}\item[]
		\item[a\_threadp: ]
			Pointer to a \classname{threadp}.
		\item[a\_thread: ]
			Pointer to a thread \classname{nxo}.
		\end{description}
	\item[Output(s): ] None.
	\item[Exception(s): ] None.
	\item[Description: ]
		Destructor.
	\end{capilist}
\label{nxo_threadp_position_get}
\index{nxo_threadp_position_get@\cfunc{nxo\_threadp\_position\_get}{}}
\citem{\cfunc[void]{nxo\_threadp\_position\_get}{const cw\_nxo\_threadp\_t
*a\_threadp, cw\_uint32\_t *r\_line, cw\_uint32\_t *r\_column}}
	\begin{capilist}
	\item[Input(s): ]
		\begin{description}\item[]
		\item[a\_threadp: ]
			Pointer to space for a \classname{threadp}.
		\item[r\_line: ]
			Pointer to a location to store a line number.
		\item[r\_column: ]
			Pointer to a location to store a column number.
		\end{description}
	\item[Output(s): ]
		\begin{description}\item[]
		\item[*r\_line: ]
			Line number.
		\item[*r\_column: ]
			Column number.
		\end{description}
	\item[Exception(s): ] None.
	\item[Description: ]
		Retrieve the line number and column number.
	\end{capilist}
\label{nxo_threadp_position_set}
\index{nxo_threadp_position_set@\cfunc{nxo\_threadp\_position\_set}{}}
\citem{\cfunc[void]{nxo\_threadp\_position\_set}{cw\_nxo\_threadp\_t
*a\_threadp, cw\_uint32\_t a\_line, cw\_uint32\_t a\_column}}
	\begin{capilist}
	\item[Input(s): ]
		\begin{description}\item[]
		\item[a\_threadp: ]
			Pointer to space for a \classname{threadp}.
		\item[a\_line: ]
			Line number.
		\item[a\_column: ]
			Column number.
		\end{description}
	\item[Output(s): ] None.
	\item[Exception(s): ] None.
	\item[Description: ]
		Set the line number and column number.
	\end{capilist}
\label{nxo_thread_new}
\index{nxo_thread_new@\cfunc{nxo\_thread\_new}{}}
\citem{\cfunc[void]{nxo\_thread\_new}{cw\_nxo\_t *a\_nxo, cw\_nx\_t *a\_nx}}
	\begin{capilist}
	\item[Input(s): ]
		\begin{description}\item[]
		\item[a\_nxo: ]
			Pointer to a thread \classname{nxo}.
		\item[a\_nx: ]
			Pointer to an \classname{nx}.
		\end{description}
	\item[Output(s): ] None.
	\item[Exception(s): ]
		\begin{description}\item[]
		\item[\htmlref{CW\_ONYXX\_OOM}{CW_ONYXX_OOM}.]
		\end{description}
	\item[Description: ]
		Constructor.
	\end{capilist}
\label{nxo_thread_start}
\index{nxo_thread_start@\cfunc{nxo\_thread\_start}{}}
\citem{\cfunc[void]{nxo\_thread\_start}{cw\_nxo\_t *a\_nxo}}
	\begin{capilist}
	\item[Input(s): ]
		\begin{description}\item[]
		\item[a\_nxo: ]
			Pointer to a thread \classname{nxo}.
		\end{description}
	\item[Output(s): ] None.
	\item[Exception(s): ] Application dependent.
	\item[Description: ]
		Start a thread running by calling the \onyxop{}{start}{}
		operator such that the top object on ostack will be executed.
	\end{capilist}
\label{nxo_thread_exit}
\index{nxo_thread_exit@\cfunc{nxo\_thread\_exit}{}}
\citem{\cfunc[void]{nxo\_thread\_exit}{cw\_nxo\_t *a\_nxo}}
	\begin{capilist}
	\item[Input(s): ]
		\begin{description}\item[]
		\item[a\_nxo: ]
			Pointer to a thread \classname{nxo}.
		\end{description}
	\item[Output(s): ] None.
	\item[Exception(s): ] None.
	\item[Description: ]
		Terminate the thread.  This has the same effect as a detached
		thread exiting.  Calling this function may is necessary
		(depending on the application) to allow the thread to be garbage
		collected, much the same way as the \onyxop{}{detach}{} and
		\onyxop{}{join}{} operators do.
	\end{capilist}
\label{nxo_thread_thread}
\index{nxo_thread_thread@\cfunc{nxo\_thread\_thread}{}}
\citem{\cfunc[void]{nxo\_thread\_thread}{cw\_nxo\_t *a\_nxo}}
	\begin{capilist}
	\item[Input(s): ]
		\begin{description}\item[]
		\item[a\_nxo: ]
			Pointer to a thread \classname{nxo}.
		\end{description}
	\item[Output(s): ] None.
	\item[Exception(s): ]
		\begin{description}\item[]
		\item[\htmlref{CW\_ONYXX\_OOM}{CW_ONYXX_OOM}.]
		\end{description}
	\item[Description: ]
		Create a new thread.  The new thread calls
		\cfunc{nxo\_thread\_start}{}.
	\end{capilist}
\label{nxo_thread_}
\index{nxo_thread_detach@\cfunc{nxo\_thread\_detach}{}}
\citem{\cfunc[void]{nxo\_thread\_detach}{cw\_nxo\_t *a\_nxo}}
	\begin{capilist}
	\item[Input(s): ]
		\begin{description}\item[]
		\item[a\_nxo: ]
			Pointer to a thread \classname{nxo}.
		\end{description}
	\item[Output(s): ] None.
	\item[Exception(s): ] None.
	\item[Description: ]
		Detach \cvar{a\_nxo} so that when it exits it can be garbage
		collected.
	\end{capilist}
\label{nxo_thread_join}
\index{nxo_thread_join@\cfunc{nxo\_thread\_join}{}}
\citem{\cfunc[void]{nxo\_thread\_join}{cw\_nxo\_t *a\_nxo}}
	\begin{capilist}
	\item[Input(s): ]
		\begin{description}\item[]
		\item[a\_nxo: ]
			Pointer to a thread \classname{nxo}.
		\end{description}
	\item[Output(s): ] None.
	\item[Exception(s): ] None.
	\item[Description: ]
		Wait for \cvar{a\_nxo} to exit.
	\end{capilist}
\label{nxo_thread_state}
\index{nxo_thread_state@\cfunc{nxo\_thread\_state}{}}
\citem{\cfunc[cw\_nxo\_threadts\_t]{nxo\_thread\_state}{const cw\_nxo\_t
*a\_nxo}}
	\begin{capilist}
	\item[Input(s): ]
		\begin{description}\item[]
		\item[a\_nxo: ]
			Pointer to a thread \classname{nxo}.
		\end{description}
	\item[Output(s): ]
		\begin{description}\item[]
		\item[retval: ] The current scanner state of \cvar{a\_nxo}.
			\begin{description}\item[]
			\item[THREADTS\_START: ] Start state.
			\item[THREADTS\_COMMENT: ] '\%' seen, but no line break
				yet.
			\item[THREADTS\_INTEGER: ] Scanning an integer.
			\item[THREADTS\_INTEGER\_RADIX: ] Scanning a radix
				integer.
			\item[THREADTS\_REAL\_FRAC: ] Scanning the fractional
				portion of a real.
			\item[THREADTS\_REAL\_EXP: ] Scanning the exponent
				porttion of a real.
			\item[THREADTS\_STRING: ] Scanning a string.
			\item[THREADTS\_STRING\_NEWLINE\_CONT: ] '{\bs}r' seen
				in a string.
			\item[THREADTS\_STRING\_PROT\_CONT: ] '{\bs}{\bs}' seen
				in a string.
			\item[THREADTS\_STRING\_CRLF\_CONT: ] '{\bs}' '{\bs}r'
				seen in a string.
			\item[THREADTS\_STRING\_CTRL\_CONT: ] '{\bs}' 'c' seen
				in a string.
			\item[THREADTS\_STRING\_HEX\_CONT: ] '{\bs}' 'x' seen in
				a string.
			\item[THREADTS\_STRING\_HEX\_FINISH: ] First hex digit
				of a ``{\bs}xDD'' string escape sequence seen.
			\item[THREADTS\_NAME\_START: ] '!', '\$', or '{\twid}'
				seen.
			\item[THREADTS\_NAME: ] Scanning a name.
			\end{description}
		\end{description}
	\item[Exception(s): ] None.
	\item[Description: ]
		Return the current scanner state.  In general this is only
		useful when implementing an interactive environment for which
		the prompt behaves differently depending on what state the
		scanner is in.  For example the interactive \binname{onyx} shell
		needs only to know whether the scanner is in the start state.
	\end{capilist}
\label{nxo_thread_deferred}
\index{nxo_thread_deferred@\cfunc{nxo\_thread\_deferred}{}}
\citem{\cfunc[cw\_bool\_t]{nxo\_thread\_deferred}{cw\_nxo\_t *a\_nxo}}
	\begin{capilist}
	\item[Input(s): ]
		\begin{description}\item[]
		\item[a\_nxo: ]
			Pointer to a thread \classname{nxo}.
		\end{description}
	\item[Output(s): ]
		\begin{description}\item[]
		\item[retval: ]
			\begin{description}\item[]
			\item[FALSE: ]
				Execution is not deferred.
			\item[TRUE: ]
				Execution is deferred.
			\end{description}
		\end{description}
	\item[Exception(s): ] None.
	\item[Description: ]
		Return whether the scanner is currently in deferred execution
		mode.  See Section~\ref{sec:onyx_syntax} for information on
		deferred execution.  In general this is only useful when
		implementing an interactive environment for which the prompt
		behaves differently depending on what state the scanner is in.
	\end{capilist}
\label{nxo_thread_reset}
\index{nxo_thread_reset@\cfunc{nxo\_thread\_reset}{}}
\citem{\cfunc[void]{nxo\_thread\_reset}{cw\_nxo\_t *a\_nxo}}
	\begin{capilist}
	\item[Input(s): ]
		\begin{description}\item[]
		\item[a\_nxo: ]
			Pointer to a thread \classname{nxo}.
		\end{description}
	\item[Output(s): ] None.
	\item[Exception(s): ] None.
	\item[Description: ]
		Reset the scanner to the start state, and turn deferral off.
		This is a dangerous feature that should be used with great
		care.  \classname{nxo\_no} objects should never be visible from
		inside the interpreter, so the caller must assure that any
		\classname{nxo\_no} objects are removed before further
		processing is done in the context of \cvar{a\_nxo}.
	\end{capilist}
\label{nxo_thread_loop}
\index{nxo_thread_loop@\cfunc{nxo\_thread\_loop}{}}
\citem{\cfunc[void]{nxo\_thread\_loop}{cw\_nxo\_t *a\_nxo}}
	\begin{capilist}
	\item[Input(s): ]
		\begin{description}\item[]
		\item[a\_nxo: ]
			Pointer to a thread \classname{nxo}.
		\end{description}
	\item[Output(s): ] None.
	\item[Exception(s): ] Application specific.
	\item[Description: ]
		Execute the top object on estack.  The caller is responsible
		for placing the object on estack, but it is removed before this
		function returns.
	\end{capilist}
\label{nxo_thread_interpret}
\index{nxo_thread_interpret@\cfunc{nxo\_thread\_interpret}{}}
\citem{\cfunc[void]{nxo\_thread\_interpret}{cw\_nxo\_t *a\_nxo,
cw\_nxo\_threadp\_t *a\_threadp, const cw\_uint8\_t *a\_str, cw\_uint32\_t
a\_len}}
	\begin{capilist}
	\item[Input(s): ]
		\begin{description}\item[]
		\item[a\_nxo: ]
			Pointer to a thread \classname{nxo}.
		\item[a\_threadp: ]
			A \classname{threadp}.
		\item[a\_str: ]
			Pointer to a string to interpret.
		\item[a\_len: ]
			Length in bytes of \cvar{a\_str}.
		\end{description}
	\item[Output(s): ] None.
	\item[Exception(s): ] Application specific.
	\item[Description: ]
		Interpret the string pointed to by \cvar{a\_str}.
	\end{capilist}
\label{nxo_thread_flush}
\index{nxo_thread_flush@\cfunc{nxo\_thread\_flush}{}}
\citem{\cfunc[void]{nxo\_thread\_flush}{cw\_nxo\_t *a\_nxo, cw\_nxo\_threadp\_t
*a\_threadp}}
	\begin{capilist}
	\item[Input(s): ]
		\begin{description}\item[]
		\item[a\_nxo: ]
			Pointer to a thread \classname{nxo}.
		\item[a\_threadp: ]
			A \classname{threadp}.
		\end{description}
	\item[Output(s): ] None.
	\item[Exception(s): ] Application specific.
	\item[Description: ]
		Do the equivalent of interpreting a carriage return in order to
		force acceptance of the previous token if no whitespace has yet
		followed.
	\end{capilist}
\label{nxo_thread_nerror}
\index{nxo_thread_nerror@\cfunc{nxo\_thread\_nerror}{}}
\citem{\cfunc[void]{nxo\_thread\_nerror}{cw\_nxo\_t *a\_nxo, cw\_nxn\_t a\_nxn}}
	\begin{capilist}
	\item[Input(s): ]
		\begin{description}\item[]
		\item[a\_nxo: ]
			Pointer to a thread \classname{nxo}.
		\item[a\_nxn: ]
			An nxn corresponding to the name of an error.
		\end{description}
	\item[Output(s): ] None.
	\item[Exception(s): ] Application dependent.
	\item[Description: ]
		Throw an error.
	\end{capilist}
\label{nxo_thread_serror}
\index{nxo_thread_serror@\cfunc{nxo\_thread\_serror}{}}
\citem{\cfunc[void]{nxo\_thread\_serror}{cw\_nxo\_t *a\_nxo, const cw\_uint8\_t
a\_str, cw\_uint32\_t a\_len}}
	\begin{capilist}
	\item[Input(s): ]
		\begin{description}\item[]
		\item[a\_nxo: ]
			Pointer to a thread \classname{nxo}.
		\item[a\_str: ]
			Pointer to a string that represents the name of an
			error.
		\item[a\_len: ]
			The length of \cvar{a\_str}.
		\end{description}
	\item[Output(s): ] None.
	\item[Exception(s): ] Application dependent.
	\item[Description: ]
		Throw an error.
	\end{capilist}
\label{nxo_thread_dstack_search}
\index{nxo_thread_dstack_search@\cfunc{nxo\_thread\_dstack\_search}{}}
\citem{\cfunc[cw\_bool\_t]{nxo\_thread\_dstack\_search}{cw\_nxo\_t *a\_nxo,
cw\_nxo\_t *a\_key, cw\_nxo\_t *r\_value}}
	\begin{capilist}
	\item[Input(s): ]
		\begin{description}\item[]
		\item[a\_nxo: ]
			Pointer to a thread \classname{nxo}.
		\item[a\_key: ]
			Pointer to an \classname{nxo}.
		\item[r\_value: ]
			Pointer to an \classname{nxo}.
		\end{description}
	\item[Output(s): ]
		\begin{description}\item[]
		\item[retval: ]
			\begin{description}\item[]
			\item[FALSE: ]
				Success.
			\item[TRUE: ]
				\cvar{a\_key} not found on dstack.
			\end{description}
		\item[r\_value: ]
			Top value in dstack associated with \cvar{a\_key}.
		\end{description}
	\item[Exception(s): ] None.
	\item[Description: ]
		Search dstack for the topmost definition of \cvar{a\_key} and
		dup its value to \cvar{r\_value}.
	\end{capilist}
\label{nxo_thread_currentlocking}
\index{nxo_thread_currentlocking@\cfunc{nxo\_thread\_currentlocking}{}}
\citem{\cfunc[cw\_bool\_t]{nxo\_thread\_currentlocking}{const cw\_nxo\_t
*a\_nxo}}
	\begin{capilist}
	\item[Input(s): ]
		\begin{description}\item[]
		\item[a\_nxo: ]
			Pointer to a thread \classname{nxo}.
		\end{description}
	\item[Output(s): ]
		\begin{description}\item[]
		\item[retval: ]
			\begin{description}\item[]
			\item[FALSE: ]
				Implicit locking deactivated for new objects.
			\item[TRUE: ]
				Implicit locking activated for new objects.
			\end{description}
		\end{description}
	\item[Exception(s): ] None.
	\item[Description: ]
		Return whether implicit locking is activated for new objects.
	\end{capilist}
\label{nxo_thread_setlocking}
\index{nxo_thread_setlocking@\cfunc{nxo\_thread\_setlocking}{}}
\citem{\cfunc[void]{nxo\_thread\_setlocking}{cw\_nxo\_t *a\_nxo, cw\_bool\_t
a\_locking}}
	\begin{capilist}
	\item[Input(s): ]
		\begin{description}\item[]
		\item[a\_nxo: ]
			Pointer to a thread \classname{nxo}.
		\item[a\_locking: ]
			\begin{description}\item[]
			\item[FALSE: ]
				Do not implicitly lock new objects.
			\item[TRUE: ]
				Implicitly lock new objects.
			\end{description}
		\end{description}
	\item[Output(s): ] None.
	\item[Exception(s): ] None.
	\item[Description: ]
		Activate or deactivate implicit locking for new objects.
	\end{capilist}
\label{nxo_thread_nx_get}
\index{nxo_thread_nx_get@\cfunc{nxo\_thread\_nx\_get}{}}
\citem{\cfunc[cw\_nx\_t *]{nxo\_thread\_nx\_get}{cw\_nxo\_t *a\_nxo}}
	\begin{capilist}
	\item[Input(s): ]
		\begin{description}\item[]
		\item[a\_nxo: ]
			Pointer to a thread \classname{nxo}.
		\end{description}
	\item[Output(s): ]
		\begin{description}\item[]
		\item[retval: ]
			Pointer to an \classname{nx}.
		\end{description}
	\item[Exception(s): ] None.
	\item[Description: ]
		Return the \classname{nx} associated with \cvar{a\_nxo}.
	\end{capilist}
\label{nxo_thread_userdict_get}
\index{nxo_thread_userdict_get@\cfunc{nxo\_thread\_userdict\_get}{}}
\citem{\cfunc[cw\_nxo\_t *]{nxo\_thread\_userdict\_get}{cw\_nxo\_t *a\_nxo}}
	\begin{capilist}
	\item[Input(s): ]
		\begin{description}\item[]
		\item[a\_nxo: ]
			Pointer to a thread \classname{nxo}.
		\end{description}
	\item[Output(s): ]
		\begin{description}\item[]
		\item[retval: ]
			Pointer to an \classname{nxo} that can safely be used
			without risk of being garbage collected.
		\end{description}
	\item[Exception(s): ] None.
	\item[Description: ]
		Return a pointer to the userdict associated with \cvar{a\_nxo}.
	\end{capilist}
\label{nxo_thread_errordict_get}
\index{nxo_thread_errordict_get@\cfunc{nxo\_thread\_errordict\_get}{}}
\citem{\cfunc[cw\_nxo\_t *]{nxo\_thread\_errordict\_get}{cw\_nxo\_t *a\_nxo}}
	\begin{capilist}
	\item[Input(s): ]
		\begin{description}\item[]
		\item[a\_nxo: ]
			Pointer to a thread \classname{nxo}.
		\end{description}
	\item[Output(s): ]
		\begin{description}\item[]
		\item[retval: ]
			Pointer to an \classname{nxo} that can safely be used
			without risk of being garbage collected.
		\end{description}
	\item[Exception(s): ] None.
	\item[Description: ]
		Return a pointer to the errordict associated with \cvar{a\_nxo}.
	\end{capilist}
\label{nxo_thread_currenterror_get}
\index{nxo_thread_currenterror_get@\cfunc{nxo\_thread\_currenterror\_get}{}}
\citem{\cfunc[cw\_nxo\_t *]{nxo\_thread\_currenterror\_get}{cw\_nxo\_t *a\_nxo}}
	\begin{capilist}
	\item[Input(s): ]
		\begin{description}\item[]
		\item[a\_nxo: ]
			Pointer to a thread \classname{nxo}.
		\end{description}
	\item[Output(s): ]
		\begin{description}\item[]
		\item[retval: ]
			Pointer to an \classname{nxo} that can safely be used
			without risk of being garbage collected.
		\end{description}
	\item[Exception(s): ] None.
	\item[Description: ]
		Return a pointer to the currenterror associated with
		\cvar{a\_nxo}.
	\end{capilist}
\label{nxo_thread_ostack_get}
\index{nxo_thread_ostack_get@\cfunc{nxo\_thread\_ostack\_get}{}}
\citem{\cfunc[cw\_nxo\_t *]{nxo\_thread\_ostack\_get}{cw\_nxo\_t *a\_nxo}}
	\begin{capilist}
	\item[Input(s): ]
		\begin{description}\item[]
		\item[a\_nxo: ]
			Pointer to a thread \classname{nxo}.
		\end{description}
	\item[Output(s): ]
		\begin{description}\item[]
		\item[retval: ]
			Pointer to an \classname{nxo} that can safely be used
			without risk of being garbage collected.
		\end{description}
	\item[Exception(s): ] None.
	\item[Description: ]
		Return a pointer to the ostack associated with \cvar{a\_nxo}.
	\end{capilist}
\label{nxo_thread_dstack_get}
\index{nxo_thread_dstack_get@\cfunc{nxo\_thread\_dstack\_get}{}}
\citem{\cfunc[cw\_nxo\_t *]{nxo\_thread\_dstack\_get}{cw\_nxo\_t *a\_nxo}}
	\begin{capilist}
	\item[Input(s): ]
		\begin{description}\item[]
		\item[a\_nxo: ]
			Pointer to a thread \classname{nxo}.
		\end{description}
	\item[Output(s): ]
		\begin{description}\item[]
		\item[retval: ]
			Pointer to an \classname{nxo} that can safely be used
			without risk of being garbage collected.
		\end{description}
	\item[Exception(s): ] None.
	\item[Description: ]
		Return a pointer to the dstack associated with \cvar{a\_nxo}.
	\end{capilist}
\label{nxo_thread_estack_get}
\index{nxo_thread_estack_get@\cfunc{nxo\_thread\_estack\_get}{}}
\citem{\cfunc[cw\_nxo\_t *]{nxo\_thread\_estack\_get}{cw\_nxo\_t *a\_nxo}}
	\begin{capilist}
	\item[Input(s): ]
		\begin{description}\item[]
		\item[a\_nxo: ]
			Pointer to a thread \classname{nxo}.
		\end{description}
	\item[Output(s): ]
		\begin{description}\item[]
		\item[retval: ]
			Pointer to an \classname{nxo} that can safely be used
			without risk of being garbage collected.
		\end{description}
	\item[Exception(s): ] None.
	\item[Description: ]
		Return a pointer to the estack associated with \cvar{a\_nxo}.
	\end{capilist}
\label{nxo_thread_istack_get}
\index{nxo_thread_istack_get@\cfunc{nxo\_thread\_istack\_get}{}}
\citem{\cfunc[cw\_nxo\_t *]{nxo\_thread\_istack\_get}{cw\_nxo\_t *a\_nxo}}
	\begin{capilist}
	\item[Input(s): ]
		\begin{description}\item[]
		\item[a\_nxo: ]
			Pointer to a thread \classname{nxo}.
		\end{description}
	\item[Output(s): ]
		\begin{description}\item[]
		\item[retval: ]
			Pointer to an \classname{nxo} that can safely be used
			without risk of being garbage collected.
		\end{description}
	\item[Exception(s): ] None.
	\item[Description: ]
		Return a pointer to the istack associated with \cvar{a\_nxo}.
	\end{capilist}
\label{nxo_thread_tstack_get}
\index{nxo_thread_tstack_get@\cfunc{nxo\_thread\_tstack\_get}{}}
\citem{\cfunc[cw\_nxo\_t *]{nxo\_thread\_tstack\_get}{cw\_nxo\_t *a\_nxo}}
	\begin{capilist}
	\item[Input(s): ]
		\begin{description}\item[]
		\item[a\_nxo: ]
			Pointer to a thread \classname{nxo}.
		\end{description}
	\item[Output(s): ]
		\begin{description}\item[]
		\item[retval: ]
			Pointer to an \classname{nxo} that can safely be used
			without risk of being garbage collected.
		\end{description}
	\item[Exception(s): ] None.
	\item[Description: ]
		Return a pointer to the tstack associated with \cvar{a\_nxo}.
	\end{capilist}
\label{nxo_thread_stdin_get}
\index{nxo_thread_stdin_get@\cfunc{nxo\_thread\_stdin\_get}{}}
\citem{\cfunc[cw\_nxo\_t *]{nxo\_thread\_stdin\_get}{cw\_nxo\_t *a\_nxo}}
	\begin{capilist}
	\item[Input(s): ]
		\begin{description}\item[]
		\item[a\_nxo: ]
			Pointer to a thread \classname{nxo}.
		\end{description}
	\item[Output(s): ]
		\begin{description}\item[]
		\item[retval: ]
			Pointer to an \classname{nxo} that can safely be used
			without risk of being garbage collected.
		\end{description}
	\item[Exception(s): ] None.
	\item[Description: ]
		Return a pointer to the stdin associated with \cvar{a\_nxo}.
	\end{capilist}
\label{nxo_thread_stdin_set}
\index{nxo_thread_stdin_set@\cfunc{nxo\_thread\_stdin\_set}{}}
\citem{\cfunc[void]{nxo\_thread\_stdin\_set}{cw\_nxo\_t *a\_nxo, cw\_nxo\_t
 *a\_stdin}}
	\begin{capilist}
	\item[Input(s): ]
		\begin{description}\item[]
		\item[a\_nxo: ]
			Pointer to a thread \classname{nxo}.
		\item[a\_stdin: ]
			Pointer to a file \classname{nxo}.
		\end{description}
	\item[Output(s): ] None.
	\item[Exception(s): ] None.
	\item[Description: ]
		Set \cvar{a\_nxo}'s stdin to \cvar{a\_stdin}.
	\end{capilist}
\label{nxo_thread_stdout_get}
\index{nxo_thread_stdout_get@\cfunc{nxo\_thread\_stdout\_get}{}}
\citem{\cfunc[cw\_nxo\_t *]{nxo\_thread\_stdout\_get}{cw\_nxo\_t *a\_nxo}}
	\begin{capilist}
	\item[Input(s): ]
		\begin{description}\item[]
		\item[a\_nxo: ]
			Pointer to a thread \classname{nxo}.
		\end{description}
	\item[Output(s): ]
		\begin{description}\item[]
		\item[retval: ]
			Pointer to an \classname{nxo} that can safely be used
			without risk of being garbage collected.
		\end{description}
	\item[Exception(s): ] None.
	\item[Description: ]
		Return a pointer to the stdout associated with \cvar{a\_nxo}.
	\end{capilist}
\label{nxo_thread_stdout_set}
\index{nxo_thread_stdout_set@\cfunc{nxo\_thread\_stdout\_set}{}}
\citem{\cfunc[void]{nxo\_thread\_stdout\_set}{cw\_nxo\_t *a\_nxo, cw\_nxo\_t
 *a\_stdout}}
	\begin{capilist}
	\item[Input(s): ]
		\begin{description}\item[]
		\item[a\_nxo: ]
			Pointer to a thread \classname{nxo}.
		\item[a\_stdout: ]
			Pointer to a file \classname{nxo}.
		\end{description}
	\item[Output(s): ] None.
	\item[Exception(s): ] None.
	\item[Description: ]
		Set \cvar{a\_nxo}'s stdout to \cvar{a\_stdout}.
	\end{capilist}
\label{nxo_thread_stderr_get}
\index{nxo_thread_stderr_get@\cfunc{nxo\_thread\_stderr\_get}{}}
\citem{\cfunc[cw\_nxo\_t *]{nxo\_thread\_stderr\_get}{cw\_nxo\_t *a\_nxo}}
	\begin{capilist}
	\item[Input(s): ]
		\begin{description}\item[]
		\item[a\_nxo: ]
			Pointer to a thread \classname{nxo}.
		\end{description}
	\item[Output(s): ]
		\begin{description}\item[]
		\item[retval: ]
			Pointer to an \classname{nxo} that can safely be used
			without risk of being garbage collected.
		\end{description}
	\item[Exception(s): ] None.
	\item[Description: ]
		Return a pointer to the stderr associated with \cvar{a\_nxo}.
	\end{capilist}
\label{nxo_thread_stderr_set}
\index{nxo_thread_stderr_set@\cfunc{nxo\_thread\_stderr\_set}{}}
\citem{\cfunc[void]{nxo\_thread\_stderr\_set}{cw\_nxo\_t *a\_nxo, cw\_nxo\_t
 *a\_stderr}}
	\begin{capilist}
	\item[Input(s): ]
		\begin{description}\item[]
		\item[a\_nxo: ]
			Pointer to a thread \classname{nxo}.
		\item[a\_stderr: ]
			Pointer to a file \classname{nxo}.
		\end{description}
	\item[Output(s): ] None.
	\item[Exception(s): ] None.
	\item[Description: ]
		Set \cvar{a\_nxo}'s stderr to \cvar{a\_stderr}.
	\end{capilist}
\end{capi}
